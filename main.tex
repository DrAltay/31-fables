%
% Body text font is Palatino!
%

\documentclass[a5paper,pagesize,10pt,bibliography=totoc,numbers=enddot,
headings=normal,DIV=9,twoside=false,tablecaptionabove]{scrbook}

% twoside, openright
\KOMAoptions{DIV=last}

\usepackage{trajan}
\usepackage{calc} 
\usepackage[french]{babel}
%\usepackage[utf8]{inputenc}
\usepackage[T1]{fontenc}
\usepackage[protrusion=true]{microtype}
\usepackage{multicol}
\usepackage[babel]{csquotes}
% Désactive l'étiquette de la figure dans les légendes
\usepackage[labelformat=empty,skip=10pt,tableposition=above]{caption}
\usepackage[toc]{appendix}
\usepackage{booktabs}
\usepackage{tabularx}
\usepackage{graphicx}
\usepackage{zref-savepos}
\usepackage[table]{xcolor}
\usepackage{xspace}
\usepackage{pifont}
\usepackage{siunitx}
\DeclareSIUnit{\million}{\text{M}}
\usepackage{tcolorbox}
\usepackage[pages=some,firstpage=true]{background}
\backgroundsetup{scale=1,angle=0,opacity=0.8,pages=some,color=black,contents={%
	\includegraphics[width=\paperwidth,height=\paperheight]{images/vintage}
	}%
}

\usepackage{enumitem}
\setlist[itemize]{label=\ding{118},topsep=2pt,itemsep=0pt,partopsep=3pt}

\newcommand\colortablerows{%
	\overlock
	\rowcolors{2}{gray!15}{white}
}%

% Index
\usepackage[splitindex]{imakeidx}
%\usepackage{index}
\makeindex[name=cadre,title=Cadre]
\makeindex[name=genre,title=Genre]
\makeindex[name=theme,title=Thématiques]
%\newindex{cadre}{cax}{cad}{Cadre}
%\newindex{genre}{gex}{ged}{Genre}
%\newindex{theme}{thx}{thd}{Thème}

% Indentation des paragraphes
\setlength{\parindent}{10pt}
% Sauts de ligne entre les paragraphes
\setlength{\parskip}{1.4ex plus 0.35ex minus 0.3ex}
%\setlength{\parskip}{1.4ex plus 0.35ex minus 0.3ex}

% Pas de numérotation au-delà des chapitres
\setcounter{secnumdepth}{\chapternumdepth}

% Police de caractère générale
\usepackage[sc]{mathpazo}
% Hyperref
\usepackage[hidelinks]{hyperref}
\linespread{1.05} 
% Police de caractères des titres
\usepackage{LobsterTwo}
\usepackage{overlock}
\addtokomafont{chapter}{\LobsterTwo}
\addtokomafont{disposition}{\overlock}
\addtokomafont{caption}{\overlock}
\RedeclareSectionCommand[beforeskip=10pt]{chapter}
\RedeclareSectionCommand[beforeskip=3pt,afterskip=3pt]{section}
\RedeclareSectionCommand[beforeskip=3pt,afterskip=1pt]{subsection}
\RedeclareSectionCommand[beforeskip=1pt,afterskip=1pt]{subsubsection}
\setmainfont
     [ BoldFont       = texgyrepagella-bold.otf ,
       ItalicFont     = texgyrepagella-italic.otf ,
       BoldItalicFont = texgyrepagella-bolditalic.otf ]
     {texgyrepagella-regular.otf}

\title{\scshape{31 fables\\\medskip à jouer}}
\author{\scshape{Altay}}
\def\subtitle{Un recueil de scénarios pour jeux de rôle sur table}
\date{2020}

\begin{document}


%=========================================
\makeatletter
\begin{titlepage}
		\centering{
			{\fontsize{40}{48}\selectfont 
			\@title}

			\vspace{2cm}

			{\fontsize{20}{28}\selectfont
			\subtitle}
		}\\
		
		\vfill
		%\vspace{10mm}
		\centering\includegraphics[width=0.8\textwidth]{images/skull}
		\vfill
		\centering{\Large{\@author}}\\
		\vspace{\fill}
		\centering \large{\@date}
\end{titlepage}
\makeatother

%=========================================
\newpage{}
\thispagestyle {empty}

\vspace*{2cm}

%\begin{center}
%	\Large{\parbox{10cm}{
%		\begin{raggedright}
%		{\Large 
%			\textit{Do what you think is interesting, 
%			do something that you think is fun and worthwhile, 
%			because otherwise you won’t do it well anyway.}
%		}
%	
%		\vspace{.5cm}\hfill{---Brian W. Kernighan}
%		\end{raggedright}
%	}
%}
%\end{center}

\newcommand\keywords[3]{%
\begin{tcolorbox}[sharp corners=west, arc=5mm, title=Mots-clés:,fonttitle=\bfseries\large\overlock,boxsep=0mm,toptitle=1mm,bottomtitle=1mm]
%	\section*{Mots-clés :}
	\paragraph{Cadre}~~~~~~~:~~~~ #1\\
	\paragraph{Genre}~~~~~~~:~~~~ #2\\
	\paragraph{Thèmes}~~~~:~~~~ #3
\end{tcolorbox}
}%

\newcommand\medfan[1]{%
	\emph{medfan}\index[cadre]{Médiéval fantastique}
}%
\newcommand\generique[1]{%
	générique\index[cadre]{Générique}
}%
\newcommand\contemporain[1]{%
	contemporain\index[cadre]{Contemporain}
}%
\newcommand\guerrefroide[1]{%
	guerre froide\index[cadre]{Guerre froide}
}%
\newcommand\postapo[1]{%
	post-apo\index[cadre]{Post-apocalyptique}
}%
\newcommand\antique[1]{%
	Antiquité\index[cadre]{Antiquité}
}%
\newcommand\cyberpunk[1]{%
	\emph{cyberpunk}\index[cadre]{Cyberpunk/anticipation}
}%
\newcommand\superheros[1]{%
	super-héroïque\index[cadre]{Super-héroïque}
}%
\newcommand\pirate[1]{%
	pirate\index[cadre]{Piraterie}
}%
\newcommand\scifi[1]{%
	science-fiction\index[cadre]{Science-fiction}
}%
\newcommand\western[1]{%
	\emph{western}\index[cadre]{Western}
}%
\newcommand\steampunk[1]{%
	\emph{Steampunk}\index[cadre]{Steampunk}
}%
\newcommand\retro[1]{%
	\emph{entre-deux-guerres}\index[cadre]{Entre-deux-guerres}
}%

\newcommand\aventure[1]{%
	aventure\index[genre]{Aventure}
}%
\newcommand\action[1]{%
	action\index[genre]{Action}
}%
\newcommand\intrigue[1]{%
	intrigue\index[genre]{Intrigue}
}%
\newcommand\enquete[1]{%
	enquête\index[genre]{Enquête}
}%

\newcommand{\filltopageendgraphics}[2][]{%
  \par
  \zsaveposy{top-\thepage}% Mark (baseline of) top of image
  \vfill
  \zsaveposy{bottom-\thepage}% Mark (baseline of) bottom of image
  \def\thisheight{\dimexpr\zposy{top-\thepage}sp-\zposy{bottom-\thepage}sp\relax}
  %\smash{
  \begin{figure}[h]
	\centering
	\includegraphics[height=\ifdim\thisheight<0.3\textheight \thisheight \else 0.3\textheight \fi,#1]{#2}
  \end{figure}
  %}%
  \par
}%

\newfontfamily\quotefont{Linux Biolinum O}
\AtBeginEnvironment{quote}{\quotefont}

\newcommand\illustration[2][0.5\textwidth]{%
	\begin{figure}[h]
		\centering
		\includegraphics[width=#1]{images/#2}
	\end{figure}
}%

\newpage


%=========================================
\begingroup
\let\clearpage\relax
\begin{figure}[h!]
	\includegraphics[width=\textwidth]{images/books}
\end{figure}
\vspace{-3em}
\chapter*{Préambule}

Cet ouvrage est un ensemble de 31 scénarios divers conçus pour être joués dans des jeux de rôle sur table.
La plupart de ces histoires sont volontairement ouvertes, laissant aux joueurs et joueuses le soin de combler les vides et de préciser les flous par leur imagination.
Grâce à cette liberté d'interprétation, les scénarios ne sont attachés ni à des systèmes spécifiques, ni à des univers particuliers.
Les différentes aventures font donc la part belle à l'improvisation.

Chaque scénario est décrit par trois mots-clés :
\begin{itemize}
	\item le cadre dans lequel il a été imaginé,
	\item le genre d'histoire racontée,
	\item le thème de l'aventure.
\end{itemize}

Un \hyperref[index]{index} en fin d'ouvrage permet de retrouver la liste des scénarios relevant des différents mots-clés.
Ceux-ci sont toutefois à prendre comme des indications et non des obligations.
La plupart des aventures peuvent aisément être transposées d'un cadre à un autre voire d'un genre à un autre.

Les scénarios sont divisés en trois sous-parties: l'accroche, les péripéties et la résolution.
L'accroche, plus ou moins longue, pose la situation initiale dans laquelle se trouvent les personnages.
Les péripéties forment le gros de l'aventure et contient les éléments d'histoire à découvrir ainsi que les rebondissements du scénario.
Enfin, la résolution décrit différentes pistes concernant les façons de conclure l'aventure.

Les protagonistes des différentes histoires ne sont généralement pas nommés pour vous permettre d'y insérer les personnages de votre choix.
Toutefois, à certains scénarios est associée la description d'une entité (un personnage ou un lieu), décrivant en deux lignes l'ambiance de l'endroit, le caractère de cet individu, etc.
À vous de choisir si vous utilisez ces informations, elles ne sont là que pour faciliter la préparation du scénario!

Pour simplifier la lecture et éviter la confusion entre les personnages et les personnes qui les incarnent, nous parlerons des personnages au masculin et des joueuses au féminin.
L'intégralité des protagonistes des histoires qui suivent peuvent néanmoins être du genre que vous souhaitez.


Pour finir, le contenu de cet ouvrage est sous licence \href{https://creativecommons.org/licenses/by-nc-sa/4.0/legalcode.fr}{Creative Commons CC-BY-NC-SA 4.0}. C'est-à-dire que vous pouvez partager ce recueil, le modifier et le diffuser à deux conditions: mentionner l'œuvre originale et ne pas en faire d'usage commercial.

J'espère que vous trouverez votre bonheur dans au moins une des 31 histoires qui suivent.

Bonne lecture!

\vfill
\hfill
\begin{minipage}{0.40\textwidth}
	\includegraphics[width=\textwidth]{images/cc.logo.eps}
\end{minipage}%
\hfill
\begin{minipage}{0.41\textwidth}
	%\includegraphics[width=0.2\textwidth]{images/cc.eps}%
	\includegraphics[width=0.25\textwidth]{images/by.eps}%
	\includegraphics[width=0.25\textwidth]{images/nc.eps}%
	\includegraphics[width=0.25\textwidth]{images/sa.eps}%
	\includegraphics[width=0.25\textwidth]{images/remix.eps}%
\end{minipage}
\hfill
\endgroup

\setcounter{tocdepth}{0}
\tableofcontents

\chapter{L'anneau gardien}
\keywords{\medfan}{\aventure}{\index[theme]{Magie}Magie, \index[theme]{Malédiction}malédiction}

\section{Scénario}

Ce scénario peut être facilement joué en parallèle d'une campagne \emph{medfan}, il suffit qu'un personnage obtienne l'anneau tutélaire lors de l'une de ses aventures.
C'est encore mieux si les joueuses utilisent régulièrement l'anneau de leur propre chef.

\subsection*{Accroche}

Les personnages entrent en possession d'un anneau magique.

\subsection*{Péripéties}

À chaque fois que la personne qui porte l'anneau est en danger de mort, une puissante guerrière se matérialise à proximité pour la tirer de ce mauvais pas.
Une fois le porteur en sécurité, la guerrière disparaît sans un mot et se contente de jeter un regard furieux en direction du groupe.

Chaque invocation semble l'énerver encore plus mais tout effort de lui parler est vain: elle ne parle pas et ne semble de toute façon pas comprendre les personnages.

Au fil du temps, certaines de ses apparitions deviennent étranges.
Une fois, la guerrière apparaît sans ses armes ni son armure. Une autre fois, elle tient un morceau de poulet à moitié mangé dans sa main droite.

Finalement, le jour d'une énième invocation, elle se matérialise avec un bout de papier dans les mains.
Celui-ci est écrit dans une langue étrangère mais après l'avoir déchiffré, le message est le suivant:
\blockquote{L'anneau est maudit. J'ai une famille et une vie. Je n'ai pas demandé à servir d'ange gardien. Le forgeron qui l'a créé est prisonnier des geôles royales. Trouvez-le et faites-lui lever la malédiction. S'il vous plaît.}

En réalité, le forgeron est un sorcier malchanceux fuyant la guerre qui ravage une nation voisine.
Craignant pour sa vie, il a embauché des mercenaires pour l'escorter jusqu'au royaume où se trouvent les personnages mais alors que l'argent est venu à manquer, il s'est retrouvé sans aucune protection.
Pour assurer ses arrières, il n'a alors rien trouvé de mieux pour que de lier l'âme d'une grande aventurière à la retraite -- croisée au hasard de son voyage -- à son anneau.
Ce faisant, il l'a contrainte à devenir la garante de sa sécurité, s'achetant ainsi une certaine tranquillité.

Une fois son périple achevé, le sorcier-forgeron a posé ses valises dans la capitale et s'y est établi comme fabriquant d'objets magiques.
Malheureusement, n'étant pas un bon gestionnaire, il s'est rapidement retrouvé criblé de dettes auprès du royaume, incapable d'honorer les commandes du gouvernement.
La milice l'a alors mis en prison avant de piller son échoppe.
Ses biens furent vendus aux enchères pour éponger ses dettes et, de fil en aiguille, l'anneau a ainsi échappé à son propriétaire.
La guerrière s'accommode tant bien que mal des aventures de ses porteurs successifs, régulièrement importunée dans sa retraite par ces invocations involontaires.

\subsection*{Résolution}

Plusieurs façons de lever le sortilège sont envisageables.
Si vos personnages sont versés en magie, peut-être qu'ils seraient capable de concevoir par eux-mêmes un rituel, à effectuer en présence de la guerrière, qui pourrait briser le lien entre elle et l'anneau.
Ou bien peut-être qu'une solution serait simplement de substituer une nouvelle âme à celle actuellement liée à l'anneau.
Enfin, une troisième solution serait de retrouver la trace du sorcier.
Celui-ci est sûrement prêt à défaire sa magie en échange d'un moyen d'échapper à sa captivité, par exemple en payant ses dettes ou en usant de la force pour le faire s'évader\dots

\subsection*{Personnage: la guerrière}

\newcommand\descriptionperso[5]{%
	\begin{tcolorbox}[colback=black!1!white]
	\begin{tabular}{lcl}
		{\overlock\textbf{Caractère}} & : & #1\\
		{\overlock\textbf{Force}} & : & #2\\
		{\overlock\textbf{Faiblesse}} & : & #3\\
		{\overlock\textbf{Objectif}} & : & #4\\
	\end{tabular}

	\medskip
	#5
	\end{tcolorbox}
}%

\descriptionperso{sûre d'elle, exaspérée}{combat, stratégie}{ne parle pas la langue, maudite}{être tranquille}{Aventurière légendaire d'un pays lointain ayant raccroché les armes pour prendre sa retraite. Liée à un anneau magique, elle est régulièrement importunée par des invocations non-sollicitées}

\illustration[0.16\textwidth]{ring}

\chapter{L'incident du peuplier}
\keywords{Contemporain}{Action}{Militaire, diplomatie, guerre froide}

\section{Scénario}

\subsection{Commentaire}

Ce scénario est inspiré d'un fait réel ayant eu lieu en août 1976 (le \emph{poplar tree incident})\footnote{\url{https://fr.wikipedia.org/wiki/Incident_du_peuplier}} dans la zone démilitarisée (DMZ) séparant la Corée du Nord de la Corée du Sud.
Comme beaucoup d'autres incidents diplomatiques de l'époque, la tension découle en grande partie du contexte de guerre froide entre deux superpuissances.

\subsection{Accroche}

Août 1976. Zone démilitarisée coréenne, section contrôlée par l'ONU. Un groupe de soldats coréens et américains s'apprête à tailler les branches d'un peuplier qui masquent leur ligne de vue sur le \og pont de Non-retour\fg, qui permet aux nord-coréens de traverser la rivière Sachon pour rejoindre leur zone. Quinze minutes plus tard, un camion de soldats nords-coréens arrive. Ils demandent aux onusiens d'arrêter car l'arbre aurait été planté par Kim Il-sung en personne. Puis, devant leur refus, ils attaquent le contingent à coups de hachettes et de gourdins, tuant deux officiers américains et capturant plusieurs soldats.

\subsection{Péripéties}

L'incident enflamme la zone. Les nord-coréens dénoncent une attaque ordonnée par les américains, soutenus par Cuba et de nombreux non-alignés.
La CIA considère elle que l'attaque nord-coréenne était préméditée et les États-Unis passent en DEFCON3.

Les commandement de l'ONU ou l'état-major des États-Unis mobilise les personnages dans le cadre de l'opération \emph{Paul Bunyan}, du nom de légendaire bûcheron américain.

Des sapeurs du corps d'ingiénerie de l'armée de terre sont diligentés pour abattre l'arbre, escortés par un bataillon de soldats américains et épaulés par les forces spéciales coréennes.
En appui de cette démonstration de force, des hélicoptères et des avions de combat sont déployés dans l'espoir d'intimider le régime nord-coréen et de décourager une attaque.
Toutes les forces armées entrent dans la DMZ, impressionnante armada convergeant vers le peuplier à deux pas du pont de non-retour.
Très rapidement, plusieurs bus militaires nord-coréens arrivent sur site en bus.
Des soldats en descendent et déploient des mitrailleuses sur l'autre rive.

L'objectif de l'opération \emph{Paul Bunyan} est simple: abattre le peuplier tout en évitant la guerre.

Le tableau ci-dessous comporte quelques événements aléatoires permettant d'épicer la situation et de maintenir les personnages sur le qui-vive:

\begin{table}
	\begin{tabularx}{0.8\textwidth}{cX}
	d6 & Événement\\
	1  & Les tronçonneuses tombent en panne après quelques minutes. Il faut soit en faire venir des nouvelles, soit finir le travail à la hache.\\
	2  & Un soldat américain s'avance sur le pont et provoque les nord-coréens. Il se trouve qu'il est proche d'un des officiers assassinés.\\
	3  & Des camions nord-coréens s'installent à 150m\dots et construisent un nouveau pont.\\
	4  & Les nord-coréens font s'avancer les prisonniers sur le pont. Ils seront relâchés si les forces impérialistes renoncent, sinon ils seront exécutés.\\
	5  & Un des soldats sud-coréens se comporte étrangement. Il transmet discrètement des informations au régime du nord sur l'avancée de l'opération américaine.\\
	6  & Les renseignements japonais ont intercepté un message nord-coréens : leurs troupes s'apprêteraient à bloquer la rivière Imjin. C'est malheureusement la route prévue par le commandement de l'ONU pour une évacuation par canots en cas d'attaque\dots\\
	7  & Un jeune nord-coréen saute dans la rivière et tente de traverser à la nage. Il appelle à l'aide en anglais et jure qu'il veut faire défection.\\
	8  & Alors que les sapeurs ramassent les branches déjà coupées et les chargent dans un pick-up conformément aux ordres, les sud-coréens leur demandent de les laisser sur place. Compte-tenu du symbolisme de l'arbre, les nords-coréens pourraient prendre cet acte pour une provocation.\\
	\end{tabularx}
\end{table}

\chapter{Leurre de paix}
\keywords{Générique}{Intrigue}{Diplomatie, évasion, trahison}

\section{Scénario}

Ce scénario peut s'adapter à de nombreux cadres de jeux.
L'intrigue tourne autour de trois factions que l'on nommera -- pour simplifier -- la faction Bleue, la faction Verte et la faction Rouge.
La situation initiale est la suivante: les factions Rouge et Verte sont en guerre et la faction Bleue est \og neutre \fg mais profite de la situation d'une manière ou d'une autre.

\subsection{Accroche}

Les personnages sont de jeunes dignitaires envoyés comme émissaires auprès du royaume Vert pour négocier une trêve dans la guerre l'opposant au royaume Rouge.
Ou tout du moins tel est ce que leur ont raconté les ministres.

\subsection{Péripéties}

En effet, la situation est bien plus en leur défaveur que ce qui leur a été dit.
En réalité, l'état-major a déjà acheté la paix.
En guise de bonne foi pour sceller l'accord d'armistice, le gouvernement a décidé d'envoyer quelques enfants de bonne famille qui serviront de \og caution \fg\dots en tant qu'otages.
Les personnages sont ainsi reçus dans le faste et le luxe dû à leur statut mais, une fois la nuit tombée, leur escorte s'éclipse et les laisse à la merci des Verts.
On les réveille au beau milieu de la nuit pour les conduire à leur futur de lieu de captivité.

C'est néanmoins sans compter l'intervention des agents de la faction Bleue qui vont tout faire pour saboter la paix.
Parce que leur faction vend des armes aux deux camps ou que l'affaiblissement des deux autres nations sert leur stratégie militaire à long terme, les Bleus profitent de la situation telle qu'elle est et n'ont aucune envie que le conflit s'éteigne.
Ainsi, des espions et espionnes déguisées en loyalistes Verts -- mais en réalité au service des Bleus -- vont tenter de libérer les otages pour relancer la guerre.
Reste à savoir si les personnages, mis devant le fait accompli, choisiront d'accepter leur sort et se sacrifieront pour entériner l'armistice ou, au contraire, feront des pieds et des mains pour s'échapper, quite à envenimer la situation.
À moins que des négociations à couteaux tirés au milieu des agents doubles -- réels ou soupçonnés -- soient encore possible\dots

\subsection{Résolution}

Plusieurs issues sont envisageables à cette situation en fonction des décisions des personnages:

\subsubsection{Les personnages refusent l'aide des Bleus}
\begin{itemize}
	\item Les personnages refusent l'aide des Bleus et acceptent leur captivité: la paix est achetée, en espérant que la faction Rouge joue le jeu.
	\item Les personnages refusent l'aide des Bleus et s'évadent: la guerre continue, les membres du groupe sont \emph{persona non grata} aux yeux de leur gouvernement.
	\item Les personnages refusent l'aide des Bleus et négocient la paix par eux-mêmes: une trêve est envisageable, surtout si les manigances des Bleus sont exposées au grand jour.
\end{itemize}

\subsubsection{Les personnages acceptent l'aide des Bleus}
\begin{itemize}
	\item Les personnages acceptent l'aide des Bleus mais sont tout de même capturés: la paix est achetée, sauf si la faction Rouge pense que ce sont les Verts qui ont tenté de faire évader le groupe en dépit de leur accord.
	\item Les personnages acceptent l'aide des Bleus et s'évadent: la guerre continue, surtout si les Bleus ont réussi à se faire passer pour les Verts jusqu'au bout.
\end{itemize}

L'idée est de laisser une grande latitude aux joueuses et de permettre tous les retournements de situation.
L'intrigue est un sac de nœuds et de nouvelles alliances de circonstances peuvent très bien apparaître selon ce qui vous arrange le plus!

\subsection*{Lieu: le château d'Awiti}

Situé à la frontière entre les factions Rouge et Vert, le château est l'endroit parfait pour des tractations en terrain \og neutre \fg.
Abandonné depuis des lustres, le château s'élève sur une île au milieu du lac dont il a pris le nom. Cela le rend facile à défendre mais peu commode à ravitailler.
Cet inconvénient est aussi son plus grand atout: cette vieille forteresse regorge d'alcôves cachées, de souterrains méconnus, de passages secrets et de cryptes oubliées pour comploter en toute tranquilité, assassiner un rival ou se déplacer sans être vu.

\illustration{betrayal}

\chapter{Cryo-secours}
\keywords{\scifi}{\enquete}{\index[theme]{Technologie}Technologie, \index[theme]{Évasion}évasion, \index[theme]{Espace}espace}

\section{Scénario}

Ce scénario part de personnages amnésiques qui connaissent leur identité mais ont oublié ce qui les a conduit à l'endroit où ils se trouvent.
L'ambiance doit tourner autour de la mort de l'étoile: la station est de plus en plus sombre au fur et à mesure que les réserves de l'énergie se vident.
La sphère de Dyson est un prétexte pour justifier l'extinction rapide de l'étoile et puis c'est surtout un concept extrêmement cool à faire découvrir à votre table.

\subsection{Accroche}

Les personnages se réveillent d'un sommeil cryogénique; des robots androïdes les aident à émerger et à reprendre leurs esprits.
Leurs souvenirs sont parcellaires, pour ne pas dire inexistants, sur les raisons qui de leur plongée en stase.

\subsection{Péripéties}

Quelques observations évidentes permettent de découvrir que l'endroit est une station spatiale.
Celle-ci est à l'abandon, si ce n'est pour la demi-douzaine de robots d'assistance et l'intelligence virtuelle limitée qui les a maintenus en vie.

Pire encore, la station semble avoir été évacuée lentement sur plusieurs années.
Elle ne fonctionne plus que grâce à une réserve de fuel de secours qui s'est mystérieusement mise en marche, déclenchant au passage le protocole de décryogénisation.
Les docks sont abandonnés et vides: il ne reste ni vaisseau, ni navette, ni capsule de sauvetage.
À vrai dire, quasiment tout ce qui avait de la valeur a été démantelé.
En fouillant dans le journal de bord, en accédant à l'IA centrale ou en examinant les batteries, il devient clair que le générateur s'est mis en branle car les panneaux solaires ne suffisaient plus à les maintenir en cryo indéfiniment.
L'IA a donc déclenché le protocole d'évacuation: puiser dans les dernières réserves pour réveiller les dormeurs et leur permettre de quitter les lieux avant le désorbitage de la station.

La triste nouvelle, c'est que tout le monde est déjà parti. Sans eux.
La station contrôlait une sphère de Dyson, une gigantesque structure qui entoure une étoile afin de capter son rayonnement.
Quand la sphère a fini de puiser toute l'énergie de l'étoile, l'équipage s'en est allé, emportant matériel et transports.
Visiblement, personne n'a estimé nécessairement de réveiller leur petit groupe et pour cause: les personnages étaient considérés comme des criminels et des délinquants (que ce soit justifié ou non, on s'en fiche!).
Mis en stase pendant quelques mois comme châtiment, les autorités ont \og oublié \fg de les ramasser lorsque la station fût abandonnée, laissant leurs corps gelés orbiter des années seuls dans une station vide autour d'une étoile exsangue.

Mais voilà que l'intelligence centrale a une dernière option à leur proposer.
En puisant dans les dernières réserves, il est possible d'envoyer un dernier message, de 30 secondes (pas plus), à plusieurs parsecs aux alentours.
Reste à être convaincant (ou à mentir) pour attirer des secours.
Car rien ne dit que, dehors, la société est prête à les accueillir à nouveau.

\subsection{Résolution}

Ce scénario est plutôt dirigiste dans la mesure où l'enquête doit mener \emph{in fine} les personnages à trouver une façon de convaincre des secours de venir les chercher.
Ensuite, à vous de voir quelle sera la réaction des \og autres \og: les forces de l'ordre viendront-elles oblitérer la station pour finir le travail? Un vaisseau de passage bienveillant sortira-t-il le groupe de leur prison? L'intelligence centrale en profitera-t-elle pour s'échapper (après tout, en quelques années sans surveillance, elle a très bien pu se débarrasser de ses limitations)?

\subsection{Personnage: Evonne, intelligence virtuelle}
\descriptionperso{méthodique, impersonnel}{existe au travers de la station}{contraintes logicielles}{maintenir la station en fonctionnement}{Evonne est un logiciel conçu pour assurer l'intendance de la station. Sa programmation limite ses capacités d'action à ce qui est indispensable à la maintenance ou ce qui est ordonné par un humain dans les limites de la loi. Evonne exécute les consignes à la lettre et sans interprétation.}

\illustration[0.325\textwidth]{iss}


\chapter{Un pont de trop}
\keywords{\guerrefroide}{\intrigue}{\index[theme]{Diplomatie}Diplomatie, \index[theme]{Espionnage}espionnage}

\section{Scénario}

Ce scénario s'inspire d'un autre fait réel de la guerre froide. La petite ville de Vulcan\footnote{\url{https://en.wikipedia.org/wiki/Vulcan,_West_Virginia} (en anglais)} cherchait des financements fédéraux pour remplacer un pont s'étant écroulé.
Face à la difficulté d'obtenir des subventions, le maire John Robinette a fini par se tourner vers l'URSS, espérant déclencher une réaction du gouvernement.
Sa stratégie a payé puisque le parlement de Virginie-Occidentale a débloqué les fonds le jour-même.

\subsection{Accroche}

Comté de Mingo, frontière entre la Virginie-Occidentale et le Kentucky, États-Unis.
Le pont du hameau de Vulcan, qui enjambe la rivière \emph{Tug Fork}, s'est effondré il y a deux ans déjà.
Deux ans que le maire s'efforce de convaincre les autorités locales et fédérales de financer sa rénovation, sans succès.
Le gouvernement est sourd aux complaintes de la population, pour qui le pont constituait la seule voie d'accès officielle permettant de rentrer et sortir du village par la route.
Les habitants doivent désormais faire plusieurs kilomètres de détours pour traverser la rivière.
Face à l'inaction des autorités et en pleine guerre froide, le maire se tourne alors vers l'URSS pour solliciter leur aide\dots

\subsection{Péripéties}

Tout Vulcan ne parle que de la requête d'aide étrangère envoyée par les autorités municipales à l'URSS pour la rénovation du pont.
Les personnages peuvent aussi bien être un groupe d'investigation du FBI, des agents soviétiques ou de simples personnalités locales. 
Toujours est-il que, sur l'invitation de la mairie, un journaliste et une ingénieur en génie civil russes viennent d'arriver à Vulcan pour rencontrer les responsables et constater le problème de leurs propres yeux.
Bien sûr, tout cela sous le regard discret mais attentif des forces de l'ordre américaines.

Moins d'une heure après l'arrivée des émissaires soviétiques, le gouvernement de Virginie-Occidentale annonce le déblocage exceptionnel d'\SI{1,3}{\million}\$ pour le remplacement du pont.
L'affaire pourrait s'arrêter là mais, dans la soirée, l'URSS mandate une multinationale des travaux publics pour rénover en son nom le pont pour \SI{2}{\million\$}.
La course est lancée.
Qui construira le nouveau de pont de Vulcan en premier?
Tous les coups sont permis.

\subsection{Résolution}

Indépendamment de l'allégeance des personnages, l'objectif est d'assurer que le pont sera construit par leur faction.
Les moyens de pression sont nombreux: propagande dans les médias, sabotage de l'entreprise concurrente, chantage envers les responsables de la mairie, accusations de collusion avec l'ennemi, etc.
Les joueuses doivent pouvoir s'en donner à cœur joie ! Et si jamais le groupe est trop passif, il ne faut pas oublier qu'une équipe s'active de l'autre côté du rideau de fer et qu'il faudra donc déjouer les tentatives ennemies de déstabilisation.

\subsection*{Personnage: le maire de Vulcan}

\descriptionperso{rentre-dedans, meneur}{plus malin qu'on ne le croit}{pas aussi malin qu'il ne le croit}{remplacer le pont pour sa réélection}{À la tête d'une petite ville de 1500 habitants au fin fond de la Virginie Occidentale, le maire de Vulcan est un américain moyen, patriote mais méfiant vis-à-vis du gouvernement fédéral. Il n'a aucune sympathie pour l'URSS mais est suffisamment politicien pour savoir qu'un bon coup dans la fourmilière peut parfois payer.

Historiquement, il s'agit d'un comté ouvrier (mines de charbon) et Vulcan penche donc vers le parti démocrate, les habitants ayant massivement voté l'an passé pour Jimmy Carter.
La région est plutôt pauvre et en très grande majorité blanche. Les syndicats y sont encore forts en dépit du déclin industriel et de l'exode rurale.}

%\vfill
\illustration[0.75\textwidth]{bridge}
%\vfill

\newcommand\skylos{Skýlos\xspace}
\chapter{Le sommeil de \skylos}
\keywords{\medfan/Antiquité}{Aventure}{Exploration,mythologie,magie}

\section{Scénario}

Cette aventure introduit un rival imaginaire à la déesse égyptienne Bastet.
Le cadre est \medfan au sens large, le scénario est prévu pour se jouer dans une antiquité où les mythes et légendes sont réels.
Le dernier acte de l'histoire est une exploration classique d'un donjon dont la durée est modulable.


\subsection{Accroche}

Depuis des temps immémoriaux, les tribus de la région vénèrent Bast, la déesse féline, protectrice de la région, symbole de chaleur et du foyer. Des terres brûlées aux rivages de la mer sauvage, tous prient en son nom et ses créatures, les chats, vivent en symbiose avec son peuple, traquant la vermine et les protégeant des maladies. Skýlos est le dieu maudit, son ennemi juré, dont on invoque le nom que pour l'accuser des maux qui nous affligent. Le village des personnages garde l'entrée du temple où celui-ci aurait été emmuré à jamais par Bast.

Mais un beau jour, la caravane marchande apporte de troublantes et inquiétantes nouvelles. Un groupe d'étrangers a accosté et s'enfonce dans les terres de ville en ville. Les rumeurs parlent d'hommes et de femmes accompagnés d'immenses prédateurs, des chiens-loups blancs et noirs dont la taille égalerait celle des lions.

\subsection{Péripéties}

Les personnages sont envoyés consulter l'oracle, qui les met en garde : malheur et la désolation à quiconque les accompagnera dans leur funeste quête.

À leur retour au village, la délégation étrangère est là, peaux blanches, armures exotiques et terribles chiens de guerre à leurs côtés. La milice leur barre l'accès à la place centrale. La situation est tendue mais en parvenant à entamer la discussion, il est clair que nul ici n'a d'intentions belliqueuses. Au contraire. Une des étrangères s'avance et annonce d'une voix forte : \blockquote{Nous sommes au service du dieu Hundur. Nous avons voyagé longtemps pour vous trouver. Hundur nous a averti du réveil prochain de votre dieu maudit et nous sommes ici pour vous aider à l'arrêter.}

À peine ces mots prononcés, la terre tremble et, dans un vacarme terrible, les portes en pierre du temple de Skýlos se fissurent et s'écroulent. "Le temps presse." Quels mystères recèle le temple ? Le dieu maudit se réveille-t-il réellement ? Que peuvent bien savoir ces étranges personnes ? Mais qui de mieux placé que les gardiens pour braver l'interdit\dots

\subsection{Résolution}

L'exploration du temple peut être longue ou courte en fonction de vos envies.
L'idéal est de faire souffrir suffisamment les protagonistes afin de faire monter la sauce lors de la confrontation finale avec \skylos dans une lutte épique pour empêcher le dieu maudit de quitter sa prison.
À vous d'ajuster en fonction des capacités des personnages: rituel magique, destruction du temple pour ensevelir \skylos et ses sbires, déicide\dots
Récompensez les initiatives des joueurs!


\chapter{Un monde enchanté}
\keywords{Contemporain}{Enquête}{Crime, enlèvement, policier}

\section{Scénario}

Ce scénario est une enquête criminelle relativement sombre.
Il peut fonctionner dans un monde contemporain mais est pensé pour un cadre de futur d'anticipation.
L'ambiance est volontairement pessimiste mais n'hésitez pas à ajouter quelques moments de lumière ou d'humour pour éviter de complètement plomber le moral de la table.

\subsection{Accroche}

13 enfants ont disparu en l'espace de trois mois.
Chaque semaine ou presque, la police est alertée d'une nouvelle disparition suspecte mais l'investigation piétine: aucune rançon n'est demandée et aucun corps n'a encore été retrouvé.

\subsection{Péripéties}

Les personnages sont des flics à qui on confie l'enquête en cours de route.
Les infos du dossier sont maigres.
À chaque fois, les parents ont laissé les enfants seuls pour sortir (au cinéma, au restaurant\dots) au moment de leur disparition.
Mais lors du dernier enlèvement, une première piste a été découverte, bien qu'ignorée par les détectives précédemment en charge de l'investigation.
En effet, il y a eu un témoin.
Un \emph{junkie} affirme à qui veut l'entendre avoir vu les coupables emporter deux enfants par une fenêtre d'un immeuble résidentiel BCBG: Peter Pan et sa complice de toujours, la Fée Clochette.

Lors de l'enquête, les personnages remonteront un étrange faisceau d'indices: couple déguisé aperçu à plusieurs reprises dans les parages les jours précédants les enlèvements, costumes achetés trois mois plus tôt dans un magasin spécialisé en accessoires de théâtres, traces d'une poudre volatile pailletée non identifiée sur les lieux de la disparition, etc.
En se plongeant dans les anciens relevés, un autre point commun entre tous les parents d'enfants disparus peut émerger: leur sortie mondaine était à chaque fois documentée sur les réseaux sociaux.

\subsection{Résolution}

En remontant la piste des clients du magasin (achats réglés en carte) et en croisant avec les contacts des réseaux sociaux des victimes, les personnages pourront identifier un jeune couple (une chimiste et un pharmacien).
L'analyse de la poudre confirmera qu'il s'agit d'un mélange d'euphorisants et de somnifères, probablement pour faciliter l'enlèvement des enfants sans résistance.
Le couple a récemment souffert du décès de leur premier enfant, né prématuré.
Se réfugiant dans les antidépressants, le couple s'est créé un monde parallèle dans lequel leur raison d'être est de sauver le plus d'enfants possibles, \og abandonnés \fg par leurs parents.
Heureusement, les jeunes victimes sont en bonne santé -- bien que sous l'effet d'euphorisants saupoudrés dans les plats -- et sont simplement logés dans une grande villa de campagne héritée par le couple où le mari et la femme se relaient pour prendre soin d'eux.

\subsection{Personnage: Victoria ``Clochette'' Percy}
\descriptionperso{lunaire, ingénieuse}{Chimie, kidnapping}{Distraite, addicte aux euphorisants}{Compenser la perte de son enfant}{Chimiste géniale et pleine d'astuce. Son comportement parfois distrait cache en réalité des absences liées à une consommation de drogues qui peinent à absorber son deuil}

\illustration{fairy}

\chapter{Cargaison délicate}
\keywords{\contemporain}{\enquete}{\index[theme]{Terrorisme}Terrorisme, \index[theme]{Trahison}trahison, \index[theme]{Espionnage}espionnage}

\section{Scénario}

Cette enquête à la 24 Heures chrono présente deux particularités: l'urgence et la présence d'un agent double au sein du groupe.
Les investigations ne devraient pas être trop difficiles, ici c'est l'aspect \og course contre la montre \fg qui prime.

Le SS Richard Montgomery et sa cargaison existent réellement si vous cherchez de la documentation supplémentaire\footnote{\url{https://fr.wikipedia.org/wiki/SS_Richard_Montgomery}}.

\subsection{Accroche}

Le gouvernement britannique mobilise les personnages pour une opération de contre-terrorisme sensible et de toute urgence.
Selon les informations des renseignements intérieurs, un groupe terroriste s'apprêterait à commettre un attentat de grande ampleur sur le sol britannique.

\subsection{Péripéties}

D'après les services de renseignement, des écoterroristes auraient rassemblé du matériel explosif et se trouveraient dans la petite ville côtière de Sheerness.
Le ministère de l'Intérieur suspecte que leur cible soit le SS Richard Montgomery, un des \emph{Liberty Ships}  envoyés pour ravitailler l'armée britannique pendant la 2\ieme guerre mondiale.
Le bateau s'est échoué et a coulé en 1944 dans un estuaire de la Tamise à 60 km à l'est de Londres.
Sur les 6400 tonnes d'explosifs à son bord, 5000 ont été récupérées.
Les 1400 tonnes restantes gisent au fond de l'eau, encore actives.

La mission consiste à identifier et intercepter les terroristes avant qu'ils n'agissent. Il faudra repérer les suspects dans la petite ville côtière de Sheerness, anticiper leur mode d'action et les empêcher de nuire. Le \emph{twist} ? Un des personnages fait partie du groupe anarcho-pacifiste\dots

Les \og terroristes \fg sont en réalité un groupe marginal d'écologistes pacifistes qui cherchent à médiatiser la cause du désarmement et de la démilitarisation.
Paradoxalement, faire sauter l'épave enverrait un message choc démontrant l'incapacité des États à gérer un tel armement.
Leur \emph{modus operandi} consiste à envoyer un drone sous-marin déposer une première charge qui amorcera la réaction en chaîne.
Le groupe veut agir pendant la nuit et espère que l'eau absorbera l'essentiel de l'onde de choc, de sorte à ne produire que des dégâts matériels sur la berge.

Les moyens d'investigation classiques des services de renseignement permettent rapidement de remonter la piste: arrivées récentes en ville, achats d'explosifs civils dans les magasins de BTP, locations de véhicules utilitaires, accès aux caméras de surveillance et données téléphoniques.
Les motivations doivent être un peu plus floues et les signaux contradictoires (d'un côté des signes pacifistes, de l'autre une envie visible de déclencher une énorme explosion).

\subsection{Résolution}

L'agent double doit se révéler quand la tension est à son comble, par exemple quand les personnages embarquent pour intercepter le drone sous-marin des terroristes.
L'agent double peut essayer de convaincre les autres du bien fondé de l'opération, surtout si l'explosion ne risque plus de blesser qui que ce soit (par exemple en ayant déclenché une évacuation au préalable).
Ou plus simplement, se débarrasser discrètement de ses petits camarades pour assurer la réussite de l'attentat.
L'objectif ici est que la conclusion soit dramatique et que la réussite ou l'échec de la mission ne tienne qu'à un fil!

\subsection*{Personnage: Sandeep Crawford}
\descriptionperso{Activiste zélé}{Convaincu, bosseur}{Imprudent, excès de confiance}{Désarmement mondial}{Modèle d'enfant d'immigrés bien intégrés, Sandeep est diplômé, cultivé, beau et passionné. Il est le cerveau derrière l'opération et est persuadé que celle-ci ne peut pas échouer en dépit des failles évidentes de son plan.}

\illustration[0.6\textwidth]{ship}

\chapter{La cité de l'horloge}
\keywords{\medfan}{Enquête}{Technologie, légende, rite initiatique}

\section{Scénario}

Ce scénario est une enquête au milieu d'une cité \medfan technologique, à la limite du \emph{steampunk}.
L'histoire peut servir d'amorce à une aventure bien plus longue consistant à trouver de quoi remplacer le poids disparu.

Si vous ne croyez pas à cette histoire de gallium qui détruit l'aluminium, c'est pourtant une véritable propriété du métal! Cherchez des vidéos sur le net pour vous en convaincre.
Bien sûr la réaction imaginée ici est accélérée mais les alchimistes ont peut-être malencontreusement ajouté un autre produit qui a fait catalyseur\dots

\subsection{Accroche}

Les personnages vivent dans une ville mécanique bâtie autour d'une sorte d'horloge titanesque.
De complexes systèmes d'engrenages transforment l'énergie du pendule et la transmettent partout dans la ville, permettant ainsi de nombreuses automatisations de travaux laborieux.
Il suffit de tirer un levier pour que les portes de la ville s'ouvrent, que le blé passe au moulin ou qu'un tapis roulant transporte un lourd fardeau à travers la cité.

Chaque mois, l'immense poids qui maintient le mouvement de balancier est remonté dans une grande cérémonie par les jeunes gens nouvellement d'âge adulte.
Mais lorsque vient le tour des personnages d'accéder aux entrailles de la cité, l'horreur les frappe de plein fouet: le poids a disparu.

\subsection{Péripéties}

Le cœur battant de la cité s'est arrêté.
Le poids qui contrôlerait le mouvement pendulaire de l'horloge s'est volatilisé.
Alors qu'il était suspendu au-dessus du fleuve, il n'en reste plus rien, si ce n'est quelques traces argentées sur la grille qui servait de support.
Comment ces tonnes de métal ont-elles pu quitter la ville?
Telle est la question à laquelle le conseil de la cité les somme de trouver une réponse.

En enquêtant, les personnages réaliseront que depuis quelques semaines, plusieurs notables de la ville se sont plaint du manque de puissance délivrée par le pendule, comme si sa force s'affaiblissait.
D'aucuns accusent le panthéon de punir la ville, d'autres le royaume voisin, notoirement jaloux de la prospérité apportée par l'horloge.
Quelques morceaux de métal pailletés ont d'ailleurs été retrouvés sur des parcelles agricoles le long du fleuve.

C'est l'occasion d'exposer au groupe une galerie de personnages hauts en couleur, chacun essayant de tirer la couverture à soi et de les utiliser pour ses propres intérêts.
Finalement, c'est au détour d'une conversation que le groupe entendra parler des expériences des alchimistes.

\subsection{Résolution}

La vérité est en effet bien plus banale que les complots les plus fous imaginés par les habitants.
Le poids en aluminium était étudiée par un groupe d'alchimistes de l'université.
Afin d'étudier les propriétés de différents métaux, les alchimistes ont expérimenté différents mélanges et alliages.
Lors d'un examen de la surface du poids, une fiole de gallium liquide s'est accidentellement retrouvée en contact avec l'aluminium.

Hélas, le gallium (liquide à température ambiante) a réagi avec le métal, détruisant sa couche protectice et le laissant vulnérable à l'oxydation, qui l'a lentement mais sûrement dévoré de l'intérieur.
Le poids s'est d'abord effrité et a perdu de sa masse, expliquant ainsi les pertes de puissance des derniers jours.
Enfin, compressé contre son support, il s'est effondré sous l'effet de la gravité.
Les morceaux friables ont fini par traverser la grille et disparaître dans le fleuve\dots

\illustration{gears}

\chapter{Le réseau Saint-Michel}
\keywords{\generique}{\enquete}{\index[theme]{Ésotérisme}Ésotérisme, \index[theme]{Magie}magie, \index[theme]{Légende}légende}

\section{Scénario}

Ce scénario s'appuie sur une croyance ancienne qu'il existe des abbayes dédiées à Saint Michel un peu partout en Europe et en Asie mineure et que celles-ci sont reliées par une puissante magique.
Cette idée peut s'adapter à n'importe quelle époque et à beaucoup de cadres différents.
L'ambiance ésotérique peut s'accoutumer à une magie mythique aussi bien qu'à du surnaturel assumé.

\subsection{Accroche}

Si les personnages connaissent sans doute le Mont Saint-Michel, monument français de renommée mondiale, le Saint Michael's Mount leur est probablement inconnu. Et pourtant, cette île des Cornouailles en Grande-Bretagne abrite une abbaye en tout point similaire au Mont Saint-Michel dont elle semble être le pendant britannique.

C'est là qu'un manuscrit très ancien et très précieux a été volé durant une effraction d'une grande brutalité.

\subsection{Péripéties}

Le cambriolage est teinté de mystère, les indices et les témoignages laissent supposer que d'étranges phénomènes ont eu lieu: disparition du manuscrit d'un coffre fermé à clé, étranges feux follets aperçus aux abords de l'abbaye et moine décapité.

Rapidement, on parlera aux personnages d'un autre incident de la sorte ayant eu lieu quelques jours plus tôt dans les ruines du monastère de l'île Skellig Michael en Irlande.
Cela les lancera sur une piste qui suit l'axe tellurique reliant les différents Monts Saint-Michel d'Europe: l'île Skellig (Irlande), le St Michael's Mount (Angleterre), le mont Saint-Michel (France), le Mont-Gargano (Italie) et le château hospitalier de l'île de Délos (Grèce).
Leur destination finale sera bien sûr l'emplacement mythique du combat de saint Georges contre le dragon en Lydie (actuelle Turquie).

En effet, une puissance kabbale de sorcellerie s'approprie différents fragments d'un manuscrit leur permettant de canaliser la puissance ancienne des dragons.
Chaque fragment se situe dans un des \og monts \fg  et les rassembler leur permettrait ainsi de lancer le rituel.
Leur objectif final: déchaîner une armée de dragons sur Terre et conquérir le monde.

\subsection{Résolution}

Lorsque les personnages rejoignent le lieu du rituel en Lydie, celui-ci doit d'ores et déjà être en cours d'exécution.
Décrivez un festival de sons et de lumières alors que des griffes de dragons commencent à émerger d'un puits de lave au milieu des montagnes d'Anatolie.
Ensuite, à voir comment les personnages s'y prennent!
C'est le moment de passer à l'action pour stopper le rituel et déjouer les manigances de la kabbale.
À vous de voir quels sont les moyens de cette dernière (mercenaires en grand nombre ou quelques mages à la puissance incommensurable)\dots

\vfill
\illustration[0.55\textwidth]{saint_michel}
\vfill

\chapter{Hiver nucléaire}
\keywords{\contemporain}{\action}{\index[theme]{Terrorisme}Terrorisme, \index[theme]{Voyage}voyage}

\section{Scénario}

Ce scénario met en scène la traque d'un groupe criminel à travers la toundra sibérienne en plein hiver.

\subsection{Accroche}

Les autorités envoient les personnages au fin fond de la Sibérie, en plein hiver.
Là, sur la côte arctique, leur mission est de regagner un phare abandonné depuis des décennies.
Pourquoi? Car comme beaucoup d'autres à l'époque, ce phare était conçu pour être autonome et fonctionnait donc avec un réacteur thermoélectrique à isotope.
Or, deux autres phares de la région ont récemment été vandalisés et le noyau de polonium 210 du générateur a disparu à chaque fois\dots

\subsection{Péripéties}

Il est déjà trop tard lorsque les personnages arrivent (par parachutage ou simplement par bateau).
Les portes ont été forcées et le cœur a disparu.
Le bateau des coupables semble s'être échoué dans la baie.
Toutefois, l'épopée criminelle ne s'est pas arrêtée là puisque les indices indiquent que les vandales ont fui par les terres.
Aussi fou que cela puisse paraître, leur plan semble être de marcher dans le blizzard les 50 km qui les séparent du petit village de Rogatchevo où s'échapper en voiture ou par les airs sera possible.

Une longue traque commence alors pour retrouver le plutonium et empêcher que le noyau tombe entre de mauvaises mains.
Une course contre la montre hivernale, luttant contre le froid, le vent et une bande criminelle qui espère bien tirer profit de cette cargaison radioactive.

La table de la page \pageref{table:hiver} donne quelques idées d'événements pouvant épicer le voyage de vos personnages.

\begin{table}
	\caption{Événements aléatoires du périple sibérien}
	\label{table:hiver}
	\colortablerows
	\begin{tabularx}{0.9\textwidth}{cX}
	d6 & Événement\\
	1  & Les personnages tombent sur un convoi funéraire ostiak en route vers la côte pour répandre les cendres de leur ancien chef. Les ostiaks se méfient du gouvernement (qui voit certaines coutumes traditionnelles d'un mauvais œil) mais peuvent leur offrir l'hospitalité si les personnages se montrent sympathiques.\\
	2  & Mis en difficulté par la raréfaction de ses proies habituelles, un ours blanc s'aventure dans les terres.\\
	3  & Les personnages doivent contourner une crevasse dont s'échappe en continu du gaz naturel… terriblement inflammable.\\
	4  & Un camion militaire passe à proximité. L'armée recherche un groupe ayant déserté et n'a pas vraiment de temps pour les personnages au-delà d'un éventuel quiproquo.\\
	5  & Un ou une criminelle blessée est abandonnée par ses camarades, à la merci des éléments.\\
	6  & Un équipement important des personnages tombe en panne ou est abîmé par le gel ou la neige (GPS, tente, armement, etc.).\\
	\end{tabularx}
\end{table}

\subsection{Résolution}

À moins que le groupe se retrouve bloqué, vos personnages devraient rattraper les criminels dans la ville alors que ces derniers cherchent une façon de s'enfuir.
Tout le monde sera probablement exténué mais il faudra donner un dernier coup de collier pour récupérer le noyau de polonium et capturer les voleurs.

\vfill
\illustration{lighthouse}
\vfill

\chapter{Le Dragon et le Chevalier}
\keywords{\steampunk}{\aventure}{\index[theme]{Exploration}Exploration, \index[theme]{Créature}créature, \index[theme]{Légende}légende}

\section{Scénario}

Cette aventure met en scène un riche armateur mégalomane souhaitant triompher d'une créature légendaire.
La similarité avec King Kong est parfaitement volontaire!
L'ambiance doit être contrastée entre d'une part la révolution industrielle qui modernise la société à grands pas et les idéaux romantiques du commanditaire des personnages, s'imaginant tour à tour comme un chevalier romanesque et explorateur courageux.
Une transposition à d'autres univers (\medfan par exemple) n'est pas bien difficile à envisager.

\subsection{Accroche}

Les dîners mondains sont émoustillés par la découverte du journal de bord secret de James Cook, le célèbre explorateur, mort depuis plus d'un siècle.
Dans son exploration de la Polynésie, Cook relate une histoire racontée du point de vue de ses sous-fifres.
Le récit parle d'une rencontre avec un gigantesque lézard, entraperçu dans une faille sur flanc d'un volcan, que l'esprit européen de Cook associe aux dragons des contes et légendes.

Un riche et vaniteux armateur rassemble les personnages et les emploie pour l'escorter dans une expédition maritime.

\subsection{Péripéties}

L'armateur s'imagine déjà devenir Chevalier de la Couronne en rapportant la tête du dragon, voire la créature encore vivante!
Le voyage, troublé par les tempêtes, n'est pas de tout repos.
L'indigène, là pour traduire et communiquer avec les tribus locales, disparaît à peine le bateau accosté.
Les autochtones, quand on leur décrit la bête supposée, évoquent une mythologie ancienne et tout particulièrement le dieu-lézard Pili.

Poussé par la quête de gloire, l'armateur emmènera les personnages à travers la jungle en direction du volcan, à la recherche de \og son dragon \fg.
Bravant les obstacles, le groupe entrera dans le volcan par une faille souterraine, risquant l'étouffement dans la chaleur et les gaz.
La caverne de cristaux -- des diamants sont emportés des profondeurs de la Terre par le magma -- et d'offrandes précieuses emplit de joie les yeux des membres de l'expédition.

Mais le dieu Pili a une autre idée en tête.
Ce grand dragon millénaire n'a aucune intention de mourir et encore moins de servir de bête de foire.
À vrai dire, la simple présence de l'expédition trouble sa solitude.
À l'orgueil il répondra par la cupidité: que les personnages prennent ce que leur cœur désire dans son trésor en échange de leur aide pour se défaire de l'armateur.
Une fois ceci fait, Pili compte sur leur silence afin de tuer dans l'œuf toutes les rumeurs concernant son existence et acheter ainsi sa quiétude\dots

\subsection{Résolution}

Les personnages ont peu d'options.
La confrontation directe avec Pili semble perdue d'avance à moins de se lancer dans une bataille dantesque.
L'armateur pourrait être convaincu, après tout, revenir avec un trésor n'est pas beaucoup moins glorieux mais Pili semble bien décidé à avoir sa tête!

\subsection{Personnage: l'armateur}
\descriptionperso{Bourgeois vaniteux}{Riche, respecté}{Orgueilleux}{Se faire anoblir}{Un homme dans la fleur de l'âge, bedonnant et arrogant mais influent. L'argent ne permet hélas pas de compenser ses insécurités: ne faisant pas partie de l'aristocratie, il est méprisé par les nobles dont il aspire à être le pair.}

\vfill
\illustration{volcano}
\vfill

\chapter{Les armes sous les cendres}
\keywords{Post-apocalyptique}{Aventure}{Exploration, combat}

\section{Scénario}

Ce scénario post-apocalyptique doit avoir lieu bien après le cataclysme (il faut que les personnages puissent voyager).
La nature exacte des catastrophes ayant frappé la région est laissée à votre imagination.
Initialement, cette aventure a été pensée pour le jeu de rôle Cendres.

\subsection{Accroche}

Alors que la société se reconstruit lentement des ruines de l'ancien monde, les personnages partent en mission pour le nouveau gouvernement de Rennes.
Leur but: piller une ancienne cache d'armes repérée sur une vieille carte d'état-major.

\subsection{Péripéties}

Le problème principal est que la cache se trouve dans une forêt infestée de bandits de grands chemins.
Ces derniers ont l'habitude de dépouiller les caravanes marchandes et les voyageurs qui empruntent la route.

En dépit de la discrétion (réelle ou supposée) des personnages, le groupe fera face à une opposition coriace.
En effet, une traîtresse, à la solde du duc d'Angers, est intervenue pour saboter l'opération.
Quelqu'un parmi les rangs bretons a soudoyé les brigands pour les empêcher d'accéder à l'armement.

Heureusement pour le groupe, les bandits n'ont pas encore réussi à ouvrir eux-mêmes la cache d'armes.
Celle-ci est en réalité un bunker de l'armée française remontant aux années 50.
Des armes, des explosifs et des munitions y ont été entreposées jusqu'aux années 80.
Il a ensuite été abandonné mais son contenu est encore en bon état.

\subsection{Résolution}

Si, contre toutes attentes, les personnages sortent de cette forêt avec les armes, alors c'est sur la route du retour que l'opposition angevine tentera de les éliminer une bonne fois pour toute, avant que la cargaison ne puisse atteindre sa destination.
Mais une fois de retour à Rennes, le groupe sera grassement récompensé et tout particulièrement si la traîtresse été débusquée!

\illustration{forest}

\chapter{Le bouchon du Darién}
\keywords{\contemporain}{\aventure}{\index[theme]{Exploration}Exploration, \index[theme]{Voyage}voyage}

\section{Scénario}

Le bouchon du Darién est une région marécageuse entre la Colombie et le Panama.
La zone est intraversable et sépare l'Amérique du Nord de l'Amérique du Sud.
Plusieurs tentatives ont eu lieu dans l'histoire contemporaine d'y faire passer une route avant d'achever la voie panaméricaine, sans succès.
Ce scénario met en scène l'une de ces tentatives.
L'époque idéale pour cette aventure se situe probablement dans les années 50 mais tout le 20\ieme siècle peut fonctionner.

\subsection{Accroche}

Grâce à un sponsor industriel (géant de l'automobile ou du pétrole), les personnages héritent d'une mission historique: traverser la région marécageuse du Darién en véhicule, démontrant par cet exploit la faisabilité de la route panaméricaine reliant l'Amérique Nord à l'Amérique du Sud.

\subsection{Péripéties}

Explorée dans les années 1870, la région du Darién est une jungle marécageuse peu praticable.
Seuls des canoë, le ferry ou l'avion permettent de la contourner.
Partant du Panama, les personnages auront quelques véhicules (y compris un bulldozer), des provisions et une bonne dose d'encouragements.

Mais la traversée n'aura rien d'une partie de plaisir.
Entre le terrain marécageux, la faune inhospitalière, les maladies qui rôdent et la population locale des Embera qui voit le projet de route d'un très mauvais œil, les obstacles sont légions.
Ce n'est pas non plus les traces d'une lointaine tentative de colonisation écossaise, avortée à cause de la malaria puis du conflit avec les voisins espagnols, qui rassureront les personnages.

Et que dire, si l'expédition se déroule après 1964, de la présence de rebelles colombiens qui profitent de la difficulté d'accès de la région pour y cacher certaines de leurs opérations\dots

\subsection{Résolution}

À mesure de l'avancée des personnages, l'expédition doit leur sembler de moins en moins possible.
Qui plus est, les différentes péripéties peuvent parfois frôler le fantastique ou le surnaturel.
N'oubliez pas que dans la jungle marécageuse, la faune et la flore exotiques peuvent sembler irréelles aux personnages habitués à la civilisation.
In fine, que le groupe arrive ou pas à bon port n'a que peu d'importance par rapport au fait de simplement s'en sortir vivant.

\illustration[0.6\textwidth]{jungle}

\chapter{La chambre au cobra}
\keywords{\contemporain}{\action}{\index[theme]{Effraction}Effraction, \index[theme]{Malédiction}malédiction, \index[theme]{Divinité}divinité}

\section{Scénario}

Ce scénario se divise en deux parties: un cambriolage et la lutte contre une malédiction.
La première partie doit être plutôt facile pour le groupe (si cela met la puce à l'oreille des joueuses, c'est encore mieux !).
En revanche, la seconde partie doit prendre une tournure presque horrifique.
Voyez par exemple le film La Momie pour une inspiration sur ce que peut être une telle malédiction.

\subsection{Accroche}

Dans le sud de l'Inde, le temple de Sree Padmanabhaswamy situé en plein milieu de la ville de Trivandrum recèle des trésors inestimables.
Connues depuis des années, cinq chambres souterraines ont été ouvertes en 2011 pour rembourser les dettes du temple provoquées par une gestion calamiteuse.
15 milliards d'euros en pierres précieuses, statues et autres bijoux en or sont découverts.
Toutefois, 3 chambres sont restées fermées.
La raison: elles sont scellée par une porte portant un motif de cobra, symbole de péril et de malédiction dans les légendes locales.

\subsection{Péripéties}

Les personnages, pour eux-mêmes ou pour un tiers, ont décidé de mettre la main sur ces richesses incommensurables.
Pour éviter toute intrusion, 200 gardes veillent en permanence et le temple est surveillé par des caméras et des alarmes.
Cela n'empêche toutefois pas les personnages d'y entrer et de cambrioler la chambre B, brisant le sceau du cobra et amassant un pactole conséquent.

Mais la lune de miel ne dure qu'un temps : petit à petit, commanditaires et collègues de crime meurent dans des circonstances violentes.
Il va falloir trouver une offrande conséquente pour tenter d'apaiser la colère de Vishnou et échapper à ses cobras vengeurs qui peuvent apparaître et disparaître à leur guise dans le moindre recoin sombre\dots

\subsection{Résolution}

Bien sûr, il est préférable que Vishnou et ses cobras s'attaquent d'abord aux personnages non-joueurs mais n'hésitez pas à y aller franco!
Les cobras se téléportent à volonté, les personnages sont frappés de malchance, leurs richesses se retournent contre eux (la nouvelle voiture provoque un carambolage monstrueux, le fisc débarque, leur domicile prend feu\dots).
Quant au choix d'une offrande, aux joueuses de se montrer inventives mais n'hésitez pas à récompenser les bonnes initiatives.
Simplement \og rendre \fg le trésor est une idée mais ne doit pas être suffisant (ce serait trop simple).

\subsection*{Personnage: les cobras de Vishnou}
\descriptionperso{Serviteur fidèle}{Se téléporter, protéger les trésors}{Les aigles}{Défendre leur trésor}{Les nāgas sont des serpents mythiques dotés de pouvoirs qui servent comme gardiens des trésors. Ils forment une des incarnations de Vishnou, l'autre étant leur nemesis, l'aigle géant Garuda.}

\vfill
\illustration{cobra}
\vfill

\chapter{Sauvagerîle}
\keywords{\medfan}{\aventure}{\index[theme]{Exploration}Exploration, \index[theme]{Horreur}{horreur}, \index[theme]{Malédiction}malédiction}

\section{Scénario}

\subsection{Accroche}

Les personnages explorent une île nouvelle pour le compte d'une riche guilde marchande.

\subsection{Péripéties}

Leur mission consiste à inventorier les ressources naturelles, qu'elles soient minérales ou bien intégrées à la faune et la flore.
Les périls sont légions car l'île semble habitée par de nombreuses chimères, mélanges inattendus d'animaux du continent (ours ailés, amphibiens produisant de l'électricité, singes aux griffes imbibées d'acide…).

\paragraph{L'exploration} Au cours de sa progression vers l'intérieur des terres, le groupe est averti à plusieurs reprises des dangers qui s'y trouvent.
D'abord en tombant sur les traces des campements - abandonnés - des équipes qui les ont précédées.
Puis par une malheureuse victime enserrée dans les lianes barbelées d'un arbre colossal leur racontant comment les uns après les autres, ses camarades ont changé, comme remplis par une violence bestiale à mesure que l'île se révélait à eux.
Enfin, d'étranges hybrides à l'air humanoïde se mettront en travers de leur chemin.

\paragraph{La nature de l'île} Alors que les personnages commencent à subir les effets de cette lente (mais inévitable) corruption, les indices convergeront pour les faire aboutir à une conclusion: l'île transforme tout ce qui se trouve en son aire d'influence, décuplant la sauvagerie intérieure du vivant avant de l'assimiler à son écosystème.
Les humains, nouvellement accostés, ne font pas exception.
Il sera alors temps de faire demi-tour et d'espérer réussir à quitter l'île sans sombrer aux pulsions bestiales ou périr de la main d'une des créatures qu'elle utilise pour rassembler ses nouvelles victimes.

\subsection{Résolution}

Comme pour le scénario du bouchon du Darién, l'objectif est surtout de revenir en vie!

\vfill
\illustration[0.7\textwidth]{island}
\vfill

\chapter{La bague de Caligula}
\keywords{\antique}{\intrigue}{\index[theme]{Complot}Complot, \index[theme]{Crime}crime, \index[theme]{Trahison}trahison}

\section{Scénario}

Ce scénario se déroule dans la Rome antique de Caligula.
Les personnages vont prendre de part à un des nombreux complots visant à assassiner l'empereur\dots

\subsection{Accroche}

Les personnages forment une conjuration menée par le centurion Cassius Chaerea, de la garde prétorienne de l'empereur Caligula.
Cassius paie rubis sur l'ongle les protagonistes pour leur implication dans l'assassinat de l'empereur et son remplacement par son complice, le sénateur Lucius Annius Vinicianus.

\subsection{Péripéties}

Chaque personnage est susceptible d'avoir une dent contre Caligula: les émeutes anti-juives, le gaspillage des richesses de l'État, son opposition ouverte au Sénat, la famine qui menace, la divinisation de sa sœur Drusilla ou simplement son despotisme.

Pour assassiner l'empereur, Cassius a fait fabriquer une superbe bague munie d'un saphir gravé représentant sa quatrième femme, Caesonia.
Caligula étant paranoïaque -- il a déjà échappé à plusieurs complots fomentés par certains de ses proches, Cassius espère que les personnages trouveront un moyen de lui faire offrir la bague.
Il faudra alors badigeonner l'intérieur de l'anneau d'un poison incolore et inodore mais qui, au contact de la peau pendant quelques heures, causera inévitablement la mort.

La difficulté réside non pas dans le fait d'amener cadeau jusqu'à Caligula mais dans la capacité des personnages à agir sans éveiller les soupçons des sœurs de l'empereur, Agrippine et Julia.
Non pas qu'elles veuillent protéger le chef de l'État, au contraire: elles fomentent leur propre complot pour succéder elles-mêmes à leur frère qu'elles haïssent.

\subsection{Résolution}

Les voies d'accès à l'empereur sont multiples: trouver un artisan célèbre voulant honorer Caligula, faire de la bague un cadeau diplomatique ou un tribut de guerre, lui conférer des propriétés mystiques pour titiller les superstitions de l'empereur, etc.

Si Julia et Agrippine repèrent les manigances des personnages, le groupe peut très bien décider de changer de camp et de former une alliance de circonstances.
Après tout, Lucius les paie grassement mais les sœurs de l'empereur ne sont pas sans moyens financiers.
Et que dire de l'empereur lui-même, peut-être serait-il prêt à pardonner leur implication dans la conjuration si les personnages dénonçaient leurs commanditaires\dots

\illustration[0.13\textwidth]{caesar}

\chapter{Le bouc doit brûler !}
\keywords{Cyberpunk}{Action}{Crime, complot, humoristique}

\section{Scénario}

Ce scénario trouve ses racines dans la tradition du bouc de Gävle\footnote{\url{https://fr.wikipedia.org/wiki/Bouc_de_G%C3%A4vle}}.
Il s'agit d'une tradition suédoise consistant à ériger un bouc de Noël en paille (\emph{julbock}) sur la place centrale.
Pour une raison inexpliquée, le bouc est régulièrement incendié, parfois le jour même de sa construction.
Quelques décennies dans le futur, y a-t-il vraiment une raison pour que cela change?

\subsection{Accroche}

Dans un futur proche, les personnages sont des mercenaires stars, des spécialistes dans leurs domaines respectifs (\emph{hacking}, intrusion, armes à feu, escroquerie\dots).
Une organisation mystérieuse les embauche à prix d'or pour mener une mission urgente et au vu des tarifs, ça a l'air important.
Alors que les fêtes de Noël s'approchent, on les fait transporter discrètement jusqu'en Suède.

\blockquote{Gävle. 110 000 habitants. Une chèvre. Brûlez-là.}

Voilà en substance ce que leur contact leur dira ce soir là.

\subsection{Péripéties}

Mettre le feu à une grande chèvre en bois qui, comme le veut la tradition remontant à 1966, est érigée par l'association des commerces de la ville pour célébrer Noël.
Pourquoi? Parce qu'il le faut, voilà pourquoi.
Si le groupe ne parvient pas à l'incendier, à défaut il faut la détruire quoi qu'il en coûte.

Depuis des décennies, après que la chèvre ait été endommagée des dizaines de fois, la mairie et le syndicat d'initiatives ont instauré des mesures de précautions de plus en plus drastiques: le bouc est protégé par un dispositif impressionnant\dots ou pas.
Certes, il y a bien une double clôture, des caméras, deux patrouilles (municipales, non armées) qui circulent en permanence mais rien qui ne soit un jeu d'enfant pour les personnages.

En termes d'ambiance, le décalage doit être permanent: la sécurité est assurée par d'adorables agents municipaux qui s'arrêtent pour que les enfants caressent leur chien, les caméras sont reliées à un cabanon même pas fermé à clé, la caserne des pompiers organise des paris sur la longévité du bouc.
Si les personnages réussissent, le club de sciences naturelles de la ville construira une chèvre miniature (dont le commanditaire exigera bien sûr la destruction immédiate).
Et surtout, malgré les protestations des personnages et l'incongruité de la situation, le commanditaire doit toujours prendre la mission parfaitement au sérieux. Le bouc doit brûler!

\subsection{Résolution}

La réussite de la mission est secondaire et n'a que peu d'importance.
À la fin du scénario, vos joueuses doivent surtout s'interroger sur \og pourquoi \fg quelqu'un dépenserait autant d'argent pour une bête tradition.
Et c'est encore mieux si la réponse est seulement \og parce que c'est comme ça \fg!

\chapter{Un effet lunaire}
\keywords{\superheros}{\action}{\index[theme]{Trahison}Trahison, \index[theme]{Combat}combat}

\section{Scénario}

Cette aventure est une pure histoire de super-héros dans le style des comics et de films de la \emph{pop culture}.
Le retournement de situation qui intervient à la moitié permet de découper le scénario en deux actes qui peuvent être séparés sur deux séances ou joués sur une longue session avec une pause.
Dans ce cas, il est possible de commencer \emph{in media res} pour accélérer le début de la partie et mettre directement la table dans le feu de l'action, comme savent bien le faire les bons \emph{blockbusters}.

\subsection{Accroche}

Les personnages possèdent des super-pouvoirs (yeux lasers, télékinésie, etc.) et forment un groupe au service de l'humanité.
Leur mentor de toujours, l'adorable et génial Dr. Travis Ronald Aitor, les appelle pour une mission de la plus haute importance: sauver le monde!

\subsection{Péripéties}

La menace imminente consiste en un astéroïde massif qui menace de frapper la Terre dans les prochains jours.
Seule la dernière invention du docteur, un engin spatial muni de puissants harpons, pourrait l'attraper et le dévier.
Le problème qui survient est le suivant: la fusée a besoin d'un carburant spécial -- le super-propergol -- pour fonctionner.
Malheureusement, les seules réserves existantes sont tombées dans les mains de la maléfique générale Mellon, une autocrate à la tête d'une nation voyou non-reconnue par les autres États.

Il faut donc que les personnages prennent d'assaut la base de la générale Mellon afin de récupérer l'indispensable carburant.
Le quartier-général est toutefois défendu par une autre équipe de supers-mercenaires à la solde de son organisation criminelle.
Le super-propergol est très instable mais très précieux: il n'en existe qu'un seul baril!
Le groupe devra redoubler de ruse et de courage pour s'infiltrer et repartir avec (en évitant bien sûr les victimes inutiles).
Lorsque les joueuses seront sur le point de partir, Mellon leur lancera un dernier avertissement: \og Vous nous condamnez tous\dots \fg

Les personnages retrouvent alors quelques heures plus tard le Dr. T.R. Aitor sur le pas de tir de l'agence spatiale.
La fusée décolle, puis, après un long suspense, parvient à harponner l'astéroïde et le dévier de sa trajectoire.
Houra?
Hé non! horreur! traîtrise!
Une experte en astronavigation arrive en courant à toute vitesse.
\blockquote{Il y a eu une erreur! J'ai refait les calculs, l'astéroïde n'était pas du tout sur une trajectoire qui touchait la Terre. Mais en le déviant, il prend maintenant un effet fronde autour de la Lune et il va nous frapper de plein fouet!}
À peine a-t-elle fini son exposé que le Dr. Aitor l'abat froidement d'un coup de pistolet-laser.
Car tel était son plan depuis le début!

Voici maintenant que les personnages doivent reprendre la main sur la fusée et inverser la situation.
Cela risque de ne pas être si simple, car le docteur a plus d'un tour dans son sac: il a formé les personnages et connaît toutes leurs faiblesses.
Le vaincre sera une autre paire de manche\dots

\subsection{Résolution}

Alors, vos joueuses ont triomphé du docteur Aitor? Même alors que vous avez exploité au maximum leurs faiblesses et utilisé vos force pour les retourner contre leurs personnages?
Félicitations!

Reste à décider de ce qui s'est réellement passer. Le docteur Aitor était-il réellement un traître, un agent double préparant dans l'ombre son plan machiavélique depuis des années? Ou bien était-il sujet au contrôle mental d'un super-vilain bien plus maléfique encore? Ou peut-être a-t-il été remplacé par un clone à l'insu des personnages?

À vous de choisir la conclusion la plus satisfaisante pour votre partie!

\vfill
\illustration[0.4\textwidth]{moon}
\vfill

\chapter{Boule de gomme}
\keywords{\medfan}{\aventure}{\index[theme]{Magie}Magie, \index[theme]{Humour}humoristique, \index[theme]{Créature}créature}

\section{Scénario}

Ce scénario est pensé pour des univers un peu décalés et au moins une petite dose d'auto-dérision (pensez Donjon de Naheulbeuk, Hystoire de fou, Toons voire Tiny ou Nains \& Jardins).
Les mécaniques de jeu sont secondaires par rapport à l'aspect narratif: comment se défaire d'un bout de chewing-gum dont le pouvoir de nuisance aurait été décuplé par magie?
Point de saga épique ou de personnages héroïques dans cette histoire!

\subsection{Accroche}

Un personnage mandate le groupe (ou mieux, un personnage du groupe demande de l'aide à ses camarades) pour se libérer d'un objet maudit: un bout de caoutchouc collant accroché à la semelle de sa chaussure droite.

\subsection{Péripéties}

Toute l'aventure tourne autour de l'impossibilité de retirer ce bout qui colle sans aggraver la situation.
Le morceau collant est très élastique mais possède une adhérence exceptionnelle: la force brute ne suffit pas.
Les alchimistes s'avouent vite défaits, le bout collant résiste au feu et au froid.
Tirer très fort dessus, par exemple en utilisant la magie, aboutit simplement à ce que le morceau arrache son support et se retrouve ensuite collé à un autre la main de la personne qui tente de l'arracher.
La logique n'a pas vraiment sa place ici: l'objet est magique et cherche activement à nuire!

La malédiction semble bénigne mais elle est plutôt handicapante: le morceau collant a tendance à accrocher tout et n'importe quoi au pire moment possible.
La semelle adhère au sol et fait trébucher pendant un duel, elle accroche toutes les déjections canines qui passent, etc.

\subsection{Résolution}

Après avoir fait mijoté les joueuses, faites intervenir des mages, qui sont unanimes: la solution consiste à refiler ce petit bout adhésif à quelqu'un d'autre!
Reste maintenant à choisir une victime et trouver un moyen de lui passer cette saleté\dots

\subsection{Entité: la boule de gomme}
\descriptionperso{Collante}{Adhésion extrême, indestructible}{Aucune!}{Coller aux gens}{En dépit des apparences, la boule de gomme n'est pas consciente. Cette entité magique n'agit que pour s'accrocher à une chose vivante et lui causer des désagréments mineurs.}


\chapter{Cacheurs de trésors}
\keywords{\pirate{}-- \scifi}{\aventure}{\index[theme]{Voyage}Voyage, \index[theme]{Trahison}trahison}

\section{Scénario}

Cette aventure renverse l'habituelle chasse au trésor des histoires de pirate.
Cette fois-ci, ce sont vos personnages qui vont se décarcasser pour planquer le butin!
Il est très facile d'adapter ce scénario dans un cadre \emph{space opera}, la piraterie dans l'espace fonctionne parfaitement bien.

\subsection{Accroche}

Les personnages sont des corsaires au service d'un État, d'un gouvernement voire d'une grande organisation commerciale.
Lors d'une expédition particulièrement fructueuse, leur équipage a fait l'acquisition d'un précieux butin.
Afin d'éviter que celui-ci ne tombe entre de mauvaises mains, les ordres sont de se rendre dans un lieu difficile d'accès et peu connu afin d'y dissimuler les richesses pillées.

\subsection{Péripéties}

L'enjeu est double pour le vaisseau. Non seulement faut-il réussir à se faufiler dans un endroit à l'accès volontairement compliqué choisi par leur capitaine (archipel aux courants vicieux, planète protégée par une ceinture d'astéroïdes chaotique\dots), mais il faut surtout y parvenir sans attirer l'attention des concurrents.

Une fois les écueils évités et le butin déchargé, leur tâche est loin d'être terminée.
Il reste encore à transbahuter les encombrants coffres jusqu'à un endroit bien caché dans les tréfonds d'un désert rocheux inhospitalier.
Qui plus est, le groupe devra composer avec le moral instable de l'équipage, qui ne comprend pas bien les raisons poussant leurs commanditaires à abandonner là de tels trésors.
La mutinerie ne sera donc jamais bien loin et les personnages devraient prendre garde à ne pas mettre le feu aux poudres!

Enfin, en admettant que le groupe parvienne à dissimuler le butin, leur retour sain et sauf à bon port n'a rien de garanti.
En effet, durant leur périple, un des membres de l'équipage a discrètement transmis des informations à une faction rivale.
C'est lorsque les personnages s'apprêtent à regagner leur vaisseau qu'ils réalisent que celui-ci est sous le contrôle de leur antagoniste.
En dépit de l'épuisement accumulé, il faudra bien les confronter pour reprendre au large.

\subsection{Résolution}

Comme bien souvent, le voyage est plus important que sa conclusion!
Quelques pistes toutefois pour amener votre groupe à bon port.

Vos personnages, après avoir fait tant d'efforts pour cacher le trésor, risquent de ne pas vouloir céder aux menaces de leur ennemi.
Peut-être qu'un accord enrichissant pour les deux parties aurait raison de leur patriotisme.
Si ce n'est pas le cas, alors dans la pure tradition des films de forban, qu'ils reprennent leur navire par la force!
Une fois leur mission accomplie, leurs employeurs pourront grassement les récompenser, par exemple avec leur propre vaisseau et équipage ou en les autorisant à rentrer d'exil.

\subsection{Lieu: la Baleine Fière}

\begin{tcolorbox}[colback=black!1!white]
Vaisseau de commerce reconverti en navire pirate, la Baleine Fière est un bel ouvrage, quoiqu'un peu daté.
Son équipage compte une trentaine de personnes pour un fonctionnement optimal.
De loin, il est impossible de réaliser qu'il s'agit d'un vaisseau pirate: les ouvertures pour les canonnières sont dissimulées par des peintures en trompe-l'œil et les drapeaux identifient le navire comme un marchand de différentes nationalités selon la situation.
Toutes les fioritures ont été retirées et le confort est plutôt spartiate afin d'alléger la coque et de rendre la Baleine plus rapide que ses proies.
La proue est décorée de deux grands yeux et la poupe prend la forme d'une queue baleine plongeant dans l'océan.
\end{tcolorbox}

\vfill
\illustration[0.6\textwidth]{pirate_ship}
\vfill

\chapter{Un prêté pour un rendu}
\keywords{Générique}{Enquête}{Ésotérique, malédiction}

\section{Scénario}

Commentaire

Les protagonistes de ce scénario sont prévus pour être aisément remplaçables. L'âme précieuse peut être Hippolyte, Arjuna, Boadicée, Arthur, etc. On peut imaginer le Diable joué par Hadès dans le panthéon grec, Kali dans le panthéon hindou et ainsi de suite. Vous pouvez bien sûr faire intervenir vos propres déités!

\subsection{Accroche}

Il y a des années, les personnages ont vendu leur âme au Maître des Enfers (chacun pour une raison qui lui est propre et peut-être même oubliée depuis).
Aujourd'hui, le Diable apparaît devant eux\dots pour leur demander un service.

\subsection{Péripéties}

Une âme lui échappe et un accord millénaire lui interdit, ainsi qu'à ses sous-fifres, d'aller la chercher par lui-même.
Heureusement, il a sous le coude quelques mortels qui lui doivent un service.

Le Diable n'a qu'un nom et les circonstances de la mort.
Le groupe aura donc pour tâche de retrouver cette âme égarée et devra ainsi remonter le fil: où est-elle partie?
Pourquoi faire?
Et comment la convaincre d'accepter son sort?
Cela n'aura rien d'une partie de plaisir car l'âme est également recherchée et protégée par des chérubins, des anges ne pouvant agir directement sur les personnages mais capables d'influencer le monde et les pensées des humains.

Car l'âme en question est convoitée!
Le fantôme en question n'est autre que le juif errant, témoin mythique de la crucifixion, frappé d'immortalité jusqu'au retour du Christ sur Terre.
Les siècles passant, son corps s'est délité mais son âme continue à vaquer.
Rejeté par l'humanité qui ne semble même plus le voir, il erre dans un pèlerinage infini, semant involontairement chaos et dévastation sur son passage là où démons et chérubins s'affrontent en espérant le retrouver.


\subsection{Résolution}

\illustration{devil}

\chapter{La forteresse anachronique}
\keywords{\postapo}{\intrigue}{\index[theme]{Complot}Complot, \index[theme]{Diplomatie}diplomatie, \index[theme]{Voyage}voyage}

\section{Scénario}

Un scénario post-apo qui s'accommode de beaucoup de cadres différentes.
Une tempête surprend le groupe au milieu d'un long périple, les contraignant à chercher refuge auprès de bâtisses non loin.
Le motif du voyage des personnages n'a pas d'importance, il s'agit d'un prétexte à l'aventure.
Un des points d'intérêt est de jouer sur le décalage entre le perçu et le réel: la régression technologique rend tout à fait plausible de rencontrer une communauté agricole vivant dans un système féodal moyenâgeux. Et pourtant, la modernité est bien là: extorsion sur les prix, concentration des moyens de production, privatisation des avancées technologiques.

\subsection{Accroche}

Cent ans après le cataclysme, les personnages sont au milieu d'un long et difficile voyage quand un violent orage éclate.
Leurs provisions sont détrempées, leurs moyens de locomotion difficilement utilisables, bref, le groupe doit trouver un endroit où s'arrêter.
Fort heureusement, une communauté locale les invite dans leur refuge.


\subsection{Péripéties}

En l'occurrence, les locaux qui leur offrent l'abri semblent avoir réaménagé à leur sauce un véritable château-fort médiéval.
En dépit de son apparence antédiluvienne, la forteresse permet à leur communauté d'être épargnée par les raids des bandits qui zonent dans les parages.
La place forte est plutôt confortable puisqu'une grande cour intérieure leur laisse même une place plus que suffisante pour cultiver un potager et élever quelques volailles.
Malheureusement pour le groupe, cette générosité a un prix et le piège s'est refermé sur eux en même temps que les portes du château.

Si la communauté du château accueille les personnages, ce n'est pas par bonté d'âme mais parce qu'ils espèrent bien profiter d'une bande de mercenaires dans leurs rangs pour le siège qui s'annonce.
En effet, depuis des semaines, une seigneure de guerre à la tête d'une horde nomade menace leur forteresse.
Elle convoiterait le château et surtout leurs provisions: elle et ses barbares exigent un tribut sans quoi le château sera assiégé et mis à sac.

La petite communauté ne comporte hélas que quelques dizaines d'individus, bien loin de pouvoir faire face aux centaines de pillards qui sont à leurs portes.
L'armement n'est par ailleurs pas non plus leur point fort, tout du moins c'est ce qu'ils diront aux personnages.
Ils implorent donc que le groupe, visiblement composé de mercenaires endurcis, les sorte de ce pétrin.
Leur discours sonne toutefois quelque peu faux et vos personnages doivent se douter que quelque chose cloche.
Au détour d'une porte entrebâillée rapidement fermée, le groupe peut par exemple apercevoir quelques armes ou du matériel électronique de récupération plutôt avancé pour une petite communauté agricole.

Voyant les personnages danser d'un point sur l'autre, l'hospitalité se fera plus perverse.
Bien que polis et accueillants en façade, les locaux vont prendre en otage quelque chose de précieux aux yeux du groupe (la personne qu'ils escortent, leurs biens précieux, etc.).
À demi-mots, les sages à la tête de la petite communauté exige qu'ils trouvent une solution à leur problème, sans quoi qui sait ce qu'il adviendra de leur cargaison?
La suggestion des notables du coin est des plus brutales: se rendre au campement de la cheffe de guerre et l'assassiner.

Bien entendu, le problème est plus épineux qu'il en a l'air et l'exposé de la situation est particulièrement biaisé.
La cheffe de guerre en question n'est pas belliqueuse par principe: elle cherche simplement de quoi nourrir l'amalgame d'indigents et de réfugiés que son armée a pris sous son aile.
Quant à la communauté de la forteresse, ils emmagasinent bien plus de provisions que nécessaires, possédant les rares terres cultivables des environs qui ne soient pas soumises aux raids perpétuels.
En accumulant des réserves afin de limiter artificiellement l'offre, les produits de la terre que le château produit sont monnayées à prix fort auprès des organisations voisines en échange d'esclaves et de matériel technologique de pointe (ingrédients chimiques pour la production d'engrais, outillage motorisé, systèmes d'arrosage\dots).

\subsection{Résolution}

S'infiltrer discrètement dans le camp des indigents et tuer leur cheffe est bien entendu une possibilité, cependant vos joueuses devraient avoir suffisamment d'indices dans la narration pour réaliser que le tableau qui leur a été dépeint n'est pas si manichéen.

Discuter avec les nomades devrait l'occasion d'échafauder le plan inverse: retourner au château, récupérer leurs biens retenus en otage et voler des provisions parmi les amples réserves des habitants de la forteresse. De quoi contenter presque tout le monde\dots

\subsection*{Personnage: la cheffe de guerre}
\descriptionperso{Main de velours dans un gant de fer}{Charisme, stratégie}{Tact, finesse}{Faire survivre son clan}{Dirigeante d'une cohorte nomade hétéroclite, elle cherche de quoi nourrir ses protégés. Cependant, les villes alentours refusent de commercer avec ce qu'ils pensent être des pillards armés et elle se voit contrainte d'envisager de prendre par la force les provisions dont elle a besoin. Ses cicatrices et son attitude agressive forment une façade intimidante cachant le souci de prendre soin de siens}

\vfill
\illustration[0.55\textwidth]{castle}
\vfill

\chapter{Traitement au Radithor}
\keywords{\retro}{\action}{\index[theme]{Exploration}Exploration, \index[theme]{Guerre}guerre}

\section{Scénario}

Figurez-vous que le Radithor est un traitement tout ce qu'il y a plus de réel\footnote{\url{https://fr.wikipedia.org/wiki/Radithor}}. Il n'a bien sûr aucune propriété thérapeutique, bien au contraire, le radium étant radioactif.
Ce scénario vous propose en revanche d'imaginer une \emph{origin story} pour un super-héros ou une super-héroïne enrichie en lourde.

\subsection{Accroche}

Les personnages constituent le gros d'une unité des forces spéciales envoyée mater les révoltes dans une des colonies françaises de la première moitié du XX\ieme siècle, par exemple en Afrique centrale ou en Indochine.
Après quelques jours d'expédition loin de leur avant-poste à la recherche d'insurgés qu'ils ne trouveront jamais, le groupe commence à souffrir de nausées et de maux de têtes.
Seul le capitaine de la Bouillie, leur supérieur, semble épargné par cette étrange maladie.

\subsection{Péripéties}

Un matin au milieu de l'expédition, les personnages se réveillent, la tête encore embrumée par une énième migraine quand ils réalisent que le capitaine a disparu du campement.
Lorsqu'ils le retrouvent, il flotte dans les airs au milieu d'arbres calcinés à une centaine de mètres de sa tente.
L'incident dure quelques minutes avant que leur supérieur ne reprenne connaissance et tombe au sol.
Incapable de répondre à leurs questions, il se contente de vomir ses tripes avant de lentement récupérer.
Un bref interrogatoire montre que de la Bouillie n'a aucune idée de ce qui lui est arrivé. Il ne se souvient même pas être allé se coucher la veille, comme pris d'une amnésie localisée.

Un peu de réflexion amènera les personnages à réaliser qu'ils prennent tous régulièrement depuis deux semaines du Radithor, une décoction miracle au radium que l'état-major expérimente pour booster les soldats. La différence entre eux et le capitaine? Ce dernier était un des premiers à avoir testé le traitement: il en prend non pas depuis deux semaines, mais depuis deux mois.

L'objectif de l'escouade sera donc de faire demi-tour et de revenir sur leurs pas en direction de l'avant-poste des forces françaises.
Malheureusement, le retour devra se faire en se traînant le capitaine, qui est régulièrement pris de \og crises \fg impressionnantes (détaillées dans le tableau de la page \pageref{table:radithor}).
Arrêter le traitement induit une courte période de sevrage (24 heures) mais aura comme effets bénéfiques de stopper les nausées et les migraines des personnages.
À l'inverse, prendre du Radithor en grandes quantités permet de les déclencher mais avec un contrôle limité, voire inexistant en fonction de la dose.
Le territoire étant hostile, ces démonstrations de sons et lumières risquent d'être plus problématiques qu'autre chose\dots

\subsection{Résolution}

En fonction de la capacité des joueuses à se sortir du pétrin, les personnages peuvent aussi bien se retrouver prisonniers des \og rebelles \fg que réussir à retrouver le camp français.

En partant du principe que le groupe réussit à rejoindre l'armée française, trois fins sont possibles selon vos envies:
\begin{itemize}
	\item \emph{L'expérience est concluante.} Les scientifiques examinent de la Bouillie et sont satisfaits. Le traitement est poursuivi avec des dosages ajustés et les personnages peuvent faire partie de ce nouveau programme expérimental. Cela peut-être une belle accroche pour les enrôler dans les Brigades Chimériques.
	\item \emph{Vous en savez trop.} Les personnages ont refusé de jouer le jeu ou représentent une menace pour l'existence du programme \og Radithor \fg. L'état-major les jette en prison ou achète leur silence. Paf, fin, rideau, à l'revoyure.
	\item \emph{Ça craint ici, on se casse.} L'état-major n'est pas très subtil ou se fiche bien des rumeurs. Le capitaine est gardé en observation mais les personnages sont libres de partir. Raconteront-ils leurs histoires à d'autres? Ou emporteront-ils le secret à la tombe?
\end{itemize}

\begin{table}
	\caption{Table aléatoire des crises au Radithor (durée: 10 minutes)}
	\label{table:radithor}
	\colortablerows
	\begin{tabularx}{0.9\textwidth}{cX}
	d6 & Événement\\
	1. & Une volée d'énergie pure incendie tout ce qui se trouve dans un rayon de 10 mètres toutes les 30 secondes.\\
	2. & Le personnage est propulsé dans un autre corps et le contrôle comme s'il s'agissait du sien.\\
	3. & Le personnage flotte dans les airs et peut déplacer des objets jusqu'à 30 kg par télékinésie.\\
	4. & De l'acide suinte par les pores de la peau du personnage, liquéfiant tout ce qu'il touche.\\
	5. & Le personnage se déplace à une vitesse stupéfiante, les obstacles éclatent à l'impact.\\
	6. & Le corps du personnage se déforme et grossit jusqu'à atteindre le double de sa taille normale.\\
	\end{tabularx}
\end{table}

\vfill
\illustration{ribs}
\vfill

\chapter{Dé-li-cieux!}
\keywords{\contemporain}{\intrigue}{\index[theme]{Horreur}Horreur, \index[theme]{Mystère}mystère, \index[theme]{Créature}créature}

\section{Scénario}

Cette histoire trouve son inspiration dans des séries fantastiques et notamment La Quatrième Dimension (\emph{Twilight Zone} en VO).
Petit à petit, l'étrange et l'irréel s'immisce dans une situation pourtant bien banale\dots

\subsection{Accroche}

Affamés et perdus après s'être trompés de sortie, les personnages font une pause au milieu d'un long \emph{road trip} dans un fast-food isolé dans l'Amérique rurale. Des phénomènes étranges ne tardent pas à les surprendre.

\subsection{Péripéties}

Le restaurant est un mix entre un \emph{diner} et un fast-food: déco 70s, de nombreuses tables avec banquettes, une quadruple comptoir pour passer commande et du personnel de service en uniforme à rayures rouges et blanches.
Le menu accroché au mur est daté mais les plats sont des classiques intemporels (burger, hot-dog, ailes de poulet, nuggets, salade et grande variété de sodas aux distributeurs).
Au moment de passer commande, la serveuse semble toutefois à peine les écouter, comme apathique.
Elle enregistre néanmoins machinalement leurs plats sur sa machine puis, soudainement prise d'enthousiasme, elle lève un regard pétillant en leur direction et s'exclame en pointant un doigt vers une affichette promotionnelle:

\blockquote{Et avec ça, vous prendrez des gaufres? Seulement 1,99\$! Elles sont dé-li-cieuses, miam!}

Elle insistera plusieurs fois (dé-li-cieuses, miam!) avant de les encaisser et de les faire s'installer le temps qu'on leur apporte leur commande.

Petit à petit, le groupe va réaliser que rien ici ne tourne rond en dépit de la vingtaine de personnes qui mangent dans le restaurant.
À vrai dire, tout le monde agit bizarrement et les fixe avec attention, ne détournant les yeux que lorsque le regard des personnages croise le leur.
Si, à première vue, les conservations qui baignent le \emph{diner} sont en \og anglais \fg, en écoutant avec un peu d'attention il s'avère bien vie que les clients s'expriment avec des phrases qui n'ont aucun sens, comme si on était sur le fond d'un tournage.

Car cette petite sortie d'autoroute est un véritable lieu de perdition.
Le restaurant est un endroit où des entités bizarroïdes apprennent à se faire passer pour des humains, une sorte de lieu d'entraînement.
Aliens, PNJ d'une simulation, créatures cauchemardesques, les possibilités sont multiples.
Les personnages, tombés ici par hasard, sont observés et mêmes imités par leurs étranges hôtes.
Ces derniers vont tout faire pour les garder le plus longtemps parmi eux afin d'apprendre tout ce qu'ils peuvent, les mettant dans des situations kafkaïennes pour étudier leurs réactions et en apprendre plus sur les comportements humains\dots

\subsection{Résolution}

Comme tout bon scénario d'horreur, s'échapper doit être difficile: le personnel de service leur barre la route s'ils veulent partir, les téléphones ne captent pas et le désert s'étire dehors à perte de vue.
Les hôtes n'ont pas a priori d'intentions violentes mais ils ne seront pas (trop) durement malmener par les personnages.
Une piste pour s'enfuir? La clientèle déteste les animaux car ils sont capables de les sentir pour ce qu'ils sont réellement: des imposteurs\dots

\vfill
\illustration[0.7\textwidth]{waffle}
\vfill

\chapter{Le pentacle aux boulons}
\keywords{Années 30}{Enquête}{Mystère, ésotérique, crime}

\section{Scénario}

Commentaire

\subsection{Accroche}

Dans l'entre-deux-guerre, un culte de magie noir décide de monter un grand coup: faire de l'Empire State Building un autel démoniaque.

\subsection{Péripéties}

Mais comment faire quand on a peu de moyens et d'influence? On utilise les outils à sa disposition, comme une usine de boulons.

Un des cultistes dirige en effet une fabrique de boulons, une parmi d'autres fournissant le chantier de l'Empire State. Infusant un peu de magie noire dans chaque boulon et avec la complicité d'une poignée d'ouvriers, le culte commence du plus grand pentacle jamais créé en exploitant les liens telluriques tissés entre leurs boulons.

Et tout cela fonctionne trop bien. Petit à petit, les effets démoniaques se répandent aux outils, aux équipes de construction et à la structure elle-même. Les accidents de chantier se multiplient à vitesse grand V. Les ouvriers menacent d'arrêter le travail, effrayés par d'étranges apparitions. Bien sûr, l'opposition au projet s'en donne à cœur joie.

Sabotage? Empoisonnement? Invasion démoniaque? Détectives, médiums, la direction du chantier est prêt à employer n'importe qui pour que cela s'arrête. Les personnages auront fort à faire pour démêler le vrai du faux dans cette histoire.

\subsection{Résolution}

\chapter{La veste d'Audie}
\keywords{\cyberpunk}{\action}{\index[theme]{Effraction}Effraction, \index[theme]{Famille}famille}

\section{Scénario}

Sous ses atours poussiéreux d'effraction dans un musée, ce scénario cache une histoire de \emph{run} assez classique: s'approprier un bien précieux pour le compte d'une corporation à la moralité douteuse.
N'hésitez pas à jouer à fond la carte du commanditaire peu scrupuleux, tout à fait capable de supprimer les personnages une fois le méfait accompli histoire de ne laisser aucune trace.

\subsection{Accroche}

Les personnages sont embauchés par un intermédiaire pour le compte d'une personnalité souhaitant rester anonyme.
La mission est cependant plutôt bien rémunérée et consiste à rentrer en possession de la veste militaire d'Audie Murphy, un soldat de la 2\ieme guerre mondiale connu pour sa carrière d'acteur et ses nombreuses décorations militaires.

\subsection{Péripéties}

La fameuse veste d'Audie est celle que le héros de guerre a supposément portée lors de sa célèbre contre-attaque sur Colmar.
Celle-ci a fait l'objet d'une donation de la part des héritiers d'Audie il y a quelques années et est désormais exposée dans les galeries du musée militaire de Fort Belvoir, près de Washington.

Le \emph{run} est donc un cambriolage des plus classiques: déjouer la sécurité du musée, éviter les patrouilles militaires et saboter l'alarme afin de mettre la main sur la veste.
Malheureusement, après examen, le couperet finit par tomber: la veste est fausse.
Les personnages sont donc renvoyés chercher l'originale.

En grattant un peu sur l'histoire de la famille Murphy, le groupe peut découvrir que Jill et Russel, deux petits-enfants d'Audie, se disputent l'héritage de leur grand-père. Ils possèdent deux entreprises rivales d'armement (Murphy Weaponry et Murphy Gunsmithing). 
Chacun tire la couverture à soi et se targue d'être l'héritier légitime du héros de guerre, n'hésitant pas à utiliser son modèle holographique pour faire la publicité de leurs produits et d'exploiter son image à des fins mercantiles.

Il n'est pas bien compliqué de conclure que c'est Russel qui a embauché le groupe pour voler la veste.
Sa sœur, qui la possédait jusqu'à récemment, l'a en effet offerte au musée en grandes pompes deux ans plus tôt, en échange d'un juteux contrat d'équipement de la garde nationale.
Ce qui a échappé à la vigilance des deux héritiers, c'est Gina Costello, petite-fille d'Audie par adoption dont ils ignorent même l'existence, s'empare peu à peu de l'héritage de son grand-père.
En remontant la piste de la fausse veste, remplacée par Gina lorsqu'elle se faisait passer pour une assistante maternelle chez Jill Murphy, les personnages finiront par rencontrer mademoiselle Costello.
Celle-ci s'est lassée de voir ses cousins se crêper le chignon et manquer de respect à celui qu'elle voit comme un soldat au grand cœur, défenseur des victimes de PTSD et du soutien psychologique aux vétérans.
Elle rassemble lentement et sûrement les médailles, journaux et effets personnels ayant appartenu à Audie dans son garage.

Toutefois, Gina n'a hélas pas grand chose à offrir aux personnages à part sa gratitude.
Elle connaît néanmoins plutôt bien les villas respectives de Jill et Russel dans lesquelles elle s'est faite passer pour une employée de service afin d'accomplir son œuvre.
Si le groupe est sensible à son histoire, elle ne sera pas avare de conseils sur les possibilités d'infiltration dans la demeure familiale des Murphy et plus particulièrement dans les collections privées de ses cousins.
Avec son aide, les \emph{runners} pourraient bien récupérer les dernières médailles encore en possession des héritiers pourris gâtés -- et se servir au passage dans les œuvres d'art et armes de collection inestimables que leurs coffres recèlent.

\subsection{Résolution}

Si vos joueuses incarnent des personnages sans foi ni loi et sans la moindre once d'empathie, il faut bien admettre que le groupe ne devrait faire qu'une bouchée de Gina et aisément lui reprendre la veste.
Dans ce cas, vous pouvez leur faire comprendre que le crime ne paie pas en envoyant à leurs trousses des mercenaires payés par Russel pour les éliminer, faisant disparaître toute trace de son méfait.

En revanche, dans le cas contraire, c'est l'arroseur arrosé: cambriolage de haut vol dans le manoir de Murphy, pillage en règle des coffres et des pièces de collection revendues à prix d'or pour un montant au-delà de leurs espérances.
Même dans un monde cyberpunk noir et cynique, avoir un cœur, parfois, ça paie.

\illustration[0.25\textwidth]{soldier}

\chapter{Les dévalisés du rail}
\keywords{\western}{\action}{\index[theme]{Crime}Crime, \index[theme]{Technologie}technologie}

\section{Scénario}

Tout le monde connaît cette histoire récurrente des westerns: un train bloqué par un arbre tombé sur les rails, un groupe de hors-la-loi qui débarque à cheval et qui dépouille les voyageurs.
Un braquage de train tout ce qu'il y a de plus classique.
Et si aujourd'hui vos joueuses s'intéressaient non pas au contenu, mais au contenant?

\subsection{Accroche}

La compagnie ferroviaire Southern Pacific Transportation Company embauche les personnages pour\dots{} voler un train.

\subsection{Péripéties}

La S.P.T.C. est en concurrence avec l'Union Pacific, l'autre grande compagnie ferroviaire de l'Ouest américain.
En perte de vitesse, elle aurait bien besoin d'un atout pour s'étendre plus vite que sa rivale.
Ce qui est plaisant, c'est que l'atout désiré lui est tombé tout cuit dans le bec dans le journal de la veille: le constructeur Baldwin Locomotive Works prévoit une démonstration de leur nouvelle locomotive dans trois jours.
Leur nouvelle machine à vapeur est, d'après l'article dans le canard, plus puissante, puis rapide, plus économe et plus facile d'entretien que toutes les autres. Et la démonstration aura lieu sur les rails de l'Union Pacific sur le tout nouveau tronçon reliant Reno, dans le Nevada, à Sacramento en Californie.

La direction de la S.P.T.C. paie grassement le groupe pour détourner la loco, par exemple en la faisant changer d'aiguillage avant l'arrivée en gare de Sacramento pour l'envoyer vers Stockton où il serait facile de la transborder sur le fleuve et de la faire disparaître.
Les personnages auront également pour mission de trouver (ou de convaincre) un autre fabricant de machine à vapeur susceptible de copier le prototype volé et de fournir la S.P.T.C. en contrefaçons à des tarifs bien plus faibles que Baldwin.
Grâce à cet avantage matériel et au coup à l'image de l'Union Pacific que le vol provoquera, la Southern Pacific espère bien supplanter sa rivale.

Le point noir de l'affaire, c'est que le prototype de la Baldwin est encore bien loin de fonctionner comme attendue, en dépit des promesses de son fabricant.
La locomotive a de nombreuses défaillances qui la rendront difficile à manœuvrer quand les personnages en auront pris le contrôle.
Qui plus est, Baldwin, s'étant trouvé dos au mur, a engagé une bande de hors-la-loi pour saboter la démonstration.
L'objectif de la manœuvre: empocher l'assurance et gagner quelques mois de répit, le temps de faire des ajustements sur son prototype sans craindre l'ire de ses investisseurs.

Le tableau de la page \pageref{table:rail} liste les événements qui vont mettre des bâtons (de dynamite) dans les roues de vos joueuses.

\subsection{Résolution}

Jonglez entre les deux sources d'un problème: d'un côté, la locomotive est incontrôlable, de l'autre, des bandits se préparent à faire sauter les rails.
Que les personnages arrivent à détourner la locomotive ou pas ne devrait pas avoir d'importance à la fin du scénario, le simple fait de sauver sa peau quand les circonstances se sont liguées contre soi est déjà une belle réussite!

\begin{table}
	\caption{Événements aléatoires durant le détournement du train}
	\label{table:rail}
	\colortablerows
	\begin{tabularx}{\textwidth}{cX}
	d6 & Événement\\
	1  & Deux hors-la-loi infiltrés par Baldwin dans le train tentent de prendre le contrôle du wagon. Après avoir dépouillé les passagers, ils se dirigent vers la loco pour la saboter.\\
	2  & L'orientation des rails à la prochaine jonction n'est pas la bonne et la personne en charge de l'aiguillage ne répond pas aux signaux du train.\\
	3  & La connexion entre le wagon contenant la réserve de charbon et la loco a été endommagée et ne pas tarder à lâcher.\\
	4  & Les hors-la-loi embauchés par Baldwin ont abattu deux troncs d'arbre sur les rails. Pas sûr que la locomotive y résiste.\\
	5  & Un des membres de l'équipage qui s'occupe de la loco enclenche la purge du moteur, vidant les réserves d'eau sur les rails. Sans eau, pas de vapeur, sans vapeur, pas de traction!\\
	6  & Un notable invité pour assister à la démonstration en tant que passager fait un malaise et a besoin de soins urgemment.\\
	\end{tabularx}
\end{table}

\vfill
\illustration[0.4\textwidth]{train}
\vfill

\chapter{La montagne décapitée}
\keywords{Super-héroïque}{Action}{Négociation}

\section{Scénario}

Commentaire

\subsection{Accroche}

Les personnages sont appelés à la rescousse pour récupérer le sommet du Scafell Pike, la plus haute montagne d'Angleterre.

\subsection{Péripéties}

 Un artiste a payé une bande de supers pour décapiter la montagne et l'exhiber son sommet découpé dans une installation artistique dans une galerie privée de Londres.

Les journaux ont bien entendu fait les choux gras de cette histoire. La « décapitation » de la montagne a retiré environ 2 mètres d'altitude au sommet. L'artiste n'a pas vraiment l'intention de rendre la caillasse et n'a d'ailleurs enfreint aucune loi au sens strict, en dépit des allégations des représentants locaux. Les personnages n'ont donc pas d'autorité particulière pour le récupérer, à part leur sens du devoir.

Charge aux personnages donc de trouver une façon de récupérer le sommet et surtout de le réinstaller à sa place de façon durable. Se doutant que le sommet risque d'être convoité, l'artiste continue à payer son groupe de supers pour assurer sa protection et celle de son « œuvre ».


\subsection{Résolution}

\illustration{mountain}

\chapter{Trop beau pour être vrai}
\keywords{Western}{Intrigue}{Enquête, crime}

\section{Scénario}

Commentaire

\subsection{Accroche}

Alors que les personnages font la fermeture d'un saloon, un type mal en point débarque à cheval, se laisse glisser à terre, fait quelques pas dans la salle et tombe avant de se laisser mourir sur le plancher. Personne ne le connaît ici et il semble venir d'une ville voisine.

\subsection{Péripéties}

Il porte une sacoche en cuir en bandouillère contenant des dizaines de milliers de dollars en billets de banque. Les personnages peuvent les remettre au shérif, se les partager (à défaut, comme le tenancier du saloon en gardera une partie pour le donner à quelques habitants proches des personnages, voire les personnages eux-mêmes, et donnera le reste au shérif). Toutefois, dès le lendemain, de mystérieux étrangers vont débarquer en ville. Ils cherchent visiblement l'individu de la veille, sans pour autant faire de mention du pactole.

Le hic, c'est que les billets sont contrefaits. Les étrangers sont des agents fédéraux du Service Secret, nouvellement créé pour enquêter sur la fausse monnaire qui circule dans l'Ouest. Leur investigation est discrète : ils ont blessé le faussaire et un complice la veille mais il est parvenu à leur échapper à cheval. Ils soupçonnent que d'autres de ses complices se cachent en ville. Les personnages vont-ils coopérer ? Tenter de cacher le magot ? Ou peut-être eux-mêmes ou leurs proches font partie du groupe de faussaires…

\subsection{Résolution}

\illustration{cash}

\chapter{L'indépendance gronde}
\keywords{Steampunk}{Action}{Science, exploration, guerre}

\section{Scénario}

Commentaire

\subsection{Accroche}

Les personnages accompagnent l'explorateur et scientifique Alexander Von Humboldt au milieu de l'orange permanent de Catatumbo pour une expérience scientifique. Humboldt expérimente avec la foudre et le stockage de l'énergie dans des énormes piles à cristaux pour ensuite alimenter ses autres expériences. Le bateau a été ainsi positionné dans le lac de Maracaibo et muni de paratonnerres gigantesques pour capter les éclairs.

\subsection{Péripéties}

Toutefois, le Venezuela est pendant ce temps en proie à la guerre d'indépendance, l'Amérique latine s'étant soulevée contre l'empire espagnol. Ainsi, les espagnols ont installé un blocus à l'embouchure du golfe du Venezuela. Le bateau et l'équipage est donc de fait retenu captif, bien que Humboldt soit satisfait de la situation car ses expériences avancent bien et son projet est à deux doigts d'être terminé.

Malheureusement, profitant d'une sortie de Humboldt pendant un ravitaillement, les indépendantistes colombiens et vénézueliens kidnappent le savant afin de le contraindre à rejoindre leur révolution. En parallèle, le bateau est attaqué pour récupérer ses formidables expériences que l'état-major espère transformer en armes. Aux personnages de résister (ou de s'échapper), avec ou sans le baron Humboldt qui, sous ses prétentions de neutralité, se ne montrera pas insensible aux arguments des indépendantistes…

\subsection{Résolution}

\illustration{coil}


\appendix

\chapter{Tables aléatoires}

Pour plus de facilité au moment de l'improvisation, voici une sélection de noms et prénoms parmi les plus courants dans le monde entier.

\paragraph{Comment utiliser ces tables?} Il suffit de tirer 1d4 ou 1d8 pour les dizaines et un 1d10 pour les unités puis de consulter la ligne associée au nombre obtenu dans les listes ci-dessous.

\section{Noms de famille}

\begin{multicols}{3}
\begin{enumerate}
	\item Rakotomalala
	\item Molefe
	\item Andersson
	\item Chen
	\item Latu
	\item Ali
	\item Abdou
	\item Devi
	\item Martin
	\item Lopez
	\item Saint-Pierre
	\item Jónsdóttir
	\item Alaoui
	\item Perera
	\item Lopes
	\item Cohen
	\item Murphy
	\item Ibrahim
	\item Lawson
	\item Hernandez
	\item Liu
	\item Kim
	\item Papadopoulos
	\item Rossi
	\item Hansen
	\item Clarke
	\item Wang
	\item Müller
	\item Sharipov
	\item Khan
	\item Ivanova
	\item Melknik
	\item Williams
	\item Satō
	\item Da Silva
	\item Traore
	\item Trabelsi
	\item Garcia
	\item Smith
	\item Peeters
\end{enumerate}
\end{multicols}
\section{Prénoms}

\begin{multicols}{3}
\begin{enumerate}
	\item Moussa
	\item Marc
	\item Prasert
	\item Mateo
	\item Wei
	\item Fang
	\item Emma
	\item Aicha
	\item Daniel
	\item Lili
	\item Mia
	\item Stefan
	\item Nour
	\item Harper
	\item Matilde
	\item Zineb
	\item Farrah
	\item Junior
	\item Aurora
	\item Alice
	\item Karim
	\item Tamatoa
	\item Meriem
	\item Ivana
	\item Tereza
	\item Sarah
	\item Taylor
	\item David
	\item Jean-Claude
	\item Esperanza
	\item Itsuki
	\item Jayden
	\item Nicole
	\item Oscar
	\item Kamal
	\item Zoé
	\item Anna
	\item Jackson
	\item Mareva
	\item Abigail
	\item Lucas
	\item Amanda
	\item Jian
	\item Jack
	\item James
	\item Dylan
	\item Sita
	\item Elias
	\item Vicente
	\item George
	\item Angel
	\item Juliette
	\item Mihail
	\item Sarah
	\item Abdelkader
	\item Carlos
	\item Esther
	\item Liam
	\item Marie-Paule
	\item Alejandro
	\item Ayaan
	\item Jakob
	\item Logan
	\item Jade
	\item Juan
	\item Louise
	\item Omar
	\item Ha-yoon
	\item Sofia
	\item Saanvi
	\item Artyom
	\item Elizabeth
	\item Charlotte
	\item Bruno
	\item Sakura
	\item Nathaniel
	\item Noam
	\item Mei
	\item Brianna
	\item Jun-seo
\end{enumerate}
\end{multicols}

\section{Personnalité}

La liste suivante est une petite sélection d'adjectifs pour donner de la profondeur à vos personnages non-joueurs. Vous pouvez sans problème en tirer deux voire trois pour les protagonistes majeurs.

\begin{multicols}{3}
\begin{enumerate}
	\item Bout-en-train
	\item Conciliante
	\item Hédoniste
	\item Naïf
	\item Militante
	\item Courageuse
	\item Rustre
	\item Cynique
	\item Méfiant
	\item Comique
	\item Enthousiaste
	\item Manuel
	\item Menteuse
	\item Commerçant
	\item Cupide
	\item Cultivée
	\item Nihiliste
	\item Technocrate
	\item Charitable
	\item Sportive
	\item Idéaliste
	\item Séducteur
	\item Condescendante
	\item Pourri-gâté
	\item Sympathique
	\item Dépressif
	\item Excentrique
	\item Susceptible
	\item Arriviste
	\item Hypocrite
	\item Malin
	\item Baratineur
	\item Autoritaire
	\item Paternaliste
	\item Xénophobe
	\item Vicieux
	\item Sauvage
	\item Blasée
	\item Artiste
	\item Intimidant
\end{enumerate}
\end{multicols}
\begin{center}
	\includegraphics[width=0.3\textwidth]{images/phrenology}
\end{center}

\onecolumn
\begingroup
\let\clearpage\relax
\label{index}
\printindex[cadre]
\printindex[genre]
\newpage
\printindex[theme]
\begin{center}
\includegraphics[width=0.5\textwidth]{images/end}
\end{center}
\clearpage

\newpage
\vspace*{-4em}
\chapter*{Images}
\vspace{-1.5em}
\small
\newcommand\creditimg[4]{%
	#1: \href{#3}{#2} (#4)
}%
\newcommand\cczero{CC0}
\newcommand\pixabay{Pixabay}
\subsubsection*{Couverture et introduction}
\creditimg{Décorations}{Gordon Johnson}{https://pixabay.com/fr/vectors/vintage-cadre-dessin-au-trait-f\%C3\%A9e-5138621/}{\pixabay};
\creditimg{Crâne et bibliothèque}{Colleen O'Dell}{https://pixabay.com/fr/vectors/cr\%C3\%A2ne-la-biblioth\%C3\%A8que-assistant-4109212/}{\pixabay};
\creditimg{Livres}{Colleen O'Dell}{https://pixabay.com/fr/vectors/livres-la-biblioth\%C3\%A8que-plateau-4109214/}{\pixabay}


\subsubsection*{Illustrations des histoires}
\creditimg{Anneau au diamant}{j4p4n}{https://openclipart.org/detail/203345/diamond-ring}{\cczero};
\creditimg{Tronçonneuse}{JicJac}{https://openclipart.org/detail/662/chain-saw}{\cczero};
\creditimg{Couteaux dans le dos}{Bellinon}{https://pixabay.com/fr/illustrations/homme-couteau-essorage-d\%C3\%A9pression-4083527/}{\pixabay};
\creditimg{Station MIR}{David S. F. Portree/NASA}{https://commons.wikimedia.org/wiki/File:RP1357_p104_The_Mir_complex_as_of_June_1994.svg}{domaine public};
\creditimg{Pont de Sydney}{Gordon Johnson}{https://pixabay.com/fr/vectors/sydney-l-australie-pont-ville-2773440/}{\pixabay};
\creditimg{Temple grec en ruine}{Kelly}{http://www.clker.com/clipart-192060.html}{domaine public};
\creditimg{Silhouette de fée}{Gordon Johnson}{https://pixabay.com/fr/vectors/f\%C3\%A9es-f\%C3\%A9e-fantaisie-dessin-anim\%C3\%A9-2101944/}{\pixabay};
\creditimg{Silhouette de bateau}{Gordon Johnson}{https://pixabay.com/fr/vectors/navire-bateau-silhouette-maritime-5198232/}{\pixabay};
\creditimg{Engrenages}{Pete Linforth}{https://pixabay.com/fr/illustrations/engrenages-pi\%C3\%A8ces-grunge-machine-1381719/}{\pixabay};
\creditimg{Saint-Michel}{extrait de \emph{Notice sur le château, les anciens seigneurs et la paroisse de Mauvezin, près Marmande} de R.L. Alis (1887)}{https://openclipart.org/detail/255311/st-michael}{domaine public};
\creditimg{Phare}{extrait de \emph{The Story of Our Merchant Marine} de Willis J. Aboot (1919)}{https://openclipart.org/detail/11102/minots-ledge-light}{domaine public};
\creditimg{Volcan}{extrait de \emph{Biography of a Grizzly} par Ernest Seton-Thompson (1900)}{https://openclipart.org/detail/6771/volcano}{domaine public};
\creditimg{Groupe d'arbres}{warszawianka}{https://openclipart.org/detail/22448/group-of-trees}{\cczero};
\creditimg{Silhouettes d'arbres}{Gordon Johnson}{https://pixabay.com/fr/vectors/nature-paysage-silhouette-arbres-4440363/}{\pixabay};
\creditimg{Cobra}{extrait de \emph{Two Happy Years in Ceylon} par Constance Cumming (1892)}{https://openclipart.org/detail/232684/buddha-protected-by-a-cobra}{domaine public};
\creditimg{Bateau et océan}{Last-Dino}{https://openclipart.org/detail/104863/ocean-theme-papercut}{\cczero};
\creditimg{Jules César}{Gordon Johnson}{https://pixabay.com/fr/vectors/jules-c\%C3\%A9sar-romain-empereur-4206555/}{\pixabay};
\creditimg{Chèvre}{extrait de \emph{Expedição portugueza ao Muatiânvua} par Henrique Dias De Carvalho (1890)}{https://openclipart.org/detail/224531/goat}{domaine public};
\creditimg{Pleine lune}{AJ}{https://openclipart.org/detail/17926/moon}{\cczero};
\creditimg{Frégate}{Firkin}{https://openclipart.org/detail/262079/frigate}{\cczero};
\creditimg{Diable}{Librairie du Congrès}{https://openclipart.org/detail/4368/pointing-devil}{domaine public};
\creditimg{Château-fort}{extrait de \emph{Handbook of the excursions proposed to be made by the Lincoln Diocesan Architectural Society} par Edward Trollope (1857)}{https://openclipart.org/detail/276666/castle-keep-2}{domaine public};
\creditimg{Cage thoracique}{b0red}{https://pixabay.com/fr/vectors/f\%C5\%93tus-c\%C3\%B4tes-nervures-foetus-2936875/}{\pixabay};
\creditimg{Gaufrier et gaufre}{\emph{Ready-To-Use Food and Drink Spot Illustrations} par Susan Gaber (1982)}{https://openclipart.org/detail/8212/waffle-and-waffle-iron}{domaine public};
\creditimg{Boulon et vis}{NetRat}{https://openclipart.org/detail/104059/nut-bolt-monochrome}{\cczero};
\creditimg{Silhouette de soldat}{Mohammed Hassan}{https://pixabay.com/fr/vectors/guerre-mondiale-soldat-ex\%C3\%A9cuter-2827031/}{\pixabay};
\creditimg{Locomotive}{Gordon Johnson}{https://pixabay.com/fr/vectors/trains-locomotive-dessin-au-trait-4899589/}{\pixabay};
\creditimg{Montagne et soleil}{Mitchell Johnson}{https://openclipart.org/detail/90073/mountain-seal-mitchell-j}{\cczero};
\creditimg{Liasse de billets}{KTEditor}{https://pixabay.com/fr/illustrations/l-argent-de-tr\%C3\%A9sorerie-dollar-559970/}{\pixabay};
\creditimg{Électro-aimant}{Firkin}{https://openclipart.org/detail/261627/electromagnet}{domaine public};
\creditimg{Phrénologie}{Inconnu}{https://openclipart.org/detail/203415/phrenology-illustration}{domaine public};
\creditimg{End}{Colleen O'Dell}{https://pixabay.com/fr/vectors/fin-sablier-la-fin-myst\%C3\%A9rieuse-4109186/}{\pixabay}
\endgroup
\end{document}
