%
% Body text font is Palatino!
%

\documentclass[a5paper,pagesize,10pt,bibtotoc,pointlessnumbers,
normalheadings,DIV=9,twoside=false]{scrbook}

% twoside, openright
\KOMAoptions{DIV=last}

\usepackage{trajan}
 
\usepackage[french]{babel}
\usepackage[utf8]{inputenc}
\usepackage[T1]{fontenc}
\usepackage[protrusion=true]{microtype}
\usepackage[babel]{csquotes}
\usepackage{hyperref}
\usepackage{tabularx}

\usepackage[sc]{mathpazo}
\linespread{1.05} 

\usepackage{xspace}

% Indentation des paragraphes
\setlength{\parindent}{10pt}
% Sauts de ligne entre les paragraphes
\setlength{\parskip}{1.4ex plus 0.35ex minus 0.3ex}
%\setlength{\parskip}{1.4ex plus 0.35ex minus 0.3ex}

% Pas de numérotation au-delà des chapitres
\setcounter{secnumdepth}{\chapternumdepth}

\title{A book title}   
\author{Author Name} 
\date{\today} 

\begin{document}


%=========================================
\begin{titlepage}
		\centering{
			{\fontsize{40}{48}\selectfont 
			A book title}
		}\\
			
		\vspace{10mm}
		\centering{\Large{Author Name}}\\
		\vspace{\fill}
		\centering \large{2011}
\end{titlepage}


%=========================================
\newpage{}
\thispagestyle {empty}

\vspace*{2cm}

\begin{center}
	\Large{\parbox{10cm}{
		\begin{raggedright}
		{\Large 
			\textit{Do what you think is interesting, 
			do something that you think is fun and worthwhile, 
			because otherwise you won’t do it well anyway.}
		}
	
		\vspace{.5cm}\hfill{---Brian W. Kernighan}
		\end{raggedright}
	}
}
\end{center}

\newcommand\keywords[3]{%
	\section*{Mots-clés :}

	\textbf{Cadre} : #1

	\textbf{Genre} : #2

	\textbf{Thème} : #3
}%

\newcommand\medfan[1]{%
	\emph{medfan}
}%

\newpage


%=========================================
\chapter*{Préambule}

Cet ouvrage est un ensemble de 31 scénarios hétéroclites conçus pour être joués dans des jeux de rôle sur table.
La plupart de ces histoires sont volontairement ouvertes, laissant aux joueurs et joueuses le soin de combler les vides et de préciser les flous par leur imagination.
Grâce à cette liberté d'interprétation, les scénarios ne sont attachés ni à des systèmes spécifiques, ni à des univers particuliers.

Chaque scénario est décrit par trois mots-clés :
\begin{itemize}
	\item le cadre dans lequel il a été imaginé,
	\item le genre d'histoire racontée,
	\item le thème de l'aventure.
\end{itemize}

Un index en fin d'ouvrage permet de retrouver la liste des scénarios relevant des différents mots-clés.
Ceux-ci sont toutefois à prendre comme des indications et non des obligations.
La plupart des aventures peuvent aisément être transposées d'un cadre à un autre voire d'un genre à un autre.

Bonne lecture !

\chapter{L'anneau gardien}
\keywords{\medfan}{Aventure}{Objet magique, malédiction, altruisme}

\section{Scénario}

\subsection*{Commentaire}

Ce scénario peut être facilement joué en parallèle d'une campagne \emph{medfan}, il suffit qu'un personnage obtienne l'anneau du titre.
C'est encore mieux si les joueurs l'utilisent régulièrement de leur propre chef.

La guerrière peut être remplacée par n'importe quelle figure combattante du moment qu'elle est suffisamment puissante pour servir d'ange gardien.

\subsection*{Accroche}

Les personnages entrent en possession d'un anneau magique.

\subsection*{Péripéties}

À chaque fois que la personne qui porte l'anneau est en danger de mort, une puissante guerrière se matérialise à proximité pour la tirer de ce mauvais pas.
Une fois le porteur en sécurité, la guerrière disparaît sans un mot et se contente de jeter un regard furieux en direction du groupe.

Chaque invocation semble l'énerver encore plus mais tout effort de lui parler est vain: elle ne parle pas et ne semble de toute façon pas les comprendre.

Au fil du temps, certaines de ses apparitions deviennent étranges. Parfois, la guerrière apparaît sans armes ni armures.
Une fois, elle se matérialise même un morceau de poulet à la main.

Un jour, elle finit par se matérialiser, tenant un morceau de papier à la main écrit dans une langue étrangère.
Après l'avoir déchiffré, le message dit ceci:
\blockquote{L'anneau est maudit. J'ai une famille et une vie. Je n'ai pas demandé à servir d'ange gardien. Le forgeron qui l'a créé est prisonnier des geôles royales. Trouvez-le et faites-lui lever la malédiction. S'il vous plaît.}


\textbf{Explications}: le forgeron est un sorcier malchanceux fuyant la guerre qui ravage une nation voisine.
Craignant pour sa vie, il a embauché des mercenaires pour l'escorter jusqu'au royaume des personnages mais alors que l'argent est venu à manquer, il s'est retrouvé sans aucune protection.
Pour assurer ses arrières, il n'a alors rien trouvé de mieux pour assurer sa sécurité que de lier l'âme d'une grande aventurière à la retraite -- croisée au hasard de son voyage -- à son anneau.

Une fois arrivé, le sorcier-forgeron a posé ses valises dans la capitale et s'y est établi comme fabriquant d'objets magiques.
Malheureusement, n'étant pas un bon gestionnaire, il s'est rapidement retrouvé criblé de dettes auprès du royaume, incapable d'honorer les commandes du gouvernement.
La milice l'a alors mis en prison avant de piller son échoppe et de vendre ses biens aux enchères pour rembourser ses dettes.
De fil en aiguille, l'anneau a ainsi échappé à son propriétaire et la guerrière subit tant bien que mal les aventures de son porteur, régulièrement importunée par ces invocations involontaires.

\subsection*{Résolution}

Plusieurs façons de lever le sortilège sont envisageables.
Si les personnages sont versés en magie, peut-être qu'un rituel impliquant la guerrière en personne pourrait briser le lien entre elle et l'anneau.
Ou bien peut-être qu'il suffirait de substituer une nouvelle âme pour libérer celle qui se trouve actuellement liée.
Enfin, en retrouvant la trace du sorcier, celui serait sûrement prêt à annuler son sort si on le sort des geôles, en payant ses dettes\dots ou bien par la force.

\end{document}
