\chapter{Un prêté pour un rendu}
\keywords{Générique}{Enquête}{Ésotérique, malédiction}

\section{Scénario}

Commentaire

Les protagonistes de ce scénario sont prévus pour être aisément remplaçables. L'âme précieuse peut être Hippolyte, Arjuna, Boadicée, Arthur, etc. On peut imaginer le Diable joué par Hadès dans le panthéon grec, Kali dans le panthéon hindou et ainsi de suite. Vous pouvez bien sûr faire intervenir vos propres déités!

\subsection{Accroche}

Il y a des années, les personnages ont vendu leur âme au Maître des Enfers (chacun pour une raison qui lui est propre et peut-être même oubliée depuis).
Aujourd'hui, le Diable apparaît devant eux\dots pour leur demander un service.

\subsection{Péripéties}

Une âme lui échappe et un accord millénaire lui interdit, ainsi qu'à ses sous-fifres, d'aller la chercher par lui-même.
Heureusement, il a sous le coude quelques mortels qui lui doivent un service.

Le Diable n'a qu'un nom et les circonstances de la mort.
Le groupe aura donc pour tâche de retrouver cette âme égarée et devra ainsi remonter le fil: où est-elle partie?
Pourquoi faire?
Et comment la convaincre d'accepter son sort?
Cela n'aura rien d'une partie de plaisir car l'âme est également recherchée et protégée par des chérubins, des anges ne pouvant agir directement sur les personnages mais capables d'influencer le monde et les pensées des humains.

Car l'âme en question est convoitée!
Le fantôme en question n'est autre que le juif errant, témoin mythique de la crucifixion, frappé d'immortalité jusqu'au retour du Christ sur Terre.
Les siècles passant, son corps s'est délité mais son âme continue à vaquer.
Rejeté par l'humanité qui ne semble même plus le voir, il erre dans un pèlerinage infini, semant involontairement chaos et dévastation sur son passage là où démons et chérubins s'affrontent en espérant le retrouver.


\subsection{Résolution}
