\chapter{Un prêté pour un rendu}
\keywords{\generique}{\enquete}{\index[theme]{Ésotérisme}Ésotérique, \index[theme]{Malédiction}malédiction, \index[theme]{Divinité}divinité}

\section{Scénario}

Cette aventure est originale surtout de par son accroche initiale et peut s'inscrire dans de très nombreux univers.
Tous les protagonistes de ce scénario sont prévus pour être aisément remplaçables. L'âme précieuse peut être Hippolyte, Arjuna, Boadicée, Arthur, etc. On peut imaginer le Diable joué par Hadès dans le panthéon grec, Kali dans le panthéon hindou et ainsi de suite. Vous pouvez bien sûr faire intervenir vos propres déités!

\subsection{Accroche}

Il y a des années, les personnages ont vendu leur âme au Maître des Enfers (chacun pour une raison qui lui est propre et peut-être même oubliée depuis).
Aujourd'hui, le Diable apparaît devant eux pour leur demander un service.

\subsection{Péripéties}

Le Diable a besoin d'eux car une âme lui échappe et un accord millénaire lui interdit d'aller la chercher par lui-même.
Heureusement, il a sous le coude quelques mortels qui lui doivent un service et qui pourront agir à sa place.

Le Diable n'a qu'un nom et les circonstances de la mort ayant amené cette âme à quitter son corps.
Le groupe aura donc pour tâche de retrouver cette âme égarée et devra ainsi remonter le fil: où est-elle partie?
Pourquoi faire?
Et comment la convaincre d'accepter son sort?
Pour les aider dans leur quête, le Diable prête aux personnages le pouvoir de vision des âmes, qui leur permettra ainsi de retrouver celle qu'ils cherchent mais aussi de suivre à la trace les manifestations \og transcendantales \fg laissées derrière elle.
Cela n'a rien d'une partie de plaisir car la victime en question est également recherchée par des chérubins, des anges ne pouvant agir directement sur les personnages mais capables d'influencer le monde et les pensées humaines.

L'âme qu'ils recherchent est en effet particulièrement convoitée!
Le fantôme n'est autre que le juif errant, témoin mythique de la crucifixion et frappé d'immortalité jusqu'au retour du Christ sur Terre.
Les siècles passant, son corps s'est délité et, par accident ou à cause du poids des années, a fini par être détruit.
Son âme continue toutefois à vaquer sur Terre. Sa santé mentale a périclité et c'est à peine si le fantôme a conscience de son existence.
Rejeté par une humanité qui ne semble même plus le voir, il erre dans un pèlerinage infini, semant involontairement chaos et dévastation sur son passage là où démons et chérubins s'affrontent en espérant retrouver sa trace.

\subsection{Résolution}

\begin{itemize}
	\item L'âme damnée est remise au Diable: les personnages sont libérés de leur pacte démoniaque. Cette épée de Damoclès disparaît de leur tête et ils peuvent profiter de la vie en sachant que les tourments éternels de l'Enfer ne sont plus une fatalité. En revanche, il n'est pas impossible que certains anges aient une dent contre eux.
	\item L'âme damnée continue à errer: le statu quo est préservé. Si les personnages semblent avoir fait un réel effort (c'est-à-dire que leur échec est dû à l'intervention divine), peut-être le Diable consentira-t-il à annuler le pacte. Après tout, l'immortalité du damné ne durera que jusqu'au retour du messie, peut-être que les personnages survivront d'ici là\dots
	\item L'âme damnée est envoyée au paradis: sa longue errance prend fin. Les chérubins ont eu pitié de la pauvre âme ou bien celle-ci a finalement trouvé le messie. Le Diable sera furieux mais peut-être trouveront-ils asile auprès des serviteurs de Dieu?
\end{itemize}

\subsection*{Personnage: l'âme damnée}
\descriptionperso{Errante infatigable et solitaire}{Piété, immortalité}{Paria, incorporelle}{Trouver le salut lors du retour du messie}{L'âme damnée a été maudite et rendue immortelle, toutefois cela n'a pas empêché le corps qu'elle habitait de s'user (par le fait du temps ou à cause d'un accident l'ayant réduit en poussières). L'humanité l'ignore et elle traverse l'espace et le temps sans laisser de trace.}

\illustration[0.17\textwidth]{devil}

