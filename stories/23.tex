\chapter{La forteresse anachronique}
\keywords{Post-apocalyptique}{Intrigue}{Complot}

\section{Scénario}

Commentaire

\subsection{Accroche}

Cent ans après le cataclysme, les personnages sont au milieu d'un long et difficile voyage quand un violent orage éclate.
Leurs provisions sont détrempées, leurs moyens de locomotion difficilement utilisables, bref, le groupe doit trouver un endroit où s'arrêter.
Fort heureusement, une communauté locale les invite dans leur refuge.


\subsection{Péripéties}

Les locaux ont réaménagé à leur sauce un véritable château-fort médiéval.
En dépit de son apparence antédiluvienne, la forteresse leur permet d'être épargnés par les raids de bandits et la place abonde pour cultiver un potager et même élever des volailles.
Malheureusement, le piège se referme sur le groupe en même temps que les portes du château.

Car si la communauté du château accueille les personnages, ce n'est pas seulement par bonté d'âme.
Depuis des semaines, une seigneure de guerre voisine convoite leur forteresse et leurs provisions et exige un tribut, sans quoi elle assiègera le château.
Hélas la petite communauté est loin de pouvoir faire face, ni en effectifs, ni en armement.
Ils espèrent donc que les personnages, visiblement des mercenaires endurcis, pourront les aider.

Après avoir pris en otage quelque chose qui leur est cher (la personne qu'ils escortent, leurs biens précieux, etc.), le conseil de la forteresse exige que le groupe trouve une solution à leur problème.
Leur suggestion est des plus brutales: se rendre au campement de la cheffe de guerre et l'assassiner.

Aux personnages de trouver une façon de résoudre le conflit.
Le problème est plus épineux qu'il en a l'air: la cheffe de guerre n'est pas belliqueuse par principe mais cherche de quoi nourrir l'amalgame d'indigents et de réfugiés que son armée a pris sous son aile.
Quant à la communauté de la forteresse, ils emmagasinent bien plus de provisions que nécessaires mais celles-ci sont monnayées à prix fort pour acheter esclaves et matériel technologique à leurs voisins.

\subsection{Résolution}

\illustration{castle}
