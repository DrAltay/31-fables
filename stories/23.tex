\chapter{La forteresse anachronique}
\keywords{\postapo}{\intrigue}{\index[theme]{Complot}Complot, \index[theme]{Diplomatie}diplomatie, \index[theme]{Voyage}voyage}

\section{Scénario}

Un scénario post-apo qui s'accommode de beaucoup de cadres différentes.
Une tempête surprend le groupe au milieu d'un long périple, les contraignant à chercher refuge auprès de bâtisses non loin.
Le motif du voyage des personnages n'a pas d'importance, il s'agit d'un prétexte à l'aventure.
Un des points d'intérêt est de jouer sur le décalage entre le perçu et le réel: la régression technologique rend tout à fait plausible de rencontrer une communauté agricole vivant dans un système féodal moyenâgeux. Et pourtant, la modernité est bien là: extorsion sur les prix, concentration des moyens de production, privatisation des avancées technologiques.

\subsection{Accroche}

Cent ans après le cataclysme, les personnages sont au milieu d'un long et difficile voyage quand un violent orage éclate.
Leurs provisions sont détrempées, leurs moyens de locomotion difficilement utilisables, bref, le groupe doit trouver un endroit où s'arrêter.
Fort heureusement, une communauté locale les invite dans leur refuge.


\subsection{Péripéties}

En l'occurrence, les locaux qui leur offrent l'abri semblent avoir réaménagé à leur sauce un véritable château-fort médiéval.
En dépit de son apparence antédiluvienne, la forteresse permet à leur communauté d'être épargnée par les raids des bandits qui zonent dans les parages.
La place forte est plutôt confortable puisqu'une grande cour intérieure leur laisse même une place plus que suffisante pour cultiver un potager et élever quelques volailles.
Malheureusement pour le groupe, cette générosité a un prix et le piège s'est refermé sur eux en même temps que les portes du château.

Si la communauté du château accueille les personnages, ce n'est pas par bonté d'âme mais parce qu'ils espèrent bien profiter d'une bande de mercenaires dans leurs rangs pour le siège qui s'annonce.
En effet, depuis des semaines, une seigneure de guerre à la tête d'une horde nomade menace leur forteresse.
Elle convoiterait le château et surtout leurs provisions: elle et ses barbares exigent un tribut sans quoi le château sera assiégé et mis à sac.

La petite communauté ne comporte hélas que quelques dizaines d'individus, bien loin de pouvoir faire face aux centaines de pillards qui sont à leurs portes.
L'armement n'est par ailleurs pas non plus leur point fort, tout du moins c'est ce qu'ils diront aux personnages.
Ils implorent donc que le groupe, visiblement composé de mercenaires endurcis, les sorte de ce pétrin.
Leur discours sonne toutefois quelque peu faux et vos personnages doivent se douter que quelque chose cloche.
Au détour d'une porte entrebâillée rapidement fermée, le groupe peut par exemple apercevoir quelques armes ou du matériel électronique de récupération plutôt avancé pour une petite communauté agricole.

Voyant les personnages danser d'un point sur l'autre, l'hospitalité se fera plus perverse.
Bien que polis et accueillants en façade, les locaux vont prendre en otage quelque chose de précieux aux yeux du groupe (la personne qu'ils escortent, leurs biens précieux, etc.).
À demi-mots, les sages à la tête de la petite communauté exige qu'ils trouvent une solution à leur problème, sans quoi qui sait ce qu'il adviendra de leur cargaison?
La suggestion des notables du coin est des plus brutales: se rendre au campement de la cheffe de guerre et l'assassiner.

Bien entendu, le problème est plus épineux qu'il en a l'air et l'exposé de la situation est particulièrement biaisé.
La cheffe de guerre en question n'est pas belliqueuse par principe: elle cherche simplement de quoi nourrir l'amalgame d'indigents et de réfugiés que son armée a pris sous son aile.
Quant à la communauté de la forteresse, ils emmagasinent bien plus de provisions que nécessaires, possédant les rares terres cultivables des environs qui ne soient pas soumises aux raids perpétuels.
En accumulant des réserves afin de limiter artificiellement l'offre, les produits de la terre que le château produit sont monnayées à prix fort auprès des organisations voisines en échange d'esclaves et de matériel technologique de pointe (ingrédients chimiques pour la production d'engrais, outillage motorisé, systèmes d'arrosage\dots).

\subsection{Résolution}

S'infiltrer discrètement dans le camp des indigents et tuer leur cheffe est bien entendu une possibilité, cependant vos joueuses devraient avoir suffisamment d'indices dans la narration pour réaliser que le tableau qui leur a été dépeint n'est pas si manichéen.

Discuter avec les nomades devrait l'occasion d'échafauder le plan inverse: retourner au château, récupérer leurs biens retenus en otage et voler des provisions parmi les amples réserves des habitants de la forteresse. De quoi contenter presque tout le monde\dots

\subsection*{Personnage: la cheffe de guerre}
\descriptionperso{Main de velours dans un gant de fer}{Charisme, stratégie}{Tact, finesse}{Faire survivre son clan}{Dirigeante d'une cohorte nomade hétéroclite, elle cherche de quoi nourrir ses protégés. Cependant, les villes alentours refusent de commercer avec ce qu'ils pensent être des pillards armés et elle se voit contrainte d'envisager de prendre par la force les provisions dont elle a besoin. Ses cicatrices et son attitude agressive forment une façade intimidante cachant le souci de prendre soin de siens}

\vfill
\illustration[0.55\textwidth]{castle}
\vfill
