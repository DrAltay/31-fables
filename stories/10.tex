\chapter{Le réseau Saint-Michel}
\keywords{\generique}{\enquete}{\index[theme]{Ésotérisme}Ésotérisme, \index[theme]{Magie}magie, \index[theme]{Légende}légende}

\section{Scénario}

Ce scénario s'appuie sur une croyance ancienne qu'il existe des abbayes dédiées à Saint Michel un peu partout en Europe et en Asie mineure et que celles-ci sont reliées par une puissante magique.
Cette idée peut s'adapter à n'importe quelle époque et à beaucoup de cadres différents.
L'ambiance ésotérique peut s'accoutumer à une magie mythique aussi bien qu'à du surnaturel assumé.

\subsection{Accroche}

Si les personnages connaissent sans doute le Mont Saint-Michel, monument français de renommée mondiale, le Saint Michael's Mount leur est probablement inconnu. Et pourtant, cette île des Cornouailles en Grande-Bretagne abrite une abbaye en tout point similaire au Mont Saint-Michel dont elle semble être le pendant britannique.

C'est là qu'un manuscrit très ancien et très précieux a été volé durant une effraction d'une grande brutalité.

\subsection{Péripéties}

Le cambriolage est teinté de mystère, les indices et les témoignages laissent supposer que d'étranges phénomènes ont eu lieu: disparition du manuscrit d'un coffre fermé à clé, étranges feux follets aperçus aux abords de l'abbaye et moine décapité.

Rapidement, on parlera aux personnages d'un autre incident de la sorte ayant eu lieu quelques jours plus tôt dans les ruines du monastère de l'île Skellig Michael en Irlande.
Cela les lancera sur une piste qui suit l'axe tellurique reliant les différents Monts Saint-Michel d'Europe: l'île Skellig (Irlande), le St Michael's Mount (Angleterre), le mont Saint-Michel (France), le Mont-Gargano (Italie) et le château hospitalier de l'île de Délos (Grèce).
Leur destination finale sera bien sûr l'emplacement mythique du combat de saint Georges contre le dragon en Lydie (actuelle Turquie).

En effet, une puissance kabbale de sorcellerie s'approprie différents fragments d'un manuscrit leur permettant de canaliser la puissance ancienne des dragons.
Chaque fragment se situe dans un des \og monts \fg  et les rassembler leur permettrait ainsi de lancer le rituel.
Leur objectif final: déchaîner une armée de dragons sur Terre et conquérir le monde.

\subsection{Résolution}

Lorsque les personnages rejoignent le lieu du rituel en Lydie, celui-ci doit d'ores et déjà être en cours d'exécution.
Décrivez un festival de sons et de lumières alors que des griffes de dragons commencent à émerger d'un puits de lave au milieu des montagnes d'Anatolie.
Ensuite, à voir comment les personnages s'y prennent!
C'est le moment de passer à l'action pour stopper le rituel et déjouer les manigances de la kabbale.
À vous de voir quels sont les moyens de cette dernière (mercenaires en grand nombre ou quelques mages à la puissance incommensurable)\dots

\vfill
\illustration[0.55\textwidth]{saint_michel}
\vfill
