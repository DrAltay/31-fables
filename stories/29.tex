\chapter{La montagne décapitée}
\keywords{\superheros}{\action}{\index[theme]{Diplomatie}Diplomatie, \index[theme]{Effraction}effraction}

\section{Scénario}

L'art contemporain fait rarement l'unanimité.


\subsection{Accroche}

Les personnages sont appelés à la rescousse pour récupérer le sommet du Scafell Pike, la plus haute montagne d'Angleterre.

\subsection{Péripéties}

Récupérer un sommet rocheux peut sembler étrange au premier abord mais c'est pourtant bien de cela que parle leur commanditaire.
Un artiste britannique un peu décalé a payé une bande de super-mercenaires pour arracher son sommet à la montagne.
Le \og chapeau \fg ainsi découpé est depuis quelques semaines exposé comme une installation artistique dans une galerie privée de Londres.
Les tabloïdes et les journaux spécialisés en ont fait leurs choux gras et des milliers de touristes se pressent aux portes de l'exposition, asticotés par la curiosité.

La \og décapitation \fg de la montagne a retiré environ 2 mètres au Scafell Pike, qui ne culmine désormais plus qu'à \SI{976}{\meter} d'altitude.
Forcément, l'office du tourisme local parle d'un véritable massacre et la polémique enfle, entre autre car le mont chute ainsi de la 257\ieme place à la 268\ieme dans le classement des sommets les plus hauts des îles britanniques.
L'artiste n'a pas vraiment l'intention de rendre la caillasse et n'a d'ailleurs enfreint aucune loi au sens strict, en dépit des allégations des autorités face aux journalistes.
Les personnages n'ont donc pas de légitimité particulière dans cette opération de sauvetage au-delà de leur sens du devoir et leur éventuel chauvinisme.

Leur mission est donc de récupérer le sommet et -- surtout -- de le remettre à sa place de façon durable.
Discrétion, délicatesse et efficacité sont les maîtres mots de cette entreprise car une intervention trop visible de leur part risquerait d'enflammer encore plus des médias déjà à fleur de peau sur le sujet.
Bien entendu, l'artiste se doute que son œuvre est convoitée et continue donc à payer son groupe de supers pour assurer sa protection et celle de son rocher chéri.

\subsection{Résolution}

Pour résoudre parfaitement cet imbroglio, les personnages doivent agir en cochant les trois cases suivants:
\begin{itemize}
	\item Intervenir subtilement: sortir les pouvoirs ou multiplier les dégâts collatéraux dans une des galeries les plus chics de l'ombre ferait mauvais genre pour leur image et celle des autorités.
	\item Intervenir délicatement: porter atteinte à l'intégrité du rocher est une bien mauvaise idée. Il n'y aurait rien de pire que de remettre le sommet à sa place pour s'apercevoir qu'il en manque un bout!
	\item Intervenir discrètement: voler une œuvre d'art dans un musée est un crime. Super ou pas, commandités par les autorités ou non, la loi n'est pas à l'avantage des personnages.
\end{itemize}

À vous de jouer sur les conséquences que peuvent avoir un manquement dans une de ces trois catégories\dots

\subsection{Personnage: la bande à Zybil}
\descriptionperso{Mercenaires à pouvoirs}{Iconoclastes, efficaces}{Cupides, irresponsables}{Se faire payer leurs honoraires}{La bande à Zylbil est un groupe de mercenaires possédant des super-pouvoirs. Hétéroclite mais soudé, ce petit collectif se vend au plus offrant. Il surprend généralement par ses approches originales mais performantes.

\paragraph{Membres:} Zybil (yeux lasers), Atlas (force démesurée), Polycarpe (dédoublement), 6-Pack (transmutation de l'alcool en énergie pure), Magnet (lévitation des petits objets), Angelo \& Angela (liquéfaction de la matière solide et inversement).}


\illustration[0.35\textwidth]{mountain}
