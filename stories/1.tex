\chapter{L'anneau gardien}
\keywords{\medfan}{\aventure}{\index[theme]{Magie}Magie, \index[theme]{Malédiction}malédiction}

\section{Scénario}

Ce scénario peut être facilement joué en parallèle d'une campagne \emph{medfan}, il suffit qu'un personnage obtienne l'anneau tutélaire lors de l'une de ses aventures.
C'est encore mieux si les joueuses utilisent régulièrement l'anneau de leur propre chef.

\subsection*{Accroche}

Les personnages entrent en possession d'un anneau magique.

\subsection*{Péripéties}

À chaque fois que la personne qui porte l'anneau est en danger de mort, une puissante guerrière se matérialise à proximité pour la tirer de ce mauvais pas.
Une fois le porteur en sécurité, la guerrière disparaît sans un mot et se contente de jeter un regard furieux en direction du groupe.

Chaque invocation semble l'énerver encore plus mais tout effort de lui parler est vain: elle ne parle pas et ne semble de toute façon pas comprendre les personnages.

Au fil du temps, certaines de ses apparitions deviennent étranges.
Une fois, la guerrière apparaît sans ses armes ni son armure. Une autre fois, elle tient un morceau de poulet à moitié mangé dans sa main droite.

Finalement, le jour d'une énième invocation, elle se matérialise avec un bout de papier dans les mains.
Celui-ci est écrit dans une langue étrangère mais après l'avoir déchiffré, le message est le suivant:
\blockquote{L'anneau est maudit. J'ai une famille et une vie. Je n'ai pas demandé à servir d'ange gardien. Le forgeron qui l'a créé est prisonnier des geôles royales. Trouvez-le et faites-lui lever la malédiction. S'il vous plaît.}

En réalité, le forgeron est un sorcier malchanceux fuyant la guerre qui ravage une nation voisine.
Craignant pour sa vie, il a embauché des mercenaires pour l'escorter jusqu'au royaume où se trouvent les personnages mais alors que l'argent est venu à manquer, il s'est retrouvé sans aucune protection.
Pour assurer ses arrières, il n'a alors rien trouvé de mieux pour que de lier l'âme d'une grande aventurière à la retraite -- croisée au hasard de son voyage -- à son anneau.
Ce faisant, il l'a contrainte à devenir la garante de sa sécurité, s'achetant ainsi une certaine tranquillité.

Une fois son périple achevé, le sorcier-forgeron a posé ses valises dans la capitale et s'y est établi comme fabriquant d'objets magiques.
Malheureusement, n'étant pas un bon gestionnaire, il s'est rapidement retrouvé criblé de dettes auprès du royaume, incapable d'honorer les commandes du gouvernement.
La milice l'a alors mis en prison avant de piller son échoppe.
Ses biens furent vendus aux enchères pour éponger ses dettes et, de fil en aiguille, l'anneau a ainsi échappé à son propriétaire.
La guerrière s'accommode tant bien que mal des aventures de ses porteurs successifs, régulièrement importunée dans sa retraite par ces invocations involontaires.

\subsection*{Résolution}

Plusieurs façons de lever le sortilège sont envisageables.
Si vos personnages sont versés en magie, peut-être qu'ils seraient capable de concevoir par eux-mêmes un rituel, à effectuer en présence de la guerrière, qui pourrait briser le lien entre elle et l'anneau.
Ou bien peut-être qu'une solution serait simplement de substituer une nouvelle âme à celle actuellement liée à l'anneau.
Enfin, une troisième solution serait de retrouver la trace du sorcier.
Celui-ci est sûrement prêt à défaire sa magie en échange d'un moyen d'échapper à sa captivité, par exemple en payant ses dettes ou en usant de la force pour le faire s'évader\dots

\subsection*{Personnage: la guerrière}

\newcommand\descriptionperso[5]{%
	\begin{tcolorbox}[colback=black!1!white]
	\begin{tabular}{lcl}
		{\overlock\textbf{Caractère}} & : & #1\\
		{\overlock\textbf{Force}} & : & #2\\
		{\overlock\textbf{Faiblesse}} & : & #3\\
		{\overlock\textbf{Objectif}} & : & #4\\
	\end{tabular}

	\medskip
	#5
	\end{tcolorbox}
}%

\descriptionperso{sûre d'elle, exaspérée}{combat, stratégie}{ne parle pas la langue, maudite}{être tranquille}{Aventurière légendaire d'un pays lointain ayant raccroché les armes pour prendre sa retraite. Liée à un anneau magique, elle est régulièrement importunée par des invocations non-sollicitées}

\illustration[0.16\textwidth]{ring}
