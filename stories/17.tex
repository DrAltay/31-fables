\chapter{La bague de Caligula}
\keywords{\antique}{\intrigue}{\index[theme]{Complot}Complot, \index[theme]{Crime}crime, \index[theme]{Trahison}trahison}

\section{Scénario}

Ce scénario se déroule dans la Rome antique de Caligula.
Les personnages vont prendre de part à un des nombreux complots visant à assassiner l'empereur\dots

\subsection{Accroche}

Les personnages forment une conjuration menée par le centurion Cassius Chaerea, de la garde prétorienne de l'empereur Caligula.
Cassius paie rubis sur l'ongle les protagonistes pour leur implication dans l'assassinat de l'empereur et son remplacement par son complice, le sénateur Lucius Annius Vinicianus.

\subsection{Péripéties}

Chaque personnage est susceptible d'avoir une dent contre Caligula: les émeutes anti-juives, le gaspillage des richesses de l'État, son opposition ouverte au Sénat, la famine qui menace, la divinisation de sa sœur Drusilla ou simplement son despotisme.

Pour assassiner l'empereur, Cassius a fait fabriquer une superbe bague munie d'un saphir gravé représentant sa quatrième femme, Caesonia.
Caligula étant paranoïaque -- il a déjà échappé à plusieurs complots fomentés par certains de ses proches, Cassius espère que les personnages trouveront un moyen de lui faire offrir la bague.
Il faudra alors badigeonner l'intérieur de l'anneau d'un poison incolore et inodore mais qui, au contact de la peau pendant quelques heures, causera inévitablement la mort.

La difficulté réside non pas dans le fait d'amener cadeau jusqu'à Caligula mais dans la capacité des personnages à agir sans éveiller les soupçons des sœurs de l'empereur, Agrippine et Julia.
Non pas qu'elles veuillent protéger le chef de l'État, au contraire: elles fomentent leur propre complot pour succéder elles-mêmes à leur frère qu'elles haïssent.

\subsection{Résolution}

Les voies d'accès à l'empereur sont multiples: trouver un artisan célèbre voulant honorer Caligula, faire de la bague un cadeau diplomatique ou un tribut de guerre, lui conférer des propriétés mystiques pour titiller les superstitions de l'empereur, etc.

Si Julia et Agrippine repèrent les manigances des personnages, le groupe peut très bien décider de changer de camp et de former une alliance de circonstances.
Après tout, Lucius les paie grassement mais les sœurs de l'empereur ne sont pas sans moyens financiers.
Et que dire de l'empereur lui-même, peut-être serait-il prêt à pardonner leur implication dans la conjuration si les personnages dénonçaient leurs commanditaires\dots

\illustration[0.13\textwidth]{caesar}
