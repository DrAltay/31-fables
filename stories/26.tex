\chapter{Le pentacle aux boulons}
\keywords{Années 30}{Enquête}{Mystère, ésotérique, crime}

\section{Scénario}

Commentaire

\subsection{Accroche}

Dans l'entre-deux-guerre, un culte de magie noir décide de monter un grand coup: faire de l'Empire State Building un autel démoniaque.

\subsection{Péripéties}

Mais comment faire quand on a peu de moyens et d'influence? On utilise les outils à sa disposition, comme une usine de boulons.

Un des cultistes dirige en effet une fabrique de boulons, une parmi d'autres fournissant le chantier de l'Empire State. Infusant un peu de magie noire dans chaque boulon et avec la complicité d'une poignée d'ouvriers, le culte commence du plus grand pentacle jamais créé en exploitant les liens telluriques tissés entre leurs boulons.

Et tout cela fonctionne trop bien. Petit à petit, les effets démoniaques se répandent aux outils, aux équipes de construction et à la structure elle-même. Les accidents de chantier se multiplient à vitesse grand V. Les ouvriers menacent d'arrêter le travail, effrayés par d'étranges apparitions. Bien sûr, l'opposition au projet s'en donne à cœur joie.

Sabotage? Empoisonnement? Invasion démoniaque? Détectives, médiums, la direction du chantier est prêt à employer n'importe qui pour que cela s'arrête. Les personnages auront fort à faire pour démêler le vrai du faux dans cette histoire.

\subsection{Résolution}
