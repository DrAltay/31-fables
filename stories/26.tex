\chapter{Le pentacle aux boulons}
\keywords{\retro}{\enquete}{\index[theme]{Mystère}Mystère, \index[theme]{Ésotérisme}ésotérique, \index[theme]{Policier}policier}

\section{Scénario}

Un scénario d'enquête mêlant la construction de l'Empire State Building, l'entre-deux-guerres et des éléments fantastiques.

\subsection{Accroche}

Dans l'entre-deux-guerre, un culte de magie noir décide de monter un grand coup: faire de l'Empire State Building un autel démoniaque.

\subsection{Péripéties}

Une secte satanique aurait bien besoin de marquer les esprits et quoi de mieux pour ça qu'invoquer Moloch au milieu de New York?
La difficulté, c'est que le culte ne comporte qu'une petite dizaine de membres actifs avec peu de moyens et encore moins d'influence.
Toutefois, les membres de la cabale sont imaginatifs et vont utiliser les outils à leur disposition comme, en première ligne, une usine de boulons.

Un des cultistes dirige en effet une fabrique de boulons, une parmi d'autres fournissant le chantier pharaonique de l'Empire State Building.
Infusant un peu de magie noire dans chaque boulon et avec la complicité d'une poignée d'ouvriers rémunérés en dessous de table, le culte débute la mise en place du plus grand pentacle jamais créé.
Pour ce faire, les ouvriers disposent les pièces dans la structure du gratte-ciel à des endroits bien choisis.
Les liens telluriques font le reste, la magie noire tissant entre les boulons une véritable construction parallèle tridimensionnelle ouvrant une porte vers un autre plan d'existence.

Et tout cela fonctionne très bien -- trop bien, même.
Petit à petit, des manifestations démoniaques s'emparent des outils, des équipes de construction et commence à grignoter la structure elle-même.
À mesure que l'emprise du pentacle s'étend, les accidents de chantier se multiplient à vitesse grand V.
Les ouvriers menacent d'arrêter le travail, effrayés par d'étranges apparitions.

Et bien sûr, l'opposition au projet s'en donne à cœur joie, à commencer par les associations de voisinage et les promoteurs immobiliers concurrents pour qui tous les prétextes sont bons pour faire disparaître cette monstruosité.

Sabotage? Empoisonnement? Invasion démoniaque? La direction du chantier est prêt à employer n'importe qui, du détective au médium, pour que cela s'arrête.
Les personnages auront fort à faire pour démêler le vrai du faux dans cette histoire et remonter la piste.

\subsection{Résolution}

Les indices sont à distiller au compte-goutte à vos joueuses mais la pelote de laine est assez simple à remonter une fois que la fabrique de boulons est identifiée.
L'assaisonnement de ce scénario d'enquête consiste à parsemer ici et là l'investigation de phénomènes étranges, qui pourraient tout aussi bien s'expliquer rationnellement si vous voulez la jouer X-Files, ou bien vraiment flippants avec des démons qui apparaissent entre les étages si vous préférez le \emph{pulp} façon Appel de Cthulhu.

Pour délayer la sauce, n'hésitez pas à envoyer le groupe sur des fausses pistes: le promoteur rival véreux de mèche avec la mafia, hallucinations induites par la légionellose dans les réserves d'eau du chantier, etc.

\vfill
\illustration[0.4\textwidth]{bolt}
\vfill
