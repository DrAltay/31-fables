\newcommand\skylos{Skýlos\xspace}
\chapter{Le sommeil de \skylos}
\keywords{\medfan/Antiquité}{Aventure}{Exploration,mythologie,magie}

\section{Scénario}

Cette aventure introduit un rival imaginaire à la déesse égyptienne Bastet.
Le cadre est \medfan au sens large, le scénario est prévu pour se jouer dans une antiquité où les mythes et légendes sont réels.
Le dernier acte de l'histoire est une exploration classique d'un donjon dont la durée est modulable.


\subsection{Accroche}

Depuis des temps immémoriaux, les tribus de la région vénèrent Bast, la déesse féline, protectrice de la région, symbole de chaleur et du foyer. Des terres brûlées aux rivages de la mer sauvage, tous prient en son nom et ses créatures, les chats, vivent en symbiose avec son peuple, traquant la vermine et les protégeant des maladies. Skýlos est le dieu maudit, son ennemi juré, dont on invoque le nom que pour l'accuser des maux qui nous affligent. Le village des personnages garde l'entrée du temple où celui-ci aurait été emmuré à jamais par Bast.

Mais un beau jour, la caravane marchande apporte de troublantes et inquiétantes nouvelles. Un groupe d'étrangers a accosté et s'enfonce dans les terres de ville en ville. Les rumeurs parlent d'hommes et de femmes accompagnés d'immenses prédateurs, des chiens-loups blancs et noirs dont la taille égalerait celle des lions.

\subsection{Péripéties}

Les personnages sont envoyés consulter l'oracle, qui les met en garde : malheur et la désolation à quiconque les accompagnera dans leur funeste quête.

À leur retour au village, la délégation étrangère est là, peaux blanches, armures exotiques et terribles chiens de guerre à leurs côtés. La milice leur barre l'accès à la place centrale. La situation est tendue mais en parvenant à entamer la discussion, il est clair que nul ici n'a d'intentions belliqueuses. Au contraire. Une des étrangères s'avance et annonce d'une voix forte : \blockquote{Nous sommes au service du dieu Hundur. Nous avons voyagé longtemps pour vous trouver. Hundur nous a averti du réveil prochain de votre dieu maudit et nous sommes ici pour vous aider à l'arrêter.}

À peine ces mots prononcés, la terre tremble et, dans un vacarme terrible, les portes en pierre du temple de Skýlos se fissurent et s'écroulent. "Le temps presse." Quels mystères recèle le temple ? Le dieu maudit se réveille-t-il réellement ? Que peuvent bien savoir ces étranges personnes ? Mais qui de mieux placé que les gardiens pour braver l'interdit\dots

\subsection{Résolution}

L'exploration du temple peut être longue ou courte en fonction de vos envies.
L'idéal est de faire souffrir suffisamment les protagonistes afin de faire monter la sauce lors de la confrontation finale avec \skylos dans une lutte épique pour empêcher le dieu maudit de quitter sa prison.
À vous d'ajuster en fonction des capacités des personnages: rituel magique, destruction du temple pour ensevelir \skylos et ses sbires, déicide\dots
Récompensez les initiatives des joueurs!

