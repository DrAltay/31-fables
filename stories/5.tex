\chapter{Un pont de trop}
\keywords{\guerrefroide}{\intrigue}{\index[theme]{Diplomatie}Diplomatie, \index[theme]{Espionnage}espionnage}

\section{Scénario}

Ce scénario s'inspire d'un autre fait réel de la guerre froide. La petite ville de Vulcan\footnote{\url{https://en.wikipedia.org/wiki/Vulcan,_West_Virginia} (en anglais)} cherchait des financements fédéraux pour remplacer un pont s'étant écroulé.
Face à la difficulté d'obtenir des subventions, le maire John Robinette a fini par se tourner vers l'URSS, espérant déclencher une réaction du gouvernement.
Sa stratégie a payé puisque le parlement de Virginie-Occidentale a débloqué les fonds le jour-même.

\subsection{Accroche}

Comté de Mingo, frontière entre la Virginie-Occidentale et le Kentucky, États-Unis.
Le pont du hameau de Vulcan, qui enjambe la rivière \emph{Tug Fork}, s'est effondré il y a deux ans déjà.
Deux ans que le maire s'efforce de convaincre les autorités locales et fédérales de financer sa rénovation, sans succès.
Le gouvernement est sourd aux complaintes de la population, pour qui le pont constituait la seule voie d'accès officielle permettant de rentrer et sortir du village par la route.
Les habitants doivent désormais faire plusieurs kilomètres de détours pour traverser la rivière.
Face à l'inaction des autorités et en pleine guerre froide, le maire se tourne alors vers l'URSS pour solliciter leur aide\dots

\subsection{Péripéties}

Tout Vulcan ne parle que de la requête d'aide étrangère envoyée par les autorités municipales à l'URSS pour la rénovation du pont.
Les personnages peuvent aussi bien être un groupe d'investigation du FBI, des agents soviétiques ou de simples personnalités locales. 
Toujours est-il que, sur l'invitation de la mairie, un journaliste et une ingénieur en génie civil russes viennent d'arriver à Vulcan pour rencontrer les responsables et constater le problème de leurs propres yeux.
Bien sûr, tout cela sous le regard discret mais attentif des forces de l'ordre américaines.

Moins d'une heure après l'arrivée des émissaires soviétiques, le gouvernement de Virginie-Occidentale annonce le déblocage exceptionnel d'\SI{1,3}{\million}\$ pour le remplacement du pont.
L'affaire pourrait s'arrêter là mais, dans la soirée, l'URSS mandate une multinationale des travaux publics pour rénover en son nom le pont pour \SI{2}{\million\$}.
La course est lancée.
Qui construira le nouveau de pont de Vulcan en premier?
Tous les coups sont permis.

\subsection{Résolution}

Indépendamment de l'allégeance des personnages, l'objectif est d'assurer que le pont sera construit par leur faction.
Les moyens de pression sont nombreux: propagande dans les médias, sabotage de l'entreprise concurrente, chantage envers les responsables de la mairie, accusations de collusion avec l'ennemi, etc.
Les joueuses doivent pouvoir s'en donner à cœur joie ! Et si jamais le groupe est trop passif, il ne faut pas oublier qu'une équipe s'active de l'autre côté du rideau de fer et qu'il faudra donc déjouer les tentatives ennemies de déstabilisation.

\subsection*{Personnage: le maire de Vulcan}

\descriptionperso{rentre-dedans, meneur}{plus malin qu'on ne le croit}{pas aussi malin qu'il ne le croit}{remplacer le pont pour sa réélection}{À la tête d'une petite ville de 1500 habitants au fin fond de la Virginie Occidentale, le maire de Vulcan est un américain moyen, patriote mais méfiant vis-à-vis du gouvernement fédéral. Il n'a aucune sympathie pour l'URSS mais est suffisamment politicien pour savoir qu'un bon coup dans la fourmilière peut parfois payer.

Historiquement, il s'agit d'un comté ouvrier (mines de charbon) et Vulcan penche donc vers le parti démocrate, les habitants ayant massivement voté l'an passé pour Jimmy Carter.
La région est plutôt pauvre et en très grande majorité blanche. Les syndicats y sont encore forts en dépit du déclin industriel et de l'exode rurale.}

%\vfill
\illustration[0.75\textwidth]{bridge}
%\vfill
