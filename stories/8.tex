\chapter{Cargaison délicate}
\keywords{\contemporain}{\enquete}{\index[theme]{Terrorisme}Terrorisme, \index[theme]{Trahison}trahison, \index[theme]{Espionnage}espionnage}

\section{Scénario}

Cette enquête à la 24 Heures chrono présente deux particularités: l'urgence et la présence d'un agent double au sein du groupe.
Les investigations ne devraient pas être trop difficiles, ici c'est l'aspect \og course contre la montre \fg qui prime.

Le SS Richard Montgomery et sa cargaison existent réellement si vous cherchez de la documentation supplémentaire\footnote{\url{https://fr.wikipedia.org/wiki/SS_Richard_Montgomery}}.

\subsection{Accroche}

Le gouvernement britannique mobilise les personnages pour une opération de contre-terrorisme sensible et de toute urgence.
Selon les informations des renseignements intérieurs, un groupe terroriste s'apprêterait à commettre un attentat de grande ampleur sur le sol britannique.

\subsection{Péripéties}

D'après les services de renseignement, des écoterroristes auraient rassemblé du matériel explosif et se trouveraient dans la petite ville côtière de Sheerness.
Le ministère de l'Intérieur suspecte que leur cible soit le SS Richard Montgomery, un des \emph{Liberty Ships}  envoyés pour ravitailler l'armée britannique pendant la 2\ieme guerre mondiale.
Le bateau s'est échoué et a coulé en 1944 dans un estuaire de la Tamise à 60 km à l'est de Londres.
Sur les 6400 tonnes d'explosifs à son bord, 5000 ont été récupérées.
Les 1400 tonnes restantes gisent au fond de l'eau, encore actives.

La mission consiste à identifier et intercepter les terroristes avant qu'ils n'agissent. Il faudra repérer les suspects dans la petite ville côtière de Sheerness, anticiper leur mode d'action et les empêcher de nuire. Le \emph{twist} ? Un des personnages fait partie du groupe anarcho-pacifiste\dots

Les \og terroristes \fg sont en réalité un groupe marginal d'écologistes pacifistes qui cherchent à médiatiser la cause du désarmement et de la démilitarisation.
Paradoxalement, faire sauter l'épave enverrait un message choc démontrant l'incapacité des États à gérer un tel armement.
Leur \emph{modus operandi} consiste à envoyer un drone sous-marin déposer une première charge qui amorcera la réaction en chaîne.
Le groupe veut agir pendant la nuit et espère que l'eau absorbera l'essentiel de l'onde de choc, de sorte à ne produire que des dégâts matériels sur la berge.

Les moyens d'investigation classiques des services de renseignement permettent rapidement de remonter la piste: arrivées récentes en ville, achats d'explosifs civils dans les magasins de BTP, locations de véhicules utilitaires, accès aux caméras de surveillance et données téléphoniques.
Les motivations doivent être un peu plus floues et les signaux contradictoires (d'un côté des signes pacifistes, de l'autre une envie visible de déclencher une énorme explosion).

\subsection{Résolution}

L'agent double doit se révéler quand la tension est à son comble, par exemple quand les personnages embarquent pour intercepter le drone sous-marin des terroristes.
L'agent double peut essayer de convaincre les autres du bien fondé de l'opération, surtout si l'explosion ne risque plus de blesser qui que ce soit (par exemple en ayant déclenché une évacuation au préalable).
Ou plus simplement, se débarrasser discrètement de ses petits camarades pour assurer la réussite de l'attentat.
L'objectif ici est que la conclusion soit dramatique et que la réussite ou l'échec de la mission ne tienne qu'à un fil!

\subsection*{Personnage: Sandeep Crawford}
\descriptionperso{Activiste zélé}{Convaincu, bosseur}{Imprudent, excès de confiance}{Désarmement mondial}{Modèle d'enfant d'immigrés bien intégrés, Sandeep est diplômé, cultivé, beau et passionné. Il est le cerveau derrière l'opération et est persuadé que celle-ci ne peut pas échouer en dépit des failles évidentes de son plan.}

\illustration[0.6\textwidth]{ship}
