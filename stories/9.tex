\chapter{La cité de l'horloge}
\keywords{\medfan}{Enquête}{Technologie, légende, rite initiatique}

\section{Scénario}

Ce scénario est une enquête au milieu d'une cité \medfan technologique, à la limite du \emph{steampunk}.
L'histoire peut servir d'amorce à une aventure bien plus longue consistant à trouver de quoi remplacer le poids disparu.

Si vous ne croyez pas à cette histoire de gallium qui détruit l'aluminium, c'est pourtant une véritable propriété du métal! Cherchez des vidéos sur le net pour vous en convaincre.
Bien sûr la réaction imaginée ici est accélérée mais les alchimistes ont peut-être malencontreusement ajouté un autre produit qui a fait catalyseur\dots

\subsection{Accroche}

Les personnages vivent dans une ville mécanique bâtie autour d'une sorte d'horloge titanesque.
De complexes systèmes d'engrenages transforment l'énergie du pendule et la transmettent partout dans la ville, permettant ainsi de nombreuses automatisations de travaux laborieux.
Il suffit de tirer un levier pour que les portes de la ville s'ouvrent, que le blé passe au moulin ou qu'un tapis roulant transporte un lourd fardeau à travers la cité.

Chaque mois, l'immense poids qui maintient le mouvement de balancier est remonté dans une grande cérémonie par les jeunes gens nouvellement d'âge adulte.
Mais lorsque vient le tour des personnages d'accéder aux entrailles de la cité, l'horreur les frappe de plein fouet: le poids a disparu.

\subsection{Péripéties}

Le cœur battant de la cité s'est arrêté.
Le poids qui contrôlerait le mouvement pendulaire de l'horloge s'est volatilisé.
Alors qu'il était suspendu au-dessus du fleuve, il n'en reste plus rien, si ce n'est quelques traces argentées sur la grille qui servait de support.
Comment ces tonnes de métal ont-elles pu quitter la ville?
Telle est la question à laquelle le conseil de la cité les somme de trouver une réponse.

En enquêtant, les personnages réaliseront que depuis quelques semaines, plusieurs notables de la ville se sont plaint du manque de puissance délivrée par le pendule, comme si sa force s'affaiblissait.
D'aucuns accusent le panthéon de punir la ville, d'autres le royaume voisin, notoirement jaloux de la prospérité apportée par l'horloge.
Quelques morceaux de métal pailletés ont d'ailleurs été retrouvés sur des parcelles agricoles le long du fleuve.

C'est l'occasion d'exposer au groupe une galerie de personnages hauts en couleur, chacun essayant de tirer la couverture à soi et de les utiliser pour ses propres intérêts.
Finalement, c'est au détour d'une conversation que le groupe entendra parler des expériences des alchimistes.

\subsection{Résolution}

La vérité est en effet bien plus banale que les complots les plus fous imaginés par les habitants.
Le poids en aluminium était étudiée par un groupe d'alchimistes de l'université.
Afin d'étudier les propriétés de différents métaux, les alchimistes ont expérimenté différents mélanges et alliages.
Lors d'un examen de la surface du poids, une fiole de gallium liquide s'est accidentellement retrouvée en contact avec l'aluminium.

Hélas, le gallium (liquide à température ambiante) a réagi avec le métal, détruisant sa couche protectice et le laissant vulnérable à l'oxydation, qui l'a lentement mais sûrement dévoré de l'intérieur.
Le poids s'est d'abord effrité et a perdu de sa masse, expliquant ainsi les pertes de puissance des derniers jours.
Enfin, compressé contre son support, il s'est effondré sous l'effet de la gravité.
Les morceaux friables ont fini par traverser la grille et disparaître dans le fleuve\dots

\illustration{gears}
