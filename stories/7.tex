\chapter{Un monde enchanté}
\keywords{Contemporain}{Enquête}{Crime, enlèvement, policier}

\section{Scénario}

Ce scénario est une enquête criminelle relativement sombre.
Il peut fonctionner dans un monde contemporain mais est pensé pour un cadre de futur d'anticipation.
L'ambiance est volontairement pessimiste mais n'hésitez pas à ajouter quelques moments de lumière ou d'humour pour éviter de complètement plomber le moral de la table.

\subsection{Accroche}

13 enfants ont disparu en l'espace de trois mois.
Chaque semaine ou presque, la police est alertée d'une nouvelle disparition suspecte mais l'investigation piétine: aucune rançon n'est demandée et aucun corps n'a encore été retrouvé.

\subsection{Péripéties}

Les personnages sont des flics à qui on confie l'enquête en cours de route.
Les infos du dossier sont maigres.
À chaque fois, les parents ont laissé les enfants seuls pour sortir (au cinéma, au restaurant\dots) au moment de leur disparition.
Mais lors du dernier enlèvement, une première piste a été découverte, bien qu'ignorée par les détectives précédemment en charge de l'investigation.
En effet, il y a eu un témoin.
Un \emph{junkie} affirme à qui veut l'entendre avoir vu les coupables emporter deux enfants par une fenêtre d'un immeuble résidentiel BCBG: Peter Pan et sa complice de toujours, la Fée Clochette.

Lors de l'enquête, les personnages remonteront un étrange faisceau d'indices: couple déguisé aperçu à plusieurs reprises dans les parages les jours précédants les enlèvements, costumes achetés trois mois plus tôt dans un magasin spécialisé en accessoires de théâtres, traces d'une poudre volatile pailletée non identifiée sur les lieux de la disparition, etc.
En se plongeant dans les anciens relevés, un autre point commun entre tous les parents d'enfants disparus peut émerger: leur sortie mondaine était à chaque fois documentée sur les réseaux sociaux.

\subsection{Résolution}

En remontant la piste des clients du magasin (achats réglés en carte) et en croisant avec les contacts des réseaux sociaux des victimes, les personnages pourront identifier un jeune couple (une chimiste et un pharmacien).
L'analyse de la poudre confirmera qu'il s'agit d'un mélange d'euphorisants et de somnifères, probablement pour faciliter l'enlèvement des enfants sans résistance.
Le couple a récemment souffert du décès de leur premier enfant, né prématuré.
Se réfugiant dans les antidépressants, le couple s'est créé un monde parallèle dans lequel leur raison d'être est de sauver le plus d'enfants possibles, \og abandonnés \fg par leurs parents.
Heureusement, les jeunes victimes sont en bonne santé -- bien que sous l'effet d'euphorisants saupoudrés dans les plats -- et sont simplement logés dans une grande villa de campagne héritée par le couple où le mari et la femme se relaient pour prendre soin d'eux.
