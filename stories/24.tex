\chapter{Traitement au Radithor}
\keywords{\retro}{\action}{\index[theme]{Exploration}Exploration, \index[theme]{Guerre}guerre}

\section{Scénario}

Figurez-vous que le Radithor est un traitement tout ce qu'il y a plus de réel\footnote{\url{https://fr.wikipedia.org/wiki/Radithor}}. Il n'a bien sûr aucune propriété thérapeutique, bien au contraire, le radium étant radioactif.
Ce scénario vous propose en revanche d'imaginer une \emph{origin story} pour un super-héros ou une super-héroïne enrichie en lourde.

\subsection{Accroche}

Les personnages constituent le gros d'une unité des forces spéciales envoyée mater les révoltes dans une des colonies françaises de la première moitié du XX\ieme siècle, par exemple en Afrique centrale ou en Indochine.
Après quelques jours d'expédition loin de leur avant-poste à la recherche d'insurgés qu'ils ne trouveront jamais, le groupe commence à souffrir de nausées et de maux de têtes.
Seul le capitaine de la Bouillie, leur supérieur, semble épargné par cette étrange maladie.

\subsection{Péripéties}

Un matin au milieu de l'expédition, les personnages se réveillent, la tête encore embrumée par une énième migraine quand ils réalisent que le capitaine a disparu du campement.
Lorsqu'ils le retrouvent, il flotte dans les airs au milieu d'arbres calcinés à une centaine de mètres de sa tente.
L'incident dure quelques minutes avant que leur supérieur ne reprenne connaissance et tombe au sol.
Incapable de répondre à leurs questions, il se contente de vomir ses tripes avant de lentement récupérer.
Un bref interrogatoire montre que de la Bouillie n'a aucune idée de ce qui lui est arrivé. Il ne se souvient même pas être allé se coucher la veille, comme pris d'une amnésie localisée.

Un peu de réflexion amènera les personnages à réaliser qu'ils prennent tous régulièrement depuis deux semaines du Radithor, une décoction miracle au radium que l'état-major expérimente pour booster les soldats. La différence entre eux et le capitaine? Ce dernier était un des premiers à avoir testé le traitement: il en prend non pas depuis deux semaines, mais depuis deux mois.

L'objectif de l'escouade sera donc de faire demi-tour et de revenir sur leurs pas en direction de l'avant-poste des forces françaises.
Malheureusement, le retour devra se faire en se traînant le capitaine, qui est régulièrement pris de \og crises \fg impressionnantes (détaillées dans le tableau de la page \pageref{table:radithor}).
Arrêter le traitement induit une courte période de sevrage (24 heures) mais aura comme effets bénéfiques de stopper les nausées et les migraines des personnages.
À l'inverse, prendre du Radithor en grandes quantités permet de les déclencher mais avec un contrôle limité, voire inexistant en fonction de la dose.
Le territoire étant hostile, ces démonstrations de sons et lumières risquent d'être plus problématiques qu'autre chose\dots

\subsection{Résolution}

En fonction de la capacité des joueuses à se sortir du pétrin, les personnages peuvent aussi bien se retrouver prisonniers des \og rebelles \fg que réussir à retrouver le camp français.

En partant du principe que le groupe réussit à rejoindre l'armée française, trois fins sont possibles selon vos envies:
\begin{itemize}
	\item \emph{L'expérience est concluante.} Les scientifiques examinent de la Bouillie et sont satisfaits. Le traitement est poursuivi avec des dosages ajustés et les personnages peuvent faire partie de ce nouveau programme expérimental. Cela peut-être une belle accroche pour les enrôler dans les Brigades Chimériques.
	\item \emph{Vous en savez trop.} Les personnages ont refusé de jouer le jeu ou représentent une menace pour l'existence du programme \og Radithor \fg. L'état-major les jette en prison ou achète leur silence. Paf, fin, rideau, à l'revoyure.
	\item \emph{Ça craint ici, on se casse.} L'état-major n'est pas très subtil ou se fiche bien des rumeurs. Le capitaine est gardé en observation mais les personnages sont libres de partir. Raconteront-ils leurs histoires à d'autres? Ou emporteront-ils le secret à la tombe?
\end{itemize}

\begin{table}
	\caption{Table aléatoire des crises au Radithor (durée: 10 minutes)}
	\label{table:radithor}
	\colortablerows
	\begin{tabularx}{0.9\textwidth}{cX}
	d6 & Événement\\
	1. & Une volée d'énergie pure incendie tout ce qui se trouve dans un rayon de 10 mètres toutes les 30 secondes.\\
	2. & Le personnage est propulsé dans un autre corps et le contrôle comme s'il s'agissait du sien.\\
	3. & Le personnage flotte dans les airs et peut déplacer des objets jusqu'à 30 kg par télékinésie.\\
	4. & De l'acide suinte par les pores de la peau du personnage, liquéfiant tout ce qu'il touche.\\
	5. & Le personnage se déplace à une vitesse stupéfiante, les obstacles éclatent à l'impact.\\
	6. & Le corps du personnage se déforme et grossit jusqu'à atteindre le double de sa taille normale.\\
	\end{tabularx}
\end{table}

\vfill
\illustration{ribs}
\vfill
