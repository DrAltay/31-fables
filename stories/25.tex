\chapter{Dé-li-cieux!}
\keywords{\contemporain}{\intrigue}{\index[theme]{Horreur}Horreur, \index[theme]{Mystère}mystère, \index[theme]{Créature}créature}

\section{Scénario}

Cette histoire trouve son inspiration dans des séries fantastiques et notamment La Quatrième Dimension (\emph{Twilight Zone} en VO).
Petit à petit, l'étrange et l'irréel s'immisce dans une situation pourtant bien banale\dots

\subsection{Accroche}

Affamés et perdus après s'être trompés de sortie, les personnages font une pause au milieu d'un long \emph{road trip} dans un fast-food isolé dans l'Amérique rurale. Des phénomènes étranges ne tardent pas à les surprendre.

\subsection{Péripéties}

Le restaurant est un mix entre un \emph{diner} et un fast-food: déco 70s, de nombreuses tables avec banquettes, une quadruple comptoir pour passer commande et du personnel de service en uniforme à rayures rouges et blanches.
Le menu accroché au mur est daté mais les plats sont des classiques intemporels (burger, hot-dog, ailes de poulet, nuggets, salade et grande variété de sodas aux distributeurs).
Au moment de passer commande, la serveuse semble toutefois à peine les écouter, comme apathique.
Elle enregistre néanmoins machinalement leurs plats sur sa machine puis, soudainement prise d'enthousiasme, elle lève un regard pétillant en leur direction et s'exclame en pointant un doigt vers une affichette promotionnelle:

\blockquote{Et avec ça, vous prendrez des gaufres? Seulement 1,99\$! Elles sont dé-li-cieuses, miam!}

Elle insistera plusieurs fois (dé-li-cieuses, miam!) avant de les encaisser et de les faire s'installer le temps qu'on leur apporte leur commande.

Petit à petit, le groupe va réaliser que rien ici ne tourne rond en dépit de la vingtaine de personnes qui mangent dans le restaurant.
À vrai dire, tout le monde agit bizarrement et les fixe avec attention, ne détournant les yeux que lorsque le regard des personnages croise le leur.
Si, à première vue, les conservations qui baignent le \emph{diner} sont en \og anglais \fg, en écoutant avec un peu d'attention il s'avère bien vie que les clients s'expriment avec des phrases qui n'ont aucun sens, comme si on était sur le fond d'un tournage.

Car cette petite sortie d'autoroute est un véritable lieu de perdition.
Le restaurant est un endroit où des entités bizarroïdes apprennent à se faire passer pour des humains, une sorte de lieu d'entraînement.
Aliens, PNJ d'une simulation, créatures cauchemardesques, les possibilités sont multiples.
Les personnages, tombés ici par hasard, sont observés et mêmes imités par leurs étranges hôtes.
Ces derniers vont tout faire pour les garder le plus longtemps parmi eux afin d'apprendre tout ce qu'ils peuvent, les mettant dans des situations kafkaïennes pour étudier leurs réactions et en apprendre plus sur les comportements humains\dots

\subsection{Résolution}

Comme tout bon scénario d'horreur, s'échapper doit être difficile: le personnel de service leur barre la route s'ils veulent partir, les téléphones ne captent pas et le désert s'étire dehors à perte de vue.
Les hôtes n'ont pas a priori d'intentions violentes mais ils ne seront pas (trop) durement malmener par les personnages.
Une piste pour s'enfuir? La clientèle déteste les animaux car ils sont capables de les sentir pour ce qu'ils sont réellement: des imposteurs\dots

\vfill
\illustration[0.7\textwidth]{waffle}
\vfill
