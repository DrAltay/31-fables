\chapter{Dé-li-cieux!}
\keywords{Contemporain}{Intrigue}{Horreur, mystère}

\section{Scénario}

Commentaire

\subsection{Accroche}

Affamés et perdus après s'être trompés de sortie, les personnages font une pause au milieu d'un long *road trip* dans un fast-food isolé dans l'Amérique rurale. Des phénomènes étranges ne tardent pas à les surprendre.

\subsection{Péripéties}

Le menu accroché au mur est daté mais les plats sont classiques. Au moment de passer commande, la serveuse semble à peine les écouter, comme apathique mais enregistre machinalement leurs plats. Puis soudainement prise d'enthousiasme, elle lève un regard pétillant en leur direction et s'exclame:

\blockquote{Et avec ça, vous prendrez des gauffres? Elles sont dé-li-cieuses, miam!}

Elle insistera plusieurs fois (dé-li-cieuses, miam!) avant de les encaisser.

Petit à petit, le groupe va réaliser que rien ici ne tourne rond en dépit de la vingtaine de personnes qui mangent là. Tout le monde agit bizarremment, parle \og anglais \fg mais avec des phrases qui n'ont aucun sens, comme si on était sur le fond d'un tournage. Qui plus est, on les observe…

L'endroit est un lieu où des entités bizarroïdes apprennent à se faire passer pour des humains. Aliens, PNJ d'une simulation, créatures cauchemardesques, les possibilités sont multiples. Les personnages, tombés ici par hasard, sont observés et mêmes imités par leurs hôtes étranges, qui voudront tout faire pour les garder le plus longtemps afin d'apprendre tout ce qu'ils peuvent.

\subsection{Résolution}

\illustration{waffle}
