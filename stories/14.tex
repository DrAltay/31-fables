\chapter{Le bouchon du Darién}
\keywords{Contemporain}{Aventure}{Exploration, voyage}

\section{Scénario}

Le bouchon du Darién est une région marécageuse entre la Colombie et le Panama.
La zone est intraversable et sépare l'Amérique du Nord de l'Amérique du Sud.
Plusieurs tentatives ont eu lieu dans l'histoire contemporaine d'y faire passer une route avant d'achever la voie panaméricaine, sans succès.
Ce scénario met en scène l'une de ces tentatives.
L'époque idéale pour cette aventure se situe probablement dans les années 50 mais tout le 20\ieme siècle peut fonctionner.

\subsection{Accroche}

Grâce à un sponsor industriel (géant de l'automobile ou du pétrole), les personnages héritent d'une mission historique: traverser la région marécageuse du Darién en véhicule, démontrant par cet exploit la faisabilité de la route panaméricaine reliant l'Amérique Nord à l'Amérique du Sud.

\subsection{Péripéties}

Explorée dans les années 1870, la région du Darién est une jungle marécageuse peu praticable.
Seuls des canoë, le ferry ou l'avion permettent de la contourner.
Partant du Panama, les personnages auront quelques véhicules (y compris un bulldozer), des provisions et une bonne dose d'encouragements.

Mais la traversée n'aura rien d'une partie de plaisir.
Entre le terrain marécageux, la faune inhospitalière, les maladies qui rôdent et la population locale des Embera qui voit le projet de route d'un très mauvais œil, les obstacles sont légions.
Ce n'est pas non plus les traces d'une lointaine tentative de colonisation écossaise, avortée à cause de la malaria puis du conflit avec les voisins espagnols, qui rassureront les personnages.

Et que dire, si l'expédition se déroule après 1964, de la présence de rebelles colombiens qui profitent de la difficulté d'accès de la région pour y cacher certaines de leurs opérations\dots

\subsection{Résolution}

À mesure de l'avancée des personnages, l'expédition doit leur sembler de moins en moins possible.
Qui plus est, les différentes péripéties peuvent parfois frôler le fantastique ou le surnaturel.
N'oubliez pas que dans la jungle marécageuse, la faune et la flore exotiques peuvent sembler irréelles aux personnages habitués à la civilisation.
In fine, que le groupe arrive ou pas à bon port n'a que peu d'importance par rapport au fait de simplement s'en sortir vivant.
