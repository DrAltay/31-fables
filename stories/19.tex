\chapter{Un effet lunaire}
\keywords{Super-héroïque}{Action}{Trahison}

\section{Scénario}

Commentaire

\subsection{Accroche}

Les personnages sont des supers et possèdent d'exceptionnels pouvoirs (télékinésie, yeux lasers, etc.).
Leur mentor de toujours, l'adorable et génial Dr. T. R. Aitor, les appelle pour une mission de la plus haute importance: sauver le monde!
Un astéroide menace de frapper la Terre et seule la dernière invention du docteur, un engin spatial muni de puissants harpons, pourait l'attraper et le dévier.
Seul problème: le super-propergol -- le carburant spécial de la fusée -- est tombé dans les mains de la maléfique générale Mellon.

\subsection{Péripéties}

Les personnages prennent d'assaut la base de la générale Mellon, défendue par une équipe de supers à la solde de sa corporation criminelle.
Le super-propergol est très instable mais très précieux: il n'en existe qu'un seul baril!
Il faudra redoubler de ruse et de courage pour s'infiltrer et repartir avec (et bien sûr, éviter les victimes inutiles).
Lorsque le groupe tentera de partir, Mellon leur donnera un dernier avertissement : \og Vous nous condamnez tous\dots \fg

Les personnages retrouvent le Dr. Aitor sur le pas de tir de l'agence spatiale.
La fusée décolle, puis après quelques minutes, harponne l'astéroïde et le détourne.
Mais\dots horreur! traîtrise!
Une scientifique arrive en courant à toute vitesse.
\blockquote{Il y a eu une erreur! L'astéroïde n'était pas du tout sur une trajectoire qui touchait la Terre. Mais en le déviant, il prend maintenant un effet fronde autour de la Lune, il va nous frapper de plein fouet!}
Froidement, le Dr. Aitor l'abat! Car tel était son plan!
Voici maintenant que les personnages doivent reprendre le contrôle sur la fusée et inverser la situation.
Mais le Dr. T. R. Aitor a plus d'un tour dans son sac.
Il a formé les personnages et connaît toutes leurs faiblesses.
Le vaincre sera une autre paire de manche\dots

\subsection{Résolution}

\illustration{moon}
