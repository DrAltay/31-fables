\chapter{Un effet lunaire}
\keywords{\superheros}{\action}{\index[theme]{Trahison}Trahison, \index[theme]{Combat}combat}

\section{Scénario}

Cette aventure est une pure histoire de super-héros dans le style des comics et de films de la \emph{pop culture}.
Le retournement de situation qui intervient à la moitié permet de découper le scénario en deux actes qui peuvent être séparés sur deux séances ou joués sur une longue session avec une pause.
Dans ce cas, il est possible de commencer \emph{in media res} pour accélérer le début de la partie et mettre directement la table dans le feu de l'action, comme savent bien le faire les bons \emph{blockbusters}.

\subsection{Accroche}

Les personnages possèdent des super-pouvoirs (yeux lasers, télékinésie, etc.) et forment un groupe au service de l'humanité.
Leur mentor de toujours, l'adorable et génial Dr. Travis Ronald Aitor, les appelle pour une mission de la plus haute importance: sauver le monde!

\subsection{Péripéties}

La menace imminente consiste en un astéroïde massif qui menace de frapper la Terre dans les prochains jours.
Seule la dernière invention du docteur, un engin spatial muni de puissants harpons, pourrait l'attraper et le dévier.
Le problème qui survient est le suivant: la fusée a besoin d'un carburant spécial -- le super-propergol -- pour fonctionner.
Malheureusement, les seules réserves existantes sont tombées dans les mains de la maléfique générale Mellon, une autocrate à la tête d'une nation voyou non-reconnue par les autres États.

Il faut donc que les personnages prennent d'assaut la base de la générale Mellon afin de récupérer l'indispensable carburant.
Le quartier-général est toutefois défendu par une autre équipe de supers-mercenaires à la solde de son organisation criminelle.
Le super-propergol est très instable mais très précieux: il n'en existe qu'un seul baril!
Le groupe devra redoubler de ruse et de courage pour s'infiltrer et repartir avec (en évitant bien sûr les victimes inutiles).
Lorsque les joueuses seront sur le point de partir, Mellon leur lancera un dernier avertissement: \og Vous nous condamnez tous\dots \fg

Les personnages retrouvent alors quelques heures plus tard le Dr. T.R. Aitor sur le pas de tir de l'agence spatiale.
La fusée décolle, puis, après un long suspense, parvient à harponner l'astéroïde et le dévier de sa trajectoire.
Houra?
Hé non! horreur! traîtrise!
Une experte en astronavigation arrive en courant à toute vitesse.
\blockquote{Il y a eu une erreur! J'ai refait les calculs, l'astéroïde n'était pas du tout sur une trajectoire qui touchait la Terre. Mais en le déviant, il prend maintenant un effet fronde autour de la Lune et il va nous frapper de plein fouet!}
À peine a-t-elle fini son exposé que le Dr. Aitor l'abat froidement d'un coup de pistolet-laser.
Car tel était son plan depuis le début!

Voici maintenant que les personnages doivent reprendre la main sur la fusée et inverser la situation.
Cela risque de ne pas être si simple, car le docteur a plus d'un tour dans son sac: il a formé les personnages et connaît toutes leurs faiblesses.
Le vaincre sera une autre paire de manche\dots

\subsection{Résolution}

Alors, vos joueuses ont triomphé du docteur Aitor? Même alors que vous avez exploité au maximum leurs faiblesses et utilisé vos force pour les retourner contre leurs personnages?
Félicitations!

Reste à décider de ce qui s'est réellement passer. Le docteur Aitor était-il réellement un traître, un agent double préparant dans l'ombre son plan machiavélique depuis des années? Ou bien était-il sujet au contrôle mental d'un super-vilain bien plus maléfique encore? Ou peut-être a-t-il été remplacé par un clone à l'insu des personnages?

À vous de choisir la conclusion la plus satisfaisante pour votre partie!

\vfill
\illustration[0.4\textwidth]{moon}
\vfill
