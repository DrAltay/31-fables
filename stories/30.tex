\chapter{Trop beau pour être vrai}
\keywords{Western}{Intrigue}{Enquête, crime}

\section{Scénario}

Commentaire

\subsection{Accroche}

Alors que les personnages font la fermeture d'un saloon, un type mal en point débarque à cheval, se laisse glisser à terre, fait quelques pas dans la salle et tombe avant de se laisser mourir sur le plancher. Personne ne le connaît ici et il semble venir d'une ville voisine.

\subsection{Péripéties}

Il porte une sacoche en cuir en bandouillère contenant des dizaines de milliers de dollars en billets de banque. Les personnages peuvent les remettre au shérif, se les partager (à défaut, comme le tenancier du saloon en gardera une partie pour le donner à quelques habitants proches des personnages, voire les personnages eux-mêmes, et donnera le reste au shérif). Toutefois, dès le lendemain, de mystérieux étrangers vont débarquer en ville. Ils cherchent visiblement l'individu de la veille, sans pour autant faire de mention du pactole.

Le hic, c'est que les billets sont contrefaits. Les étrangers sont des agents fédéraux du Service Secret, nouvellement créé pour enquêter sur la fausse monnaire qui circule dans l'Ouest. Leur investigation est discrète : ils ont blessé le faussaire et un complice la veille mais il est parvenu à leur échapper à cheval. Ils soupçonnent que d'autres de ses complices se cachent en ville. Les personnages vont-ils coopérer ? Tenter de cacher le magot ? Ou peut-être eux-mêmes ou leurs proches font partie du groupe de faussaires…

\subsection{Résolution}

\illustration{cash}
