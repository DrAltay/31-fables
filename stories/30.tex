\chapter{Trop beau pour être vrai}
\keywords{\western}{\intrigue}{\index[theme]{Crime}Crime, \index[theme]{Évasion}évasion, \index[theme]{Diplomatie}diplomatie}

\section{Scénario}

Un bon vieux scénario de western avec un soupçon de qui pro quo.
Un magot, des agents de la loi, des gangsters et quelques personnes \og innocentes \fg mêlées de force à une affaire qui ne les concerne pas\dots
Avec un peu d'imagination, il n'est pas invraisemblable de transposer cette histoire à l'époque contemporaine ou à d'autres cadres plus fantaisistes.

\subsection{Accroche}

Alors que les personnages font la fermeture d'un saloon, un type mal en point débarque à cheval, se laisse glisser à terre, fait quelques pas dans la salle et tombe avant de se laisser mourir sur le plancher.
Personne ne le connaît ici et il semble venir d'une ville voisine.

\subsection{Péripéties}

Le gus qui vient de clamser au beau milieu du saloon porte une sacoche en cuir en bandoulière.
Pas besoin d'être un fin limier pour s'apercevoir que celle-ci contient des dizaines de milliers de dollars en billets de banque\footnote{Une petite fortune pour l'époque, plus d'un million de dollars de 2020.}.
Les personnages peuvent décider de les remettre au shérif ou de se les partager.
À défaut, c'est tenancier du bar qui prendra les choses en mains.
Il en gardera une partie pour le donner à quelques habitants, dont des proches des personnages voire les personnages eux-mêmes.
Il comptera sur leur silence et donnera le reste (environ 1/4 du pactole) au shérif.

Mais dès le lendemain matin, de mystérieux étrangers vont débarquer en ville.
Ils posent de nombreuses questions aux habitants et ils cherchent visiblement l'individu de la veille, sans pour autant faire de mention du pactole.

Le hic, c'est que les billets sont contrefaits et que le macchabée de la veille est un faussaire.
Les étrangers sont des agents fédéraux du Service Secret, nouvellement créé pour enquêter sur la fausse monnaie qui circule dans l'Ouest américain.
Leur investigation est discrète: ils ont blessé le faussaire et un de ses complices dans une fusillade la veille mais il est parvenu à leur échapper à cheval.
Ils soupçonnent maintenant que d'autres de ses complices se cachent en ville.
Les personnages vont-ils coopérer avec leur enquête?
Ou bien appâtés par le perspective d'un gain facile, tenter de cacher le magot?
Ou peut-être qu'eux-mêmes ou certains de leurs proches font partie de la bande qui a produit ces contrefaçons\dots

\subsection{Résolution}

À un moment ou à un autre, vos joueuses vont sûrement confronter les agents spéciaux.
Ceux-ci seront bien forcés de dévoiler leurs identités et la raison de leur présence.
Quid alors des personnages? Et bien leur réaction dépendra de leurs intérêts du moment et de leurs vertus morales.
Éliminer les agents pour stopper l'enquête est une façon de s'accaparer le magot mais n'oublions pas que le faussaire a des complices qui eux aussi aimeraient bien récupérer leur \og argent \fg.
Dans tous les cas, fusillades, coups de couteaux dans le dos et négociations à la dynamite sont à prévoir!

\subsection{Lieu: Grizzly Gulch, Arizona}

\begin{tcolorbox}[colback=black!1!white]
Petite ville d'un millier d'âmes coincée entre les territoires Navajos et Apache, Grizzly Gulch est la dernière étape des caravanes avant Flagstaff.
Ici, tout le monde se connaît. La loi est représentée par un shérif flanqué de deux adjoints mais les U.S. Marshals ne sont jamais bien loin.
Un hôtel de ville, deux saloons, deux barbiers, un \emph{drugstore}, des épiceries qui revendent les produits agricoles des innombrables ranchs du coin, un ferronnier, une sage-femme faisant office de médecin de campagne et bien sûr deux armureries pour les besoins en quincaillerie.
\end{tcolorbox}

\vfill
\illustration[0.6\textwidth]{cash}
\vfill
