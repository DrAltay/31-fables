\chapter{La veste d'Audie}
\keywords{\cyberpunk}{\action}{\index[theme]{Effraction}Effraction, \index[theme]{Famille}famille}

\section{Scénario}

Sous ses atours poussiéreux d'effraction dans un musée, ce scénario cache une histoire de \emph{run} assez classique: s'approprier un bien précieux pour le compte d'une corporation à la moralité douteuse.
N'hésitez pas à jouer à fond la carte du commanditaire peu scrupuleux, tout à fait capable de supprimer les personnages une fois le méfait accompli histoire de ne laisser aucune trace.

\subsection{Accroche}

Les personnages sont embauchés par un intermédiaire pour le compte d'une personnalité souhaitant rester anonyme.
La mission est cependant plutôt bien rémunérée et consiste à rentrer en possession de la veste militaire d'Audie Murphy, un soldat de la 2\ieme guerre mondiale connu pour sa carrière d'acteur et ses nombreuses décorations militaires.

\subsection{Péripéties}

La fameuse veste d'Audie est celle que le héros de guerre a supposément portée lors de sa célèbre contre-attaque sur Colmar.
Celle-ci a fait l'objet d'une donation de la part des héritiers d'Audie il y a quelques années et est désormais exposée dans les galeries du musée militaire de Fort Belvoir, près de Washington.

Le \emph{run} est donc un cambriolage des plus classiques: déjouer la sécurité du musée, éviter les patrouilles militaires et saboter l'alarme afin de mettre la main sur la veste.
Malheureusement, après examen, le couperet finit par tomber: la veste est fausse.
Les personnages sont donc renvoyés chercher l'originale.

En grattant un peu sur l'histoire de la famille Murphy, le groupe peut découvrir que Jill et Russel, deux petits-enfants d'Audie, se disputent l'héritage de leur grand-père. Ils possèdent deux entreprises rivales d'armement (Murphy Weaponry et Murphy Gunsmithing). 
Chacun tire la couverture à soi et se targue d'être l'héritier légitime du héros de guerre, n'hésitant pas à utiliser son modèle holographique pour faire la publicité de leurs produits et d'exploiter son image à des fins mercantiles.

Il n'est pas bien compliqué de conclure que c'est Russel qui a embauché le groupe pour voler la veste.
Sa sœur, qui la possédait jusqu'à récemment, l'a en effet offerte au musée en grandes pompes deux ans plus tôt, en échange d'un juteux contrat d'équipement de la garde nationale.
Ce qui a échappé à la vigilance des deux héritiers, c'est Gina Costello, petite-fille d'Audie par adoption dont ils ignorent même l'existence, s'empare peu à peu de l'héritage de son grand-père.
En remontant la piste de la fausse veste, remplacée par Gina lorsqu'elle se faisait passer pour une assistante maternelle chez Jill Murphy, les personnages finiront par rencontrer mademoiselle Costello.
Celle-ci s'est lassée de voir ses cousins se crêper le chignon et manquer de respect à celui qu'elle voit comme un soldat au grand cœur, défenseur des victimes de PTSD et du soutien psychologique aux vétérans.
Elle rassemble lentement et sûrement les médailles, journaux et effets personnels ayant appartenu à Audie dans son garage.

Toutefois, Gina n'a hélas pas grand chose à offrir aux personnages à part sa gratitude.
Elle connaît néanmoins plutôt bien les villas respectives de Jill et Russel dans lesquelles elle s'est faite passer pour une employée de service afin d'accomplir son œuvre.
Si le groupe est sensible à son histoire, elle ne sera pas avare de conseils sur les possibilités d'infiltration dans la demeure familiale des Murphy et plus particulièrement dans les collections privées de ses cousins.
Avec son aide, les \emph{runners} pourraient bien récupérer les dernières médailles encore en possession des héritiers pourris gâtés -- et se servir au passage dans les œuvres d'art et armes de collection inestimables que leurs coffres recèlent.

\subsection{Résolution}

Si vos joueuses incarnent des personnages sans foi ni loi et sans la moindre once d'empathie, il faut bien admettre que le groupe ne devrait faire qu'une bouchée de Gina et aisément lui reprendre la veste.
Dans ce cas, vous pouvez leur faire comprendre que le crime ne paie pas en envoyant à leurs trousses des mercenaires payés par Russel pour les éliminer, faisant disparaître toute trace de son méfait.

En revanche, dans le cas contraire, c'est l'arroseur arrosé: cambriolage de haut vol dans le manoir de Murphy, pillage en règle des coffres et des pièces de collection revendues à prix d'or pour un montant au-delà de leurs espérances.
Même dans un monde cyberpunk noir et cynique, avoir un cœur, parfois, ça paie.

\illustration[0.25\textwidth]{soldier}
