\chapter{Le Dragon et le Chevalier}
\keywords{Révolution industrielle/\emph{Steampunk}}{Aventure}{Exploration, créature, légende}

\section{Scénario}

Cette aventure met en scène un riche armateur mégalomane souhaitant triompher d'une créature légendaire.
La similarité avec King Kong est parfaitement volontaire!
L'ambiance doit être contrastée entre d'une part la révolution industrielle qui modernise la société à grands pas et les idéaux romantiques du commanditaire des personnages, s'imaginant tour à tour comme un chevalier romanesque et explorateur courageux.
Une transposition à d'autres univers (\medfan par exemple) n'est pas bien difficile à envisager.

\subsection{Accroche}

Les dîners mondains sont émoustillés par la découverte du journal de bord secret de James Cook, le célèbre explorateur, mort depuis plus d'un siècle.
Dans son exploration de la Polynésie, Cook relate une histoire racontée du point de vue de ses sous-fifres.
Le récit parle d'une rencontre avec un gigantesque lézard, entraperçu dans une faille sur flanc d'un volcan, que l'esprit européen de Cook associe aux dragons des contes et légendes.

Un riche et vaniteux armateur rassemble les personnages et les emploie pour l'escorter dans une expédition maritime.

\subsection{Péripéties}

L'armateur s'imagine déjà devenir Chevalier de la Couronne en rapportant la tête du dragon, voire la créature encore vivante!
Le voyage, troublé par les tempêtes, n'est pas de tout repos.
L'indigène, là pour traduire et communiquer avec les tribus locales, disparaît à peine le bateau accosté.
Les autochtones, quand on leur décrit la bête supposée, évoquent une mythologie ancienne et tout particulièrement le dieu-lézard Pili.

Poussé par la quête de gloire, l'armateur emmènera les personnages à travers la jungle en direction du volcan, à la recherche de \og son dragon \fg.
Bravant les obstacles, le groupe entrera dans le volcan par une faille souterraine, risquant l'étouffement dans la chaleur et les gaz.
La caverne de cristaux -- des diamants sont emportés des profondeurs de la Terre par le magma -- et d'offrandes précieuses emplit de joie les yeux des membres de l'expédition.

Mais le dieu Pili a une autre idée en tête.
Ce grand dragon millénaire n'a aucune intention de mourir et encore moins de servir de bête de foire.
À vrai dire, la simple présence de l'expédition trouble sa solitude.
À l'orgueil il répondra par la cupidité: que les personnages prennent ce que leur cœur désire dans son trésor\dots en échange de leur aide pour se débarasser de l'armateur.
Une fois ceci fait, Pili compte sur leur silence afin de tuer dans l'œuf toutes les rumeurs concernant son existence et acheter ainsi sa tranquilité\dots

\subsection{Résolution}

Les personnages ont peu d'options.
La confrontation directe avec Pili semble perdue d'avance à moins de se lancer dans une bataille dantesque.
L'armateur pourrait être convaincu, après tout, revenir avec un trésor n'est pas beaucoup moins glorieux mais Pili semble bien décidé à avoir sa tête!
