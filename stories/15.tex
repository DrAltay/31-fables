\chapter{La chambre au cobra}
\keywords{\contemporain}{\action}{\index[theme]{Effraction}Effraction, \index[theme]{Malédiction}malédiction, \index[theme]{Divinité}divinité}

\section{Scénario}

Ce scénario se divise en deux parties: un cambriolage et la lutte contre une malédiction.
La première partie doit être plutôt facile pour le groupe (si cela met la puce à l'oreille des joueuses, c'est encore mieux !).
En revanche, la seconde partie doit prendre une tournure presque horrifique.
Voyez par exemple le film La Momie pour une inspiration sur ce que peut être une telle malédiction.

\subsection{Accroche}

Dans le sud de l'Inde, le temple de Sree Padmanabhaswamy situé en plein milieu de la ville de Trivandrum recèle des trésors inestimables.
Connues depuis des années, cinq chambres souterraines ont été ouvertes en 2011 pour rembourser les dettes du temple provoquées par une gestion calamiteuse.
15 milliards d'euros en pierres précieuses, statues et autres bijoux en or sont découverts.
Toutefois, 3 chambres sont restées fermées.
La raison: elles sont scellée par une porte portant un motif de cobra, symbole de péril et de malédiction dans les légendes locales.

\subsection{Péripéties}

Les personnages, pour eux-mêmes ou pour un tiers, ont décidé de mettre la main sur ces richesses incommensurables.
Pour éviter toute intrusion, 200 gardes veillent en permanence et le temple est surveillé par des caméras et des alarmes.
Cela n'empêche toutefois pas les personnages d'y entrer et de cambrioler la chambre B, brisant le sceau du cobra et amassant un pactole conséquent.

Mais la lune de miel ne dure qu'un temps : petit à petit, commanditaires et collègues de crime meurent dans des circonstances violentes.
Il va falloir trouver une offrande conséquente pour tenter d'apaiser la colère de Vishnou et échapper à ses cobras vengeurs qui peuvent apparaître et disparaître à leur guise dans le moindre recoin sombre\dots

\subsection{Résolution}

Bien sûr, il est préférable que Vishnou et ses cobras s'attaquent d'abord aux personnages non-joueurs mais n'hésitez pas à y aller franco!
Les cobras se téléportent à volonté, les personnages sont frappés de malchance, leurs richesses se retournent contre eux (la nouvelle voiture provoque un carambolage monstrueux, le fisc débarque, leur domicile prend feu\dots).
Quant au choix d'une offrande, aux joueuses de se montrer inventives mais n'hésitez pas à récompenser les bonnes initiatives.
Simplement \og rendre \fg le trésor est une idée mais ne doit pas être suffisant (ce serait trop simple).

\subsection*{Personnage: les cobras de Vishnou}
\descriptionperso{Serviteur fidèle}{Se téléporter, protéger les trésors}{Les aigles}{Défendre leur trésor}{Les nāgas sont des serpents mythiques dotés de pouvoirs qui servent comme gardiens des trésors. Ils forment une des incarnations de Vishnou, l'autre étant leur nemesis, l'aigle géant Garuda.}

\vfill
\illustration{cobra}
\vfill
