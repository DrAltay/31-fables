\chapter{Hiver nucléaire}
\keywords{\contemporain}{\action}{\index[theme]{Terrorisme}Terrorisme, \index[theme]{Voyage}voyage}

\section{Scénario}

Ce scénario met en scène la traque d'un groupe criminel à travers la toundra sibérienne en plein hiver.

\subsection{Accroche}

Les autorités envoient les personnages au fin fond de la Sibérie, en plein hiver.
Là, sur la côte arctique, leur mission est de regagner un phare abandonné depuis des décennies.
Pourquoi? Car comme beaucoup d'autres à l'époque, ce phare était conçu pour être autonome et fonctionnait donc avec un réacteur thermoélectrique à isotope.
Or, deux autres phares de la région ont récemment été vandalisés et le noyau de polonium 210 du générateur a disparu à chaque fois\dots

\subsection{Péripéties}

Il est déjà trop tard lorsque les personnages arrivent (par parachutage ou simplement par bateau).
Les portes ont été forcées et le cœur a disparu.
Le bateau des coupables semble s'être échoué dans la baie.
Toutefois, l'épopée criminelle ne s'est pas arrêtée là puisque les indices indiquent que les vandales ont fui par les terres.
Aussi fou que cela puisse paraître, leur plan semble être de marcher dans le blizzard les 50 km qui les séparent du petit village de Rogatchevo où s'échapper en voiture ou par les airs sera possible.

Une longue traque commence alors pour retrouver le plutonium et empêcher que le noyau tombe entre de mauvaises mains.
Une course contre la montre hivernale, luttant contre le froid, le vent et une bande criminelle qui espère bien tirer profit de cette cargaison radioactive.

La table de la page \pageref{table:hiver} donne quelques idées d'événements pouvant épicer le voyage de vos personnages.

\begin{table}
	\caption{Événements aléatoires du périple sibérien}
	\label{table:hiver}
	\colortablerows
	\begin{tabularx}{0.9\textwidth}{cX}
	d6 & Événement\\
	1  & Les personnages tombent sur un convoi funéraire ostiak en route vers la côte pour répandre les cendres de leur ancien chef. Les ostiaks se méfient du gouvernement (qui voit certaines coutumes traditionnelles d'un mauvais œil) mais peuvent leur offrir l'hospitalité si les personnages se montrent sympathiques.\\
	2  & Mis en difficulté par la raréfaction de ses proies habituelles, un ours blanc s'aventure dans les terres.\\
	3  & Les personnages doivent contourner une crevasse dont s'échappe en continu du gaz naturel… terriblement inflammable.\\
	4  & Un camion militaire passe à proximité. L'armée recherche un groupe ayant déserté et n'a pas vraiment de temps pour les personnages au-delà d'un éventuel quiproquo.\\
	5  & Un ou une criminelle blessée est abandonnée par ses camarades, à la merci des éléments.\\
	6  & Un équipement important des personnages tombe en panne ou est abîmé par le gel ou la neige (GPS, tente, armement, etc.).\\
	\end{tabularx}
\end{table}

\subsection{Résolution}

À moins que le groupe se retrouve bloqué, vos personnages devraient rattraper les criminels dans la ville alors que ces derniers cherchent une façon de s'enfuir.
Tout le monde sera probablement exténué mais il faudra donner un dernier coup de collier pour récupérer le noyau de polonium et capturer les voleurs.

\vfill
\illustration{lighthouse}
\vfill
