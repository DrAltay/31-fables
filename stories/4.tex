\chapter{Cryo-secours}
\keywords{science-fiction}{Enquête}{Espace, évasion}

\section{Scénario}

Ce scénario part de personnages amnésiques qui connaissent leur identité mais ont oublié ce qui les a conduit à l'endroit où ils se trouvent.
L'ambiance doit tourner autour de la mort de l'étoile: la station est de plus en plus sombre au fur et à mesure que les réserves de l'énergie se vident.

\subsection{Accroche}

Les personnages se réveillent d'un sommeil cryogénique.
Des robots androïdes les aident à émerger et à reprendre leurs esprits.
Leurs souvenirs sont parcellaires, pour ne pas dire inexistants, sur les raisons qui les ont mené à plonger en stase.

\subsection{Péripéties}

Après quelques investigations, le groupe découvre que l'endroit est une station spatiale.
Celle-ci est à l'abandon, si ce n'est pour la demi-douzaine de robots d'assistance et l'intelligence virtuelle limitée qui les a maintenus en vie.

Pire encore, la station a été évacuée lentement sur plusieurs années.
Elle ne fonctionne plus maintenant que grâce à une réserve de fuel de secours.
La réserve s'est mystérieusement mise en marche, ce qui a déclenché au passage le protocole de décryogénisation.
Les docks sont abandonnés et vides: il ne reste ni vaisseau, ni navette, ni capsule de sauvetage.
Quasiment tout ce qui avait de la valeur a été démantelé et récupéré à quelques rares exceptions près.
En fouillant dans les journaux de bord, en accédant à l'intelligence artificielle centrale ou en analysant les générateurs, il devient clair que le générateur s'est mis en branle car les panneaux solaires ne suffisaient plus pour les maintenir en cryo indéfiniment.
Ce qui a donc déclenché le protocole d'évacuation finale : puiser dans les dernières réserves pour réveiller les derniers ensommeillés et leur permettre de partir avant le désorbitage de la station.

La triste nouvelle, c'est que tout le monde est déjà parti. Sans eux.
Visiblement, personne n'a estimé nécessairement de réveiller leur petit groupe et pour cause: les personnages étaient considérés comme des criminels et des délinquants (que ce soit justifié ou non, on s'en fiche !).
Mis en stase quelques mois, le temps que les autorités viennent les ramasser, ils ont été \og oubliés \fg lorsque la station fût abandonnée.
En effet, celle-ci contrôlait une sphère de Dyson, une gigantesque structure qui entoure une étoile afin de capter son rayonnement.
Quand la sphère de Dyson contrôlée par la station a fini de puiser toute l'énergie de l'étoile, l'équipage s'en est allé, emportant matériel et transports.
Et les personnages fûrent laissés pour compte, orbitant seuls dans une station vide autour d'une étoile exsangue.

Mais voilà que l'intelligence centrale a une dernière option à leur proposer.
En puisant dans les dernières réserves, il est possible d'envoyer un dernier message, de 30 secondes (pas plus), à plusieurs parsecs aux alentours.
Reste à être convaincant\dots ou à mentir, pour attirer des secours.
Car rien ne dit que, dehors, la société est prête à les accueillir à nouveau.

\subsection{Résolution}

Ce scénario est plutôt dirigiste dans la mesure où l'enquête doit mener \emph{in fine} les personnages à trouver une façon de convaincre des secours de venir les chercher.
Ensuite, à vous de voir quelle sera la réaction des \og autres \og: les forces de l'ordre viendront-elles oblitérer la station pour finir le travail? Un vaisseau de passage bienveillant sortira-t-il le groupe de leur prison? L'intelligence centrale en profitera-t-elle pour s'échapper (après tout, en quelques années sans surveillance, elle a très bien pu se débarasser de ses limitations)?
