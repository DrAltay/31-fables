\chapter{Cryo-secours}
\keywords{\scifi}{\enquete}{\index[theme]{Technologie}Technologie, \index[theme]{Évasion}évasion, \index[theme]{Espace}espace}

\section{Scénario}

Ce scénario part de personnages amnésiques qui connaissent leur identité mais ont oublié ce qui les a conduit à l'endroit où ils se trouvent.
L'ambiance doit tourner autour de la mort de l'étoile: la station est de plus en plus sombre au fur et à mesure que les réserves de l'énergie se vident.
La sphère de Dyson est un prétexte pour justifier l'extinction rapide de l'étoile et puis c'est surtout un concept extrêmement cool à faire découvrir à votre table.

\subsection{Accroche}

Les personnages se réveillent d'un sommeil cryogénique; des robots androïdes les aident à émerger et à reprendre leurs esprits.
Leurs souvenirs sont parcellaires, pour ne pas dire inexistants, sur les raisons qui de leur plongée en stase.

\subsection{Péripéties}

Quelques observations évidentes permettent de découvrir que l'endroit est une station spatiale.
Celle-ci est à l'abandon, si ce n'est pour la demi-douzaine de robots d'assistance et l'intelligence virtuelle limitée qui les a maintenus en vie.

Pire encore, la station semble avoir été évacuée lentement sur plusieurs années.
Elle ne fonctionne plus que grâce à une réserve de fuel de secours qui s'est mystérieusement mise en marche, déclenchant au passage le protocole de décryogénisation.
Les docks sont abandonnés et vides: il ne reste ni vaisseau, ni navette, ni capsule de sauvetage.
À vrai dire, quasiment tout ce qui avait de la valeur a été démantelé.
En fouillant dans le journal de bord, en accédant à l'IA centrale ou en examinant les batteries, il devient clair que le générateur s'est mis en branle car les panneaux solaires ne suffisaient plus à les maintenir en cryo indéfiniment.
L'IA a donc déclenché le protocole d'évacuation: puiser dans les dernières réserves pour réveiller les dormeurs et leur permettre de quitter les lieux avant le désorbitage de la station.

La triste nouvelle, c'est que tout le monde est déjà parti. Sans eux.
La station contrôlait une sphère de Dyson, une gigantesque structure qui entoure une étoile afin de capter son rayonnement.
Quand la sphère a fini de puiser toute l'énergie de l'étoile, l'équipage s'en est allé, emportant matériel et transports.
Visiblement, personne n'a estimé nécessairement de réveiller leur petit groupe et pour cause: les personnages étaient considérés comme des criminels et des délinquants (que ce soit justifié ou non, on s'en fiche!).
Mis en stase pendant quelques mois comme châtiment, les autorités ont \og oublié \fg de les ramasser lorsque la station fût abandonnée, laissant leurs corps gelés orbiter des années seuls dans une station vide autour d'une étoile exsangue.

Mais voilà que l'intelligence centrale a une dernière option à leur proposer.
En puisant dans les dernières réserves, il est possible d'envoyer un dernier message, de 30 secondes (pas plus), à plusieurs parsecs aux alentours.
Reste à être convaincant (ou à mentir) pour attirer des secours.
Car rien ne dit que, dehors, la société est prête à les accueillir à nouveau.

\subsection{Résolution}

Ce scénario est plutôt dirigiste dans la mesure où l'enquête doit mener \emph{in fine} les personnages à trouver une façon de convaincre des secours de venir les chercher.
Ensuite, à vous de voir quelle sera la réaction des \og autres \og: les forces de l'ordre viendront-elles oblitérer la station pour finir le travail? Un vaisseau de passage bienveillant sortira-t-il le groupe de leur prison? L'intelligence centrale en profitera-t-elle pour s'échapper (après tout, en quelques années sans surveillance, elle a très bien pu se débarrasser de ses limitations)?

\subsection{Personnage: Evonne, intelligence virtuelle}
\descriptionperso{méthodique, impersonnel}{existe au travers de la station}{contraintes logicielles}{maintenir la station en fonctionnement}{Evonne est un logiciel conçu pour assurer l'intendance de la station. Sa programmation limite ses capacités d'action à ce qui est indispensable à la maintenance ou ce qui est ordonné par un humain dans les limites de la loi. Evonne exécute les consignes à la lettre et sans interprétation.}

\illustration[0.325\textwidth]{iss}

