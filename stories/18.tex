\chapter{Le bouc doit brûler !}
\keywords{Cyberpunk}{Action}{Crime, complot, humoristique}

\section{Scénario}

Ce scénario trouve ses racines dans la tradition du bouc de Gävle\footnote{\url{https://fr.wikipedia.org/wiki/Bouc_de_G%C3%A4vle}}.
Il s'agit d'une tradition suédoise consistant à ériger un bouc de Noël en paille (\emph{julbock}) sur la place centrale.
Pour une raison inexpliquée, le bouc est régulièrement incendié, parfois le jour même de sa construction.
Quelques décennies dans le futur, y a-t-il vraiment une raison pour que cela change?

\subsection{Accroche}

Dans un futur proche, les personnages sont des mercenaires stars, des spécialistes dans leurs domaines respectifs (\emph{hacking}, intrusion, armes à feu, escroquerie\dots).
Une organisation mystérieuse les embauche à prix d'or pour mener une mission urgente et au vu des tarifs, ça a l'air important.
Alors que les fêtes de Noël s'approchent, on les fait transporter discrètement jusqu'en Suède.

\blockquote{Gävle. 110 000 habitants. Une chèvre. Brûlez-là.}

Voilà en substance ce que leur contact leur dira ce soir là.

\subsection{Péripéties}

Mettre le feu à une grande chèvre en bois qui, comme le veut la tradition remontant à 1966, est érigée par l'association des commerces de la ville pour célébrer Noël.
Pourquoi? Parce qu'il le faut, voilà pourquoi.
Si le groupe ne parvient pas à l'incendier, à défaut il faut la détruire quoi qu'il en coûte.

Depuis des décennies, après que la chèvre ait été endommagée des dizaines de fois, la mairie et le syndicat d'initiatives ont instauré des mesures de précautions de plus en plus drastiques: le bouc est protégé par un dispositif impressionnant\dots ou pas.
Certes, il y a bien une double clôture, des caméras, deux patrouilles (municipales, non armées) qui circulent en permanence mais rien qui ne soit un jeu d'enfant pour les personnages.

En termes d'ambiance, le décalage doit être permanent: la sécurité est assurée par d'adorables agents municipaux qui s'arrêtent pour que les enfants caressent leur chien, les caméras sont reliées à un cabanon même pas fermé à clé, la caserne des pompiers organise des paris sur la longévité du bouc.
Si les personnages réussissent, le club de sciences naturelles de la ville construira une chèvre miniature (dont le commanditaire exigera bien sûr la destruction immédiate).
Et surtout, malgré les protestations des personnages et l'incongruité de la situation, le commanditaire doit toujours prendre la mission parfaitement au sérieux. Le bouc doit brûler!

\subsection{Résolution}

La réussite de la mission est secondaire et n'a que peu d'importance.
À la fin du scénario, vos joueuses doivent surtout s'interroger sur \og pourquoi \fg quelqu'un dépenserait autant d'argent pour une bête tradition.
Et c'est encore mieux si la réponse est seulement \og parce que c'est comme ça \fg!
