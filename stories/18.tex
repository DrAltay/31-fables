\chapter{Le bouc doit brûler !}
\keywords{\cyberpunk}{\action}{\index[theme]{Crime}Crime, \index[theme]{Humour}humoristique}

\section{Scénario}

Ce scénario tourne autour d'une radition suédoise consistant à ériger un bouc de Noël en paille de \SI{10}{\meter} (\emph{julbock}) sur la place centrale.
Pour une raison inexpliquée, le bouc de la ville de Gävle\footnote{\url{https://fr.wikipedia.org/wiki/Bouc_de_G\%C3\%A4vle}} est incendié chaque année, parfois le jour même de sa construction.
Quelques décennies dans le futur, y a-t-il vraiment une raison pour que cela change?

\subsection{Accroche}

Dans un futur proche, les personnages sont des mercenaires stars, des spécialistes dans leurs domaines respectifs (\emph{hacking}, intrusion, armes à feu, escroquerie\dots).
Une organisation mystérieuse les embauche à prix d'or pour mener une mission urgente. Au vu des tarifs, ça a l'air important.
On les fait transporter discrètement jusqu'en Suède à l'approche des fêtes de Noël.
Voilà en substance ce que leur contact leur dira ce soir là:

\blockquote{Gävle. 110 000 habitants. Une chèvre. Brûlez-là.}

\subsection{Péripéties}

La mission consiste donc à mettre le feu à une chèvre en bois qui, comme le veut la tradition remontant à 1966, est érigée par l'association des commerces pour célébrer Noël.
À défaut d'un incendie, le commanditaire se contentera de sa destruction.

La chèvre ayant déjà été endommagée des dizaines de fois, le syndicat d'initiatives a pris des précautions de plus en plus drastiques: le bouc est protégé par un dispositif impressionnant (ou pas).
Certes, il y a bien une double clôture, des caméras, deux patrouilles (municipales, non armées) qui circulent en permanence mais rien qui ne soit un jeu d'enfant pour les personnages.

Le décalage avec les \emph{runs} habituels est frappant: les vigiles s'arrêtent dans leur ronde pour que les enfants caressent le chien, les caméras sont reliées à un cabanon même pas fermé à clé, la caserne des pompiers organise des paris sur la longévité du bouc.
Une fois la mission accomplie, le club de sciences construira une chèvre miniature (dont le commanditaire exigera bien sûr la destruction immédiate).
Surtout, malgré les protestations des personnages et l'incongruité de la situation, le commanditaire doit toujours prendre la mission parfaitement au sérieux.

\subsection{Résolution}

La réussite de la mission n'a que peu d'importance.
À la fin du scénario, vos joueuses doivent surtout s'interroger sur \og pourquoi \fg quelqu'un dépenserait autant d'argent pour une bête tradition.
Lubie de millionnaire ou coup publicitaire de génie?

\illustration[0.2\textwidth]{goat}
