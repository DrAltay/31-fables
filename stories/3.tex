\chapter{Leurre de paix}
\keywords{Générique}{Action}{Diplomatie, évasion, trahison}

\section{Scénario}

\subsection{Commentaire}

Ce scénario est adaptable à de très nombreux cadre de jeux.
L'intrigue tourne autour de trois factions: la faction Bleue, la faction Verte et la faction Rouge.
Le seul prérequis est le suivant: les factions Rouge et Verte sont en guerre et la faction Bleue est \og neutre \fg mais profite de la situation d'une manière ou d'une autre.

\subsection{Accroche}

Les personnages sont de jeunes dignitaires envoyés comme émissaires auprès du royaume Vert pour négocier une trêve dans la guerre l'opposant au royaume Rouge.
Ou tout du moins tel est ce que leur ont raconté les ministres.

\subsection{Péripéties}

En réalité, l'état-major a déjà acheté la paix.
Afin de sceller l'armistice et en guise de bonne foi,le gouvernement a décidé d'envoyer quelques enfants de bonne famille qui seront gardés otages comme \og caution \fg.
Les personnages sont reçus dans le faste et le luxe dû à leur statut mais, à la nuit tombée, leur escorte s'éclipse et les laisse à la merci des Verts.
On les réveille au beau milieu de la nuit pour les conduire à leur futur de lieu de captivité.

C'est néanmoins sans compter l'intervention des agents de la faction Bleue qui vont tout faire pour saboter la paix.
Parce que leur faction vend des armes aux deux camps ou que l'affaiblissement des deux autres nations sert leurs plans à long terme, les Bleus profitent de la situation telle qu'elle est et n'ont aucune envie que le conflit s'arrête.
Ainsi, des espions et espionnes déguisées en loyalistes Verts -- mais en réalité au service des Bleus -- vont tenter de libérer les otages et ainsi relancer la guerre.
Reste à savoir si les personnages, mis devant le fait accompli, choisiront d'accepter leur sort et se sacrifieront pour entériner l'armistice ou, au contraire, feront des pieds et des mains pour s'échapper, quite à envenimer la situation.
À moins que des négociations à couteaux tirés au milieu des agents doubles -- réels ou soupçonnés -- soient encore possible\dots

\subsection{Résolution}

Plusieurs issues sont envisageables à cette situation en fonction des décisions des personnages:

\subsubsection{Les personnages refusent l'aide des Bleus}
\begin{itemize}
	\item Les personnages refusent l'aide des Bleus et acceptent leur captivité: la paix est achetée, en espérant que la faction Rouge joue le jeu.
	\item Les personnages refusent l'aide des Bleus et s'évadent: la guerre continue, ils deviennent très sûrement \emph{persona non grata} aux yeux de leur gouvernement.
	\item Les personnages refusent l'aide des Bleus et négocient la paix par eux-mêmes: une trêve est envisageable, surtout si les manigances des Bleus sont exposées au grand jour.
\end{itemize}

\subsubsection{Les personnages acceptent l'aide des Bleus}
\begin{itemize}
	\item Les personnages acceptent l'aide des Bleus mais sont tout de même capturés: la paix est achetée, sauf si la faction Rouge pense que ce sont les Verts qui ont tenté de faire évader le groupe en dépit de leur accord.
	\item Les personnages acceptent l'aide des Bleus et s'évadent: la guerre continue, surtout si les Bleus ont réussi à se faire passer pour les Verts jusqu'au bout.
\end{itemize}
