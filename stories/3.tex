\chapter{Leurre de paix}
\keywords{\generique}{\intrigue}{\index[theme]{Diplomatie}Diplomatie, \index[theme]{Évasion}évasion, \index[theme]{Trahison}trahison}

\section{Scénario}

Ce scénario peut s'adapter à de nombreux cadres de jeux.
L'intrigue tourne autour de trois factions que l'on nommera -- pour simplifier -- la faction Bleue, la faction Verte et la faction Rouge.
La situation initiale est la suivante: les factions Rouge et Verte sont en guerre et la faction Bleue est \og neutre \fg mais profite de la situation d'une manière ou d'une autre.

\subsection{Accroche}

Les personnages sont de jeunes dignitaires envoyés comme émissaires auprès du royaume Vert pour négocier une trêve dans la guerre l'opposant au royaume Rouge.

\subsection{Péripéties}

Ou tout du moins tel est ce que leur ont raconté les ministres.
En réalité, l'état-major a déjà acheté la paix.
En guise de bonne foi pour sceller l'armistice, le gouvernement a promis des enfants de bonne famille qui serviront de \og caution \fg en tant qu'otages.
Les personnages sont ainsi reçus dans le faste et le luxe dû à leur statut mais, une fois la nuit tombée, leur escorte s'éclipse et les laisse à la merci des Verts.
On les réveille au beau milieu de la nuit pour les conduire à leur futur de lieu de captivité.

C'est néanmoins sans compter l'intervention des agents de la faction Bleue qui vont tout faire pour saboter la paix.
Parce que leur faction vend des armes aux deux camps ou que l'affaiblissement des deux autres nations sert leur stratégie militaire à long terme, les Bleus profitent de la situation telle qu'elle est et n'ont aucune envie que le conflit s'éteigne.
Ainsi, des espions et espionnes déguisées en loyalistes Verts -- mais en réalité au service des Bleus -- vont tenter de libérer les otages pour relancer la guerre.
Reste à savoir si les personnages, mis devant le fait accompli, choisiront d'accepter leur sort et se sacrifieront pour entériner l'armistice ou, au contraire, feront des pieds et des mains pour s'échapper, quitte à envenimer la situation.
À moins que des négociations à couteaux tirés au milieu des agents doubles -- réels ou soupçonnés -- ne soient encore possibles\dots

\subsection{Résolution}

\subsubsection{Les personnages refusent l'aide des Bleus\dots}
\begin{itemize}
	\item et acceptent leur captivité: la paix est achetée, en espérant que la faction Rouge joue le jeu.
	\item et s'évadent: la guerre continue, les membres du groupe sont \emph{persona non grata} aux yeux de leur gouvernement.
	\item et négocient la paix par eux-mêmes: une trêve est envisageable, surtout si les manigances des Bleus sont exposées au grand jour.
\end{itemize}

\subsubsection{Les personnages acceptent l'aide des Bleus\dots}
\begin{itemize}
	\item mais sont tout de même capturés: la paix est achetée, sauf si la faction Rouge pense que ce sont les Verts qui ont tenté de faire évader le groupe en dépit de leur accord.
	\item et s'évadent: la guerre continue, surtout si les Bleus ont réussi à se faire passer pour les Verts jusqu'au bout.
\end{itemize}

L'idée est de laisser une grande latitude aux joueuses et de permettre tous les retournements de situation.
L'intrigue est un sac de nœuds et de nouvelles alliances de circonstances peuvent très bien apparaître selon ce qui vous arrange le plus!

\subsection*{Lieu: le château d'Awiti}

\begin{tcolorbox}[colback=black!1!white]
Situé à la frontière entre les factions Rouge et Vert, le château est l'endroit parfait pour des tractations en terrain \og neutre \fg.
Le château s'élève sur une île au milieu du lac dont il a pris le nom. Cela le rend facile à défendre mais peu commode à ravitailler.
De fait, la forteresse est abandonnée depuis des lustres.
Cet inconvénient est aussi son plus grand atout: la structure regorge d'alcôves cachées, de souterrains méconnus, de passages secrets et de cryptes oubliées pour comploter en toute tranquillité, assassiner un rival ou se déplacer sans être vu.
\end{tcolorbox}

\illustration[0.45\textwidth]{betrayal}
