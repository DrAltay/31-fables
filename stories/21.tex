\chapter{Cacheurs de trésors}
\keywords{\pirate{}-- \scifi}{\aventure}{\index[theme]{Voyage}Voyage, \index[theme]{Trahison}trahison}

\section{Scénario}

Cette aventure renverse l'habituelle chasse au trésor des histoires de pirate.
Cette fois-ci, ce sont vos personnages qui vont se décarcasser pour planquer le butin!
Il est très facile d'adapter ce scénario dans un cadre \emph{space opera}, la piraterie dans l'espace fonctionne parfaitement bien.

\subsection{Accroche}

Les personnages sont des corsaires au service d'un État, d'un gouvernement voire d'une grande organisation commerciale.
Lors d'une expédition particulièrement fructueuse, leur équipage a fait l'acquisition d'un précieux butin.
Afin d'éviter que celui-ci ne tombe entre de mauvaises mains, les ordres sont de se rendre dans un lieu difficile d'accès et peu connu afin d'y dissimuler les richesses pillées.

\subsection{Péripéties}

L'enjeu est double pour le vaisseau. Non seulement faut-il réussir à se faufiler dans un endroit à l'accès volontairement compliqué choisi par leur capitaine (archipel aux courants vicieux, planète protégée par une ceinture d'astéroïdes chaotique\dots), mais il faut surtout y parvenir sans attirer l'attention des concurrents.

Une fois les écueils évités et le butin déchargé, leur tâche est loin d'être terminée.
Il reste encore à transbahuter les encombrants coffres jusqu'à un endroit bien caché dans les tréfonds d'un désert rocheux inhospitalier.
Qui plus est, le groupe devra composer avec le moral instable de l'équipage, qui ne comprend pas bien les raisons poussant leurs commanditaires à abandonner là de tels trésors.
La mutinerie ne sera donc jamais bien loin et les personnages devraient prendre garde à ne pas mettre le feu aux poudres!

Enfin, en admettant que le groupe parvienne à dissimuler le butin, leur retour sain et sauf à bon port n'a rien de garanti.
En effet, durant leur périple, un des membres de l'équipage a discrètement transmis des informations à une faction rivale.
C'est lorsque les personnages s'apprêtent à regagner leur vaisseau qu'ils réalisent que celui-ci est sous le contrôle de leur antagoniste.
En dépit de l'épuisement accumulé, il faudra bien les confronter pour reprendre au large.

\subsection{Résolution}

Comme bien souvent, le voyage est plus important que sa conclusion!
Quelques pistes toutefois pour amener votre groupe à bon port.

Vos personnages, après avoir fait tant d'efforts pour cacher le trésor, risquent de ne pas vouloir céder aux menaces de leur ennemi.
Peut-être qu'un accord enrichissant pour les deux parties aurait raison de leur patriotisme.
Si ce n'est pas le cas, alors dans la pure tradition des films de forban, qu'ils reprennent leur navire par la force!
Une fois leur mission accomplie, leurs employeurs pourront grassement les récompenser, par exemple avec leur propre vaisseau et équipage ou en les autorisant à rentrer d'exil.

\subsection{Lieu: la Baleine Fière}

\begin{tcolorbox}[colback=black!1!white]
Vaisseau de commerce reconverti en navire pirate, la Baleine Fière est un bel ouvrage, quoiqu'un peu daté.
Son équipage compte une trentaine de personnes pour un fonctionnement optimal.
De loin, il est impossible de réaliser qu'il s'agit d'un vaisseau pirate: les ouvertures pour les canonnières sont dissimulées par des peintures en trompe-l'œil et les drapeaux identifient le navire comme un marchand de différentes nationalités selon la situation.
Toutes les fioritures ont été retirées et le confort est plutôt spartiate afin d'alléger la coque et de rendre la Baleine plus rapide que ses proies.
La proue est décorée de deux grands yeux et la poupe prend la forme d'une queue baleine plongeant dans l'océan.
\end{tcolorbox}

\vfill
\illustration[0.6\textwidth]{pirate_ship}
\vfill
