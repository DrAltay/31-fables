\chapter{Cacheurs de trésors}
\keywords{Pirate/SF}{Aventure}{}

\section{Scénario}

Commentaire

\subsection{Accroche}

Pour une raison ou pour une autre les personnages sont des corsaires au service d'un État ou d'un gouvernement.
Afin d'éviter que leur large butin ne tombe entre de mauvaises mains, des ordres leur intiment de choisir un lieu difficile d'accès et peu connu pour dissimuler les richesses pillées.

\subsection{Péripéties}

Il s'agira donc d'échapper à l'attention des autres vaisseaux, se faufiler dans un archipel aux courants vicieux (ou une ceinture d'astéroïdes chaotique) avant d'atteindre l'île choisie par leur capitaine.

Une fois sur place, leur tâche est loin d'être terminée puisqu'il faudra encore transbahuter les encombrants coffres jusqu'à un endroit bien caché et qui plus est accessible seulement après un long chemin périlleux dans un désert rocheux inhospitalier.
Qui plus est, la mutinerie ne sera jamais bien loin car peu parmi l'équipage acceptent d'abandonner de tels trésors\dots

En admettant que le groupe parvienne à dissimuler les richesses, leur retour ne sera pas garanti.
Car, discrètement, un des membres de l'équipage a transmis des informations à la faction ennemie!
Lorsque les personnages s'apprêtent à retourner à leur vaisseau, celui-ci est sous le contrôle de leurs antagonistes et, en dépit de l'épuisement accumulé, il faudra bien les confronter pour reprendre au large.
À moins qu'un accord enrichissant pour les deux parties n'ait raison de leur patriotisme\dots

\subsection{Résolution}

\illustration{pirate_ship}
