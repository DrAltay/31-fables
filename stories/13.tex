\chapter{Les armes sous les cendres}
\keywords{Post-apocalyptique}{Aventure}{Exploration, combat}

\section{Scénario}

Ce scénario post-apocalyptique doit avoir lieu bien après le cataclysme (il faut que les personnages puissent voyager).
La nature exacte des catastrophes ayant frappé la région est laissée à votre imagination.
Initialement, cette aventure a été pensée pour le jeu de rôle Cendres.

\subsection{Accroche}

Alors que la société se reconstruit lentement des ruines de l'ancien monde, les personnages partent en mission pour le nouveau gouvernement de Rennes.
Leur but: piller une ancienne cache d'armes repérée sur une vieille carte d'état-major.

\subsection{Péripéties}

Le problème principal est que la cache se trouve dans une forêt infestée de bandits de grands chemins.
Ces derniers ont l'habitude de dépouiller les caravanes marchandes et les voyageurs qui empruntent la route.

En dépit de la discrétion (réelle ou supposée) des personnages, le groupe fera face à une opposition coriace.
En effet, une traîtresse, à la solde du duc d'Angers, est intervenue pour saboter l'opération.
Quelqu'un parmi les rangs bretons a soudoyé les brigands pour les empêcher d'accéder à l'armement.

Heureusement pour le groupe, les bandits n'ont pas encore réussi à ouvrir eux-mêmes la cache d'armes.
Celle-ci est en réalité un bunker de l'armée française remontant aux années 50.
Des armes, des explosifs et des munitions y ont été entreposées jusqu'aux années 80.
Il a ensuite été abandonné mais son contenu est encore en bon état.

\subsection{Résolution}

Si, contre toutes attentes, les personnages sortent de cette forêt avec les armes, alors c'est sur la route du retour que l'opposition angevine tentera de les éliminer une bonne fois pour toute, avant que la cargaison ne puisse atteindre sa destination.
Mais une fois de retour à Rennes, le groupe sera grassement récompensé et tout particulièrement si la traîtresse été débusquée!

\illustration{forest}
