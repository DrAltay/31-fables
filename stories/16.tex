\chapter{Sauvagerîle}
\keywords{\medfan}{\aventure}{\index[theme]{Exploration}Exploration, \index[theme]{Horreur}{horreur}, \index[theme]{Malédiction}malédiction}

\section{Scénario}

\subsection{Accroche}

Les personnages explorent une île nouvelle pour le compte d'une riche guilde marchande.

\subsection{Péripéties}

Leur mission consiste à inventorier les ressources naturelles, qu'elles soient minérales ou bien intégrées à la faune et la flore.
Les périls sont légions car l'île semble habitée par de nombreuses chimères, mélanges inattendus d'animaux du continent (ours ailés, amphibiens produisant de l'électricité, singes aux griffes imbibées d'acide…).

\paragraph{L'exploration} Au cours de sa progression vers l'intérieur des terres, le groupe est averti à plusieurs reprises des dangers qui s'y trouvent.
D'abord en tombant sur les traces des campements - abandonnés - des équipes qui les ont précédées.
Puis par une malheureuse victime enserrée dans les lianes barbelées d'un arbre colossal leur racontant comment les uns après les autres, ses camarades ont changé, comme remplis par une violence bestiale à mesure que l'île se révélait à eux.
Enfin, d'étranges hybrides à l'air humanoïde se mettront en travers de leur chemin.

\paragraph{La nature de l'île} Alors que les personnages commencent à subir les effets de cette lente (mais inévitable) corruption, les indices convergeront pour les faire aboutir à une conclusion: l'île transforme tout ce qui se trouve en son aire d'influence, décuplant la sauvagerie intérieure du vivant avant de l'assimiler à son écosystème.
Les humains, nouvellement accostés, ne font pas exception.
Il sera alors temps de faire demi-tour et d'espérer réussir à quitter l'île sans sombrer aux pulsions bestiales ou périr de la main d'une des créatures qu'elle utilise pour rassembler ses nouvelles victimes.

\subsection{Résolution}

Comme pour le scénario du bouchon du Darién, l'objectif est surtout de revenir en vie!

\vfill
\illustration[0.7\textwidth]{island}
\vfill
