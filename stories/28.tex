\chapter{Les dévalisés du rail}
\keywords{\western}{\action}{\index[theme]{Crime}Crime, \index[theme]{Technologie}technologie}

\section{Scénario}

Tout le monde connaît cette histoire récurrente des westerns: un train bloqué par un arbre tombé sur les rails, un groupe de hors-la-loi qui débarque à cheval et qui dépouille les voyageurs.
Un braquage de train tout ce qu'il y a de plus classique.
Et si aujourd'hui vos joueuses s'intéressaient non pas au contenu, mais au contenant?

\subsection{Accroche}

La compagnie ferroviaire Southern Pacific Transportation Company embauche les personnages pour\dots{} voler un train.

\subsection{Péripéties}

La S.P.T.C. est en concurrence avec l'Union Pacific, l'autre grande compagnie ferroviaire de l'Ouest américain.
En perte de vitesse, elle aurait bien besoin d'un atout pour s'étendre plus vite que sa rivale.
Ce qui est plaisant, c'est que l'atout désiré lui est tombé tout cuit dans le bec dans le journal de la veille: le constructeur Baldwin Locomotive Works prévoit une démonstration de leur nouvelle locomotive dans trois jours.
Leur nouvelle machine à vapeur est, d'après l'article dans le canard, plus puissante, puis rapide, plus économe et plus facile d'entretien que toutes les autres. Et la démonstration aura lieu sur les rails de l'Union Pacific sur le tout nouveau tronçon reliant Reno, dans le Nevada, à Sacramento en Californie.

La direction de la S.P.T.C. paie grassement le groupe pour détourner la loco, par exemple en la faisant changer d'aiguillage avant l'arrivée en gare de Sacramento pour l'envoyer vers Stockton où il serait facile de la transborder sur le fleuve et de la faire disparaître.
Les personnages auront également pour mission de trouver (ou de convaincre) un autre fabricant de machine à vapeur susceptible de copier le prototype volé et de fournir la S.P.T.C. en contrefaçons à des tarifs bien plus faibles que Baldwin.
Grâce à cet avantage matériel et au coup à l'image de l'Union Pacific que le vol provoquera, la Southern Pacific espère bien supplanter sa rivale.

Le point noir de l'affaire, c'est que le prototype de la Baldwin est encore bien loin de fonctionner comme attendue, en dépit des promesses de son fabricant.
La locomotive a de nombreuses défaillances qui la rendront difficile à manœuvrer quand les personnages en auront pris le contrôle.
Qui plus est, Baldwin, s'étant trouvé dos au mur, a engagé une bande de hors-la-loi pour saboter la démonstration.
L'objectif de la manœuvre: empocher l'assurance et gagner quelques mois de répit, le temps de faire des ajustements sur son prototype sans craindre l'ire de ses investisseurs.

Le tableau de la page \pageref{table:rail} liste les événements qui vont mettre des bâtons (de dynamite) dans les roues de vos joueuses.

\subsection{Résolution}

Jonglez entre les deux sources d'un problème: d'un côté, la locomotive est incontrôlable, de l'autre, des bandits se préparent à faire sauter les rails.
Que les personnages arrivent à détourner la locomotive ou pas ne devrait pas avoir d'importance à la fin du scénario, le simple fait de sauver sa peau quand les circonstances se sont liguées contre soi est déjà une belle réussite!

\begin{table}
	\caption{Événements aléatoires durant le détournement du train}
	\label{table:rail}
	\colortablerows
	\begin{tabularx}{\textwidth}{cX}
	d6 & Événement\\
	1  & Deux hors-la-loi infiltrés par Baldwin dans le train tentent de prendre le contrôle du wagon. Après avoir dépouillé les passagers, ils se dirigent vers la loco pour la saboter.\\
	2  & L'orientation des rails à la prochaine jonction n'est pas la bonne et la personne en charge de l'aiguillage ne répond pas aux signaux du train.\\
	3  & La connexion entre le wagon contenant la réserve de charbon et la loco a été endommagée et ne pas tarder à lâcher.\\
	4  & Les hors-la-loi embauchés par Baldwin ont abattu deux troncs d'arbre sur les rails. Pas sûr que la locomotive y résiste.\\
	5  & Un des membres de l'équipage qui s'occupe de la loco enclenche la purge du moteur, vidant les réserves d'eau sur les rails. Sans eau, pas de vapeur, sans vapeur, pas de traction!\\
	6  & Un notable invité pour assister à la démonstration en tant que passager fait un malaise et a besoin de soins urgemment.\\
	\end{tabularx}
\end{table}

\vfill
\illustration[0.4\textwidth]{train}
\vfill
