\chapter{Les dévalisés du rail}
\keywords{Western}{Action}{Crime}

\section{Scénario}

Commentaire

\subsection{Accroche}

La compagnie ferroviaire Southern Pacific Transportation Company embauche les personnages pour… voler un train. Oui oui, le train lui-même, pas son contenu.

\subsection{Péripéties}

 La S.P.T.C. est en concurrence avec l'Union Pacific et a besoin d'un atout pour s'étendre plus vite que sa rivale. Et cet atout lui est tombé tout cuit dans le bec : le constructeur Baldwin Locomotive Works prévoit une démonstration de leur nouvelle locomotive sur les rails de l'Union Pacific.

Charge ensuite au groupe de trouver un autre fabricant peu scrupuleux qui serait susceptible de copier le prototype volé et de fournir la S.P.T.C. en contrefaçons à des tarifs bien plus faibles que Baldwin. Grâce à cet avantage matériel et au coup à l'image de l'Union Pacific que le vol provoquera, la Southern Pacific espère bien supplanter sa rivale.

C'est sans compter que le prototype de la Baldwin est encore bien loin de fonctionner comme attendue, malgré les promesses de son fabricant. La locomotive a de nombreuses défaillances et Baldwin a payé un groupe de bandits pour saboter la démonstration afin de toucher l'assurance et gagner quelques mois de répit.

\subsection{Résolution}

\illustration{train}
