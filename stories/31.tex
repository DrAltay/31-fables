\chapter{L'indépendance gronde}
\keywords{\steampunk}{\action}{\index[theme]{Technologie}Technologie, \index[theme]{Guerre}guerre, \index[theme]{Évasion}évasion}

\section{Scénario}

Coincés au milieu d'une guerre d'indépendance, les personnages vont se retrouver mêlés de force au conflit quand leur commanditaire se fait enlever.

\subsection{Accroche}

Les personnages accompagnent l'explorateur et scientifique Alexander Von Humboldt au milieu de l'orage de Catatumbo pour une de ses folles expériences.
Humboldt expérimente avec le stockage de l'énergie électrique dans des énormes piles à cristaux qui lui servent ensuite à alimenter ses autres inventions.
Leur bateau d'exploration mouille ainsi au large du lac de Maracaibo et est équipé de paratonnerres pour capter la foudre.

\subsection{Péripéties}

Toutefois, le Venezuela est pendant ce temps en proie à la guerre d'indépendance.
Le soulèvement de l'Amérique latine face à l'empire espagnol a débuté il y a quelques années et ne montre aucun signe de faiblesse.
En réaction, les espagnols ont installé un blocus à l'embouchure du golfe du Venezuela, retenant de fait prisonnier le bateau de Von Humboldt et son équipage.
L'inventeur prend la situation avec bonhommie, cette captivité forcée lui permettant d'avancer sur ses dernières expériences.
Son projet est à deux doigts d'être terminé et il a déjà accumulé des quantités faramineuses d'énergie électrique dans ses \og piles \fg.

Malheureusement, une telle puissance énergétique peut avoir de nombreuses applications militaires.
Les rumeurs les plus folles circulent sur les travaux de Humboldt et attisent les convoitises.
Profitant d'une sortie du savant pendant un ravitaillement, les indépendantistes colombiens et vénézuéliens le kidnappent afin de le contraindre à rejoindre leur révolution.
En parallèle, le bateau est pris d'assaut par un commando cherchant à récupérer ses formidables expériences (que l'état-major indépendantiste espère transformer en armes).
Aux personnages de résister ou de s'échapper du blocus, si possible avec le baron Humboldt.
Mais ce n'est pas si simple car celui-ci, derrière une neutralité de façade, n'est pas insensible aux arguments des indépendantistes\dots

\subsection{Résolution}

\begin{itemize}
	\item Le groupe est indépendantiste: une fois Humboldt retrouvé, il \og suffira \fg aux personnages d'aider l'opération vénézuélienne pour briser le blocus espagnol et rentrer chez eux. 
	\item Le groupe est impérialiste: récupérer Humboldt et ses cristaux chargés d'énergie est une excellente façon de monnayer un passage sûr pour reprendre la mer. Le baron ne sera toutefois pas très coopératif.
	\item Le groupe n'est loyal qu'à Humboldt: le principal sera de récupérer le baron (au diable ses expériences!) et de trouver un moyen de s'enfuir sans être fait prisonnier par un camp ni par l'autre. Plus facile à dire qu'à faire.
\end{itemize}

Les principaux obstacles de cet aventure sont les limites que les joueuses vont s'imposer elles-mêmes.
Humboldt (et ses cristaux) sont un McGuffin convoité par trois factions différentes (le groupe, les espagnols, les indépendantistes).
En fonction de la solution choisie par le groupe, la méthode de passage du blocus variera, tout comme l'antagoniste principal.

Les espagnols ont une armada à leur disposition mais leur puissance est principalement navale.
Ils contrôlent le golfe.
En revanche, les indépendantistes connaissent le terrain et ont le soutien de la population. Ils sont moins bien équipés mais en territoire allié.


\subsection{Lieu: l'orage permanent de Catatumbo}

\begin{tcolorbox}[colback=black!1!white]
Au-dessus de l'embouchure du fleuve Catatumbo, à sa jonction avec le lac Maracaibo se trouve un orage permanent.
Environ 150 nuits par an, pendant dix heures, la foudre frappe le lac de puissants éclairs orangés.
Ceux-ci éclairent à des dizaines de kilomètres à la ronde, permettant d'y voir presque comme en plein jour.
Étrangement, malgré la fréquence des éclairs (quasiment toutes les quinze secondes), le tonnerre est presque inaudible\dots
\end{tcolorbox}

\vfill
\illustration[0.3\textwidth]{coil}
\vfill
