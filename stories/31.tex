\chapter{L'indépendance gronde}
\keywords{Steampunk}{Action}{Science, exploration, guerre}

\section{Scénario}

Commentaire

\subsection{Accroche}

Les personnages accompagnent l'explorateur et scientifique Alexander Von Humboldt au milieu de l'orange permanent de Catatumbo pour une expérience scientifique. Humboldt expérimente avec la foudre et le stockage de l'énergie dans des énormes piles à cristaux pour ensuite alimenter ses autres expériences. Le bateau a été ainsi positionné dans le lac de Maracaibo et muni de paratonnerres gigantesques pour capter les éclairs.

\subsection{Péripéties}

Toutefois, le Venezuela est pendant ce temps en proie à la guerre d'indépendance, l'Amérique latine s'étant soulevée contre l'empire espagnol. Ainsi, les espagnols ont installé un blocus à l'embouchure du golfe du Venezuela. Le bateau et l'équipage est donc de fait retenu captif, bien que Humboldt soit satisfait de la situation car ses expériences avancent bien et son projet est à deux doigts d'être terminé.

Malheureusement, profitant d'une sortie de Humboldt pendant un ravitaillement, les indépendantistes colombiens et vénézueliens kidnappent le savant afin de le contraindre à rejoindre leur révolution. En parallèle, le bateau est attaqué pour récupérer ses formidables expériences que l'état-major espère transformer en armes. Aux personnages de résister (ou de s'échapper), avec ou sans le baron Humboldt qui, sous ses prétentions de neutralité, se ne montrera pas insensible aux arguments des indépendantistes…

\subsection{Résolution}

\illustration{coil}
